\documentclass[10pt,fleqn]{article}
%\usepackage{graphicx}


\setlength{\topmargin}{-.75in}
\addtolength{\textheight}{2.00in}
\setlength{\oddsidemargin}{.00in}
\addtolength{\textwidth}{.75in}

\title{Math 4610 Lecture Notes \\
            \ \\
        Errors and Computing the Rate of Convergence
  \footnote{These notes are part of an Open Resource Educational project
            sponsored by Utah State University}}

\author{Joe Koebbe}

\nofiles

\pagestyle{empty}

\setlength{\parindent}{0in}

% new math commands


\setlength{\oddsidemargin}{-0.25in}
\setlength{\evensidemargin}{-0.25in}
\setlength{\textwidth}{6.75in}
\setlength{\headheight}{0.0in}
\setlength{\topmargin}{-0.25in}
\setlength{\textheight}{9.00in}

\makeindex

\usepackage{mathrsfs}

%\usepackage[pdftex]{graphicx}
\usepackage{epstopdf}

\newcounter{beans}

\newcommand{\ds}{\displaystyle}
\newcommand{\limit}[2]{\displaystyle\lim_{#1\to#2}}

\newcommand{\binomial}[2]{\ \left( \begin{array}{c}
                                  #1 \\
                                  #2
                                 \end{array}
                            \right) \
                         }
\newcommand{\ExampleRule}[2]
  {
  \noindent
  \rule{\linewidth}{1pt}
  \begin{example}
    #1
    \label{#2}
  \end{example}
  \rule{\linewidth}{1pt}
  \vskip0.125in
  }

\newcommand{\defbox}[1]
  {
   \ \\
   \noindent
   \setlength\fboxrule{1pt}
   \fbox{
        \begin{minipage}{6.5in}
          #1
        \end{minipage}
        }
   \ \\
  }
\newcommand{\verysmallworkbox}[1]
  {
   \ \\
   \noindent
   \setlength\fboxrule{1pt}
   \fbox{
        \begin{minipage}{6.5in}
           #1
           \ \\
           \vskip0.5in \ \\
           \ \\
        \end{minipage}
        }
   \ \\
  }
\newcommand{\smallworkbox}[1]
  {
   \ \\
   \noindent
   \setlength\fboxrule{1pt}
   \fbox{
        \begin{minipage}{6.5in}
           #1
           \ \\
           \vskip2.5in \ \\
           \ \\
        \end{minipage}
        }
   \ \\
  }
\newcommand{\halfworkbox}[1]
  {
   \ \\
   \noindent
   \setlength\fboxrule{1pt}
   \fbox{
        \begin{minipage}{6.5in}
           #1 \hfill
           \ \\
           \vskip3.25in \ \\
           \ \\
        \end{minipage}
        }
   \ \\
  }
\newcommand{\largeworkbox}[1]
  {
   \ \\
   \noindent
   \setlength\fboxrule{1pt}
   \fbox{
        \begin{minipage}{6.5in}
           #1
           \ \\
           \vskip7.5in \ \\
           \ \\
        \end{minipage}
        }
   \ \\
  }
\newcommand{\flexworkbox}[2]
  {
   \ \\
   \noindent
   \setlength\fboxrule{1pt}
   \fbox{
        \begin{minipage}{6.5in}
           #1
           \ \\

           \vskip#2 \ \\
           \ \\
        \end{minipage}
        }
   \ \\
  }


% symbols for sets of numbers

\newcommand{\natnumb}{$\cal N$}
\newcommand{\whonumb}{$\cal W$}
\newcommand{\intnumb}{$\cal Z$}
\newcommand{\ratnumb}{$\cal Q$}
\newcommand{\irrnumb}{$\cal I$}
\newcommand{\realnumb}{$\cal R$}
\newcommand{\cmplxnumb}{$\cal C$}

% misc. commands

\newcommand{\mma}{{\it Mathematica}}
\newcommand{\sech}{\mbox{ sech}}
 
\newtheorem{theorem}{Theorem}
\newtheorem{example}{Example}
\newtheorem{definition}{Definition}
\newtheorem{problem}{Problem}

\setcounter{secnumdepth}{2}
\setcounter{tocdepth}{4}


\begin{document}
\maketitle
\newpage

%%%%%%%%%%%%%%%%%%%%%%%%%%%%%%%%%%%%%%%%%%%%%%%%%%%%%%%%%%%%%%%%%%%%%%%%%%%%%%%%
%%%%%%%%%%%%%%%%%%%%%%%%%%%%%%%%%%%%%%%%%%%%%%%%%%%%%%%%%%%%%%%%%%%%%%%%%%%%%%%%
\vskip0.1in\hrule\vskip0.1in
\noindent
{\bf Errors and Rate of Convergence: Errors When a Solution is Known}
\vskip0.1in\hrule\vskip0.1in
\noindent
The basic mission in this section of the course notes is to determine
convergence rates for various algorithms we use to compute approximate solutions
in many mathematical problems. For example, we can consider applying the
Bisection method for finding roots of real-valued functions of a real variable.
The algorithm for computing approximate locations of roots amounts to an initial
application of the Intermediate Value Theorem to a continuous function and then
proceed by repeatedly halving the interval and determining which of the two
subintervals contains a root for the function. In addition to the Bisection
method, other methods, like Newton's method, can be applied to the same problem.

Given that there are multiple methods that can be applied to obtain an 
approximation for a root, it makes sense to determine which of these methods
is \lq\lq best\rq\rq\ for a given function. One measurement of the effectiveness
of a given method is whether or not the algorithm will converge to any type of
solution. Another measure involves the rate at which a given method converges.
For example, in the root finding method, if the function in question is
continuous and an interval, $[a, b]$, exists such that $f(a)\cdot f(b)<0$, the
Bisection method is guaranteed to converge. For Newton's method, the function
needs to be twice continuously differentiable, the derivative cannot be zero at
or near the root, and finally the initial guess at a solution must be
sufficiently close to the root. The last condition is something that causes
some issues with both Newton's method and the Secant method. In fact, if the
initial guess is not close enough, there is no guarantee that a root can be
found. From this point of view, the Bisection method seems to be better.
However, the Bisection method reduces the error in successive iteration slower
than either Newton's method of the Secant method.

In this rest of this section of notes, we will investigate how to compute the
rate of convergence for various methods to solve the same problem. Readers
should realize that the same issue of convergence applies to just about any
computational method for solving just about any type of mathematical problem.
From root finding to solving linear systems of equations to image processing
algorithms to numerical solution of differential equations, multiple methods
have been developed to solve the same problem. We need ways to determine which
method will be best for a given problem and conditions.

It should also be noted that there are a lot of other considerations that
cause practical problem. For example, limitations on processors, CPU speed, 
and storage on a computer. We will consider these issues at another point in
the course. 
%%%%%%%%%%%%%%%%%%%%%%%%%%%%%%%%%%%%%%%%%%%%%%%%%%%%%%%%%%%%%%%%%%%%%%%%%%%%%%%%
%%%%%%%%%%%%%%%%%%%%%%%%%%%%%%%%%%%%%%%%%%%%%%%%%%%%%%%%%%%%%%%%%%%%%%%%%%%%%%%%
\vskip0.1in\hrule\vskip0.1in
\noindent
{\bf Errors and Rate of Convergence: Definition of Convergence and Rate of
Convergence}
\vskip0.1in\hrule\vskip0.1in
\noindent
To start a discussion about rate of convergence, we need some definitions. We
have already defined the absolute and relative error. The formulas for these
are
$$
  \makebox{absolute error} = | x - x^* |
$$
and 
$$
  \makebox{relative error} = {{| x - x^* |}\over{|x^*|}}
$$
for approximations and solution values that are real numbers. Suppose now that
an algorithm (say the Bisection method) has created a sequence of approximations
$$
  \lbrace x_0, x_1, x_2, \ldots, x_k, \ldots \rbrace
$$
Note that it will be assumed that the approximations, $x_k$, are computed using
the incerasing index. The error in each of these approximations is given by
$$
  e_k = x_k - x^*
$$
or
$$
  e_k = {{ x_k - x^* }\over{x^*}}
$$
depending on which error makes sense.

The result will be a sequence of errors of the form
$$
  \lbrace |e_0|, |e_1|, |e_2|, \ldots, |e_k|, \ldots \rbrace
$$
The questions we need to answer as computational mathematicians are:

{\bf Convergence Questions:}
\begin{list}{$\bullet$}{\usecounter{beans} \parsep=0pt \listparindent=0pt
\topsep=0pt \rightmargin=.35in \leftmargin=.35in \labelsep=5 pt
\itemsep=2pt}
  \item {\bf Convergence to the Solution:} For convergence we must have the
        following limit.
        $$
          \lim_{k\rightarrow\infty} |e_k| = 0
        $$  
  \item {\bf Reduction of Error per Iteration:} How much does the sequence of
        approximations reduce in each iteration? We can state this as the
        following inequality.
        $$
          |e_{k+1}| < |e_k|
        $$
  \item {\bf Rate of Convergence:} If the sequence of errors converges to zero,
        how fast is the convergence?  We will measure this using the following
        inequality.
        $$
          |e_{k+1}| < C\cdot |e_k|^r
        $$
\end{list}
The power, $r$ in the last question about rate of convergence is called the rate
of convergence for a given algorithm.

There are several situations that we will need to consider. Firet, if we decide
to test the algorithm, it makes sense to test the algorithm for a problem where
the solution is known. For example, suppose we want to test some of our root
finding methods. Then we may choose a function that has a root at a specifi
point and then apply the algorithm through writing a piece of computer code.
Then use the results to make sure we generate a sequence of approximations that
converge to a known value. The second situation is when we are faced with
computing a root when the exact solution is not known. The rest of notes in
this section will treate these problems.
%%%%%%%%%%%%%%%%%%%%%%%%%%%%%%%%%%%%%%%%%%%%%%%%%%%%%%%%%%%%%%%%%%%%%%%%%%%%%%%%
%%%%%%%%%%%%%%%%%%%%%%%%%%%%%%%%%%%%%%%%%%%%%%%%%%%%%%%%%%%%%%%%%%%%%%%%%%%%%%%%
\vskip0.1in\hrule\vskip0.1in
\noindent
{\bf Errors and Rate of Convergence: Case 1. An Exact Solution is Known}
\vskip0.1in\hrule\vskip0.1in
\noindent
For the case when we are testing a bit of computer code we will assume that the
sequence of approximations is given and that given an exact solution, we can
compute the errors from the given information

As an example, let's apply this idea to the use of the Bisection method and a
specific root finding problem.






%%%%%%%%%%%%%%%%%%%%%%%%%%%%%%%%%%%%%%%%%%%%%%%%%%%%%%%%%%%%%%%%%%%%%%%%%%%%%%%%
%%%%%%%%%%%%%%%%%%%%%%%%%%%%%%%%%%%%%%%%%%%%%%%%%%%%%%%%%%%%%%%%%%%%%%%%%%%%%%%%
\vskip0.1in\hrule\vskip0.1in
\noindent
{\bf Errors and Rate of Convergence: Case 2. An Exact Solution is Not Known}
\vskip0.1in\hrule\vskip0.1in

As an example, let's apply this idea to the use of the Bisection method.






%%%%%%%%%%%%%%%%%%%%%%%%%%%%%%%%%%%%%%%%%%%%%%%%%%%%%%%%%%%%%%%%%%%%%%%%%%%%%%%%
%%%%%%%%%%%%%%%%%%%%%%%%%%%%%%%%%%%%%%%%%%%%%%%%%%%%%%%%%%%%%%%%%%%%%%%%%%%%%%%%
\vskip0.1in\hrule\vskip0.1in
\noindent
{\bf Errors and Rate of Convergence: Stability of an Algorithm under Small
Perturbation }
\vskip0.1in\hrule\vskip0.1in









%%%%%%%%%%%%%%%%%%%%%%%%%%%%%%%%%%%%%%%%%%%%%%%%%%%%%%%%%%%%%%%%%%%%%%%%%%%%%%%%
%%%%%%%%%%%%%%%%%%%%%%%%%%%%%%%%%%%%%%%%%%%%%%%%%%%%%%%%%%%%%%%%%%%%%%%%%%%%%%%%





\vskip0.1in\hrule\vskip0.1in
\vfill
\eject







%%%%%%%%%%%%%%%%%%%%%%%%%%%%%%%%%%%%%%%%%%%%%%%%%%%%%%%%%%%%%%%%%%%%%%%%%%%%%%%%
%%%%%%%%%%%%%%%%%%%%%%%%%%%%%%%%%%%%%%%%%%%%%%%%%%%%%%%%%%%%%%%%%%%%%%%%%%%%%%%%
\vskip0.1in\hrule\vskip0.1in
\noindent
{{\bf GitHub Primer for Math 4610 at USU:} Setting up a Repository} 
\vskip0.1in\hrule\vskip0.1in
\noindent
Once you log in, you will need to build a repository for use in the class and
to turn in homework and completed tasks and projects. For the course, you will
create a repository named the following:
\begin{verbatim}

    math4610

\end{verbatim}
Use only the characters above and using the following rules:
\begin{enumerate}
  \item Use only lower case characters - github is case senesitive.
  \item Do not put any blanks in the name of the repository.
\end{enumerate}
Note that the instructor will use only this repository name in looking for your
work.
\vskip0.1in\hrule\vskip0.1in
\vfill
\begin{figure}[h]
\centering
\includegraphics{../images/github_02.png}
\caption{{Screenshot} taken using {\bf Snip \& Sketch}. This is an app on
         my Windows 10 box}
\end{figure}
\eject
%%%%%%%%%%%%%%%%%%%%%%%%%%%%%%%%%%%%%%%%%%%%%%%%%%%%%%%%%%%%%%%%%%%%%%%%%%%%%%%%
%%%%%%%%%%%%%%%%%%%%%%%%%%%%%%%%%%%%%%%%%%%%%%%%%%%%%%%%%%%%%%%%%%%%%%%%%%%%%%%%
\vskip0.1in\hrule\vskip0.1in
\noindent
{{\bf Github Primer for Math 4610 at USU:} List the Contents of the Home
    Directory} 
\vskip0.1in\hrule\vskip0.1in
\noindent
If you have an account on GitHub, you will already know a lot about these
things. However, when you are logged in you will see the main screen with any
repositories you may already have created. We will go through the steps to
build and name repositories in the next few pages.

\vskip0.1in\hrule\vskip0.1in
\vfill
\begin{figure}[h]
\centering
\includegraphics{../images/github_03.png}
\caption{{Screenshot} taken using {\bf Snip \& Sketch}. This is an app on
         my Windows 10 box}
\end{figure}
\eject
%%%%%%%%%%%%%%%%%%%%%%%%%%%%%%%%%%%%%%%%%%%%%%%%%%%%%%%%%%%%%%%%%%%%%%%%%%%%%%%%
%%%%%%%%%%%%%%%%%%%%%%%%%%%%%%%%%%%%%%%%%%%%%%%%%%%%%%%%%%%%%%%%%%%%%%%%%%%%%%%%
\end{document}
