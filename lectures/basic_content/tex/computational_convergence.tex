\documentclass[10pt,fleqn]{article}
%\usepackage{graphicx}


\setlength{\topmargin}{-.75in}
\addtolength{\textheight}{2.00in}
\setlength{\oddsidemargin}{.00in}
\addtolength{\textwidth}{.75in}

\title{Math 4610 Lecture Notes \\
            \ \\
        Errors and Computing the Rate of Convergence
  \footnote{These notes are part of an Open Resource Educational project
            sponsored by Utah State University}}

\author{Joe Koebbe}

\nofiles

\pagestyle{empty}

\setlength{\parindent}{0in}

% new math commands


\setlength{\oddsidemargin}{-0.25in}
\setlength{\evensidemargin}{-0.25in}
\setlength{\textwidth}{6.75in}
\setlength{\headheight}{0.0in}
\setlength{\topmargin}{-0.25in}
\setlength{\textheight}{9.00in}

\makeindex

\usepackage{mathrsfs}

%\usepackage[pdftex]{graphicx}
\usepackage{epstopdf}

\newcounter{beans}

\newcommand{\ds}{\displaystyle}
\newcommand{\limit}[2]{\displaystyle\lim_{#1\to#2}}

\newcommand{\binomial}[2]{\ \left( \begin{array}{c}
                                  #1 \\
                                  #2
                                 \end{array}
                            \right) \
                         }
\newcommand{\ExampleRule}[2]
  {
  \noindent
  \rule{\linewidth}{1pt}
  \begin{example}
    #1
    \label{#2}
  \end{example}
  \rule{\linewidth}{1pt}
  \vskip0.125in
  }

\newcommand{\defbox}[1]
  {
   \ \\
   \noindent
   \setlength\fboxrule{1pt}
   \fbox{
        \begin{minipage}{6.5in}
          #1
        \end{minipage}
        }
   \ \\
  }
\newcommand{\verysmallworkbox}[1]
  {
   \ \\
   \noindent
   \setlength\fboxrule{1pt}
   \fbox{
        \begin{minipage}{6.5in}
           #1
           \ \\
           \vskip0.5in \ \\
           \ \\
        \end{minipage}
        }
   \ \\
  }
\newcommand{\smallworkbox}[1]
  {
   \ \\
   \noindent
   \setlength\fboxrule{1pt}
   \fbox{
        \begin{minipage}{6.5in}
           #1
           \ \\
           \vskip2.5in \ \\
           \ \\
        \end{minipage}
        }
   \ \\
  }
\newcommand{\halfworkbox}[1]
  {
   \ \\
   \noindent
   \setlength\fboxrule{1pt}
   \fbox{
        \begin{minipage}{6.5in}
           #1 \hfill
           \ \\
           \vskip3.25in \ \\
           \ \\
        \end{minipage}
        }
   \ \\
  }
\newcommand{\largeworkbox}[1]
  {
   \ \\
   \noindent
   \setlength\fboxrule{1pt}
   \fbox{
        \begin{minipage}{6.5in}
           #1
           \ \\
           \vskip7.5in \ \\
           \ \\
        \end{minipage}
        }
   \ \\
  }
\newcommand{\flexworkbox}[2]
  {
   \ \\
   \noindent
   \setlength\fboxrule{1pt}
   \fbox{
        \begin{minipage}{6.5in}
           #1
           \ \\

           \vskip#2 \ \\
           \ \\
        \end{minipage}
        }
   \ \\
  }


% symbols for sets of numbers

\newcommand{\natnumb}{$\cal N$}
\newcommand{\whonumb}{$\cal W$}
\newcommand{\intnumb}{$\cal Z$}
\newcommand{\ratnumb}{$\cal Q$}
\newcommand{\irrnumb}{$\cal I$}
\newcommand{\realnumb}{$\cal R$}
\newcommand{\cmplxnumb}{$\cal C$}

% misc. commands

\newcommand{\mma}{{\it Mathematica}}
\newcommand{\sech}{\mbox{ sech}}
 
\newtheorem{theorem}{Theorem}
\newtheorem{example}{Example}
\newtheorem{definition}{Definition}
\newtheorem{problem}{Problem}

\setcounter{secnumdepth}{2}
\setcounter{tocdepth}{4}


\begin{document}
\maketitle
\newpage

%%%%%%%%%%%%%%%%%%%%%%%%%%%%%%%%%%%%%%%%%%%%%%%%%%%%%%%%%%%%%%%%%%%%%%%%%%%%%%%%
%%%%%%%%%%%%%%%%%%%%%%%%%%%%%%%%%%%%%%%%%%%%%%%%%%%%%%%%%%%%%%%%%%%%%%%%%%%%%%%%
\vskip0.1in\hrule\vskip0.1in
\noindent
{\bf Errors and Rate of Convergence: An Introduction}
\vskip0.1in\hrule\vskip0.1in
\noindent
The basic mission in this section of the course notes is to determine
convergence rates for various algorithms we use to compute approximate solutions
in many mathematical problems. For example, we can consider applying the
Bisection method for finding roots of real-valued functions of a real variable.
The algorithm for computing approximate locations of roots amounts to an initial
application of the Intermediate Value Theorem to a continuous function and then
proceed by repeatedly halving the interval and determining which of the two
subintervals contains a root for the function. In addition to the Bisection
method, other methods, like Newton's method, can be applied to the same problem.

Given that there are multiple methods that can be applied to obtain an 
approximation for a root, it makes sense to determine which of these methods
is \lq\lq best\rq\rq\ for a given function. One measurement of the effectiveness
of a given method is whether or not the algorithm will converge to any type of
solution. Another measure involves the rate at which a given method converges.
For example, in the root finding method, if the function in question is
continuous and an interval, $[a, b]$, exists such that $f(a)\cdot f(b)<0$, the
Bisection method is guaranteed to converge. For Newton's method, the function
needs to be twice continuously differentiable, the derivative cannot be zero at
or near the root, and finally the initial guess at a solution must be
sufficiently close to the root. The last condition is something that causes
some issues with both Newton's method and the Secant method. In fact, if the
initial guess is not close enough, there is no guarantee that a root can be
found. From this point of view, the Bisection method seems to be better.
However, the Bisection method reduces the error in successive iteration slower
than either Newton's method of the Secant method.

In this rest of this section of notes, we will investigate how to compute the
rate of convergence for various methods to solve the same problem. Readers
should realize that the same issue of convergence applies to just about any
computational method for solving just about any type of mathematical problem.
From root finding to solving linear systems of equations to image processing
algorithms to numerical solution of differential equations, multiple methods
have been developed to solve the same problem. We need ways to determine which
method will be best for a given problem and conditions.

It should also be noted that there are a lot of other considerations that
cause practical problem. For example, limitations on processors, CPU speed, 
and storage on a computer. We will consider these issues at another point in
the course. 
\vskip0.1in\hrule\vskip0.1in
\newpage
%%%%%%%%%%%%%%%%%%%%%%%%%%%%%%%%%%%%%%%%%%%%%%%%%%%%%%%%%%%%%%%%%%%%%%%%%%%%%%%%
%%%%%%%%%%%%%%%%%%%%%%%%%%%%%%%%%%%%%%%%%%%%%%%%%%%%%%%%%%%%%%%%%%%%%%%%%%%%%%%%
\vskip0.1in\hrule\vskip0.1in
\noindent
{\bf Errors and Rate of Convergence: Definitions of Convergence and Rate of
Convergence}
\vskip0.1in\hrule\vskip0.1in
\noindent
To start a discussion about rate of convergence, we need some definitions. We
have already defined the absolute and relative error. The formulas for these
are
$$
  \makebox{absolute error} = | x - x^* |
$$
and 
$$
  \makebox{relative error} = {{| x - x^* |}\over{|x^*|}}
$$
for approximations and solution values that are real numbers. Suppose now that
an algorithm (say the Bisection method) has created a sequence of approximations
$$
  \lbrace x_0, x_1, x_2, \ldots, x_k, \ldots \rbrace
$$
Note that it will be assumed that the approximations, $x_k$, are computed using
the incerasing index. The error in each of these approximations is given by
$$
  e_k = x_k - x^*
$$
or
$$
  e_k = {{ x_k - x^* }\over{x^*}}
$$
depending on which error makes sense in the computation.

The result will be a sequence of errors of the form
$$
  \lbrace e_0, e_1, e_2, \ldots, e_k, \ldots \rbrace
$$
The questions we need to answer as computational mathematicians are:

\vskip0.1in\hrule\vskip0.1in
\begin{list}{$\bullet$}{\usecounter{beans} \parsep=0pt \listparindent=0pt
\topsep=0pt \rightmargin=.35in \leftmargin=.35in \labelsep=5 pt
\itemsep=2pt}
  \item {\bf Convergence to the Solution:} For convergence can we determine if
        the error in our computation, $e_k$, goes to zero as the iteration
        index, $k$, tends to infinity. Thus if the iterations are converging to
        the exact solution,
        $$
          \lim_{k\rightarrow\infty} |e_k| = 0
        $$  
  \item {\bf Reduction in Error per Iteration:} How much does the sequence of
        errors reduce for each iteration? We can state this in the following
        way.
        $$
          |e_{k+1}| < |e_k|
        $$
        If this inequality cannot be established, then the approximations may
        not converge to the exact solution.
  \item {\bf Rate of Convergence:} If the sequence of errors converges to zero,
        how fast is the convergence?  We will measure this using the following
        inequality.
        $$
          |e_{k+1}| < C\cdot |e_k|^r
        $$
        The positive constant $C$ cannot depend on the index, $k$, and the
        power, $r$, is called the convergence rate. The convergence rate , $r$,
        must be greater than zero to guarantee convergence.
\end{list}
\vskip0.1in\hrule\vskip0.1in
\noindent
There are several situations that we will need to consider. Firet, if we decide
to test the algorithm, it makes sense to test the algorithm for a problem where
the solution is known. For example, suppose we consider the very simple equation
$$
  e^x - \pi = 0 \ \ \ \ \rightarrow \ \ \ \ f(x) = e^x - \pi = 0
$$
We can eaily compute a solution using simple algebraic techniques. The single
root for this function is
$$
  x = \ln(\pi)
$$
The solution is an irrational number. However, any hand held calculator can be
used to compute an accurate approximation of the solution. This solution can
slso easily be intrduced into a root finding code.

Note that we can choose a simpler problem with integer roots, say
$$
  f(x) = x^2 - 5\ x + 6 = 0
$$
with one of two roots $x^*=3$ to search for. As soon as the computer starts to
work, there will be a loss of precision since roundoff error will creep in to
the iterations. So, using an example like
$$
  f(x) = e^x - \pi = 0
$$
is as good as any. This example will be used to generate results to make sure
we generate a sequence of approximations that converge to the known solution
given above.
%%%%%%%%%%%%%%%%%%%%%%%%%%%%%%%%%%%%%%%%%%%%%%%%%%%%%%%%%%%%%%%%%%%%%%%%%%%%%%%%
%%%%%%%%%%%%%%%%%%%%%%%%%%%%%%%%%%%%%%%%%%%%%%%%%%%%%%%%%%%%%%%%%%%%%%%%%%%%%%%%
The second situation is when we are faced with approximating a root when the
exact solution is not known. A third problem involves finding multiple unknown
roots for a given function. The multiple roots problem will be discussed in the
last section in these notes. The rest of notes in this section will treat these
problems.
\vskip0.1in\hrule\vskip0.1in
\newpage
%%%%%%%%%%%%%%%%%%%%%%%%%%%%%%%%%%%%%%%%%%%%%%%%%%%%%%%%%%%%%%%%%%%%%%%%%%%%%%%%
%%%%%%%%%%%%%%%%%%%%%%%%%%%%%%%%%%%%%%%%%%%%%%%%%%%%%%%%%%%%%%%%%%%%%%%%%%%%%%%%
\vskip0.1in\hrule\vskip0.1in
\noindent
{\bf Errors and Rate of Convergence: Case 1. An Exact Solution is Known}
\vskip0.1in\hrule\vskip0.1in
\noindent
For this case, we will use the example given previously where
$$
  f(x) = e^x - \pi = 0
$$
with $x=\ln(\pi)$. Suppose that the closed interval, $[-2.2, 6.8]$, is chosen
to test the Bisection code (written in Fortran that you might be required to
translate).

\begin{verbatim}

          program main
    c
    c variables for the test
    c ----------------------
    c
          real a, b, c, pi, tol
          real ek, ekp1
          real a11, a12, a22, b1, b2
          real aval, rval
          real detval
          integer n
    c
    c initialize the interval and tolerance desired
    c ---------------------------------------------
    c
          a = -2.2
          b = 6.8
          tol = 0.0000001
    c
    c initialize storage for the linear regression to determine the
    c computational rate of convergence - note that the normal equations
    c matrix is symmetric.
    c --------------------
    c
          a11 = 0.0
          a12 = 0.0
          a22 = 0.0
    c
          b1 = 0.0
          b2 = 0.0
    c
    c compute the number of bisections needed for the given tolerance
    c ---------------------------------------------------------------
    c
          n = - alog10( tol / ( b - a ) ) / alog10(2.0) - 1
          a11 = dble(n)
    c
    c print out a header for the output
    c ---------------------------------
    c
          print *," Iter:   ek      ekp1     log(ek)      log(ekp1)"
          print *," -----------------------------------------------"
    c
    c compute an upper bound on the initial error
    c -------------------------------------------
    c
          ek = b - a
    c
    c do the iterations
    c -----------------
    c
          do 1 i=1, n
    c
    c compute the midpoint of the interval
    c ------------------------------------
    c
             c = 0.5 * ( a + b )
    c
    c test for the root
    c -----------------
    c
             if(f(a)*f(c) .lt. 0.0) then
                b = c
             else
                a = c
             endif
    c
    c compute an upper bound on the error for the root location
    c ---------------------------------------------------------
    c
             ekp1 = b - a
    c
    c print out the next line in the table of errors
    c ----------------------------------------------
    c
             print *, i, ek, ekp1, alog10(ek), alog10(ekp1)
    c
    c update the error variables in the normal equations
    c --------------------------------------------------
    c
             a12 = a12 + alog10(ek)
             a22 = a22 + alog10(ek) * alog10(ek)
             b1 = b1 + alog10(ekp1)
             b2 = b2 + alog10(ek) * alog10(ekp1)
    c
    c update the errors for the next step
    c -----------------------------------
    c
             ek = ekp1
    c
        1 continue
    c
    c compute the solution using the inverse of the 2x2 matrix for the linear
    c reduction of the matrix
    c -----------------------
    c
          detval = a11 * a22 - a12 * a12
          aval = ( a22 * b1 - a12 * b2 ) / detval
          rval = ( a11 * b2 - a12 * b1 ) / detval
    c
    c output the results - note that the constant needs to be exponentiated
    c ---------------------------------------------------------------------
    c
          print *, "shift constant:   ", exp(aval)
          print *, "rate of convergence:   ", rval
    c
    c this code will be used to test convergence rates for the bisection method
    c -------------------------------------------------------------------------
    c
          stop
          end
    c
    c a simple function used to test the bisection method
    c ---------------------------------------------------
    c
          real function f(x)
          real pi
          pi = 3.141592653589793
          f = exp(x) - pi
          return
          end

\end{verbatim}
\vskip0.1in\hrule\vskip0.1in
\noindent
Since the function is continuous on the given interval and
$$
  f(-2.2)\times f(6.8)<0
$$
the Bisection method will produce an approximate solution that is accurate up to
machine precision. The relationship we need for analyzing convergence involve
the current error and the previous error. The code above produces five columns
of output. The first column is the iteration index, the second is the previous 
error and the third is the current error. The output looks like the following.
The last two columns will be used below after performing a transformation on the
error data.

The code produces the following output
\vskip0.1in\hrule\vskip0.1in
\noindent
\begin{verbatim}

        Iter:     ek               ekp1           log(ek)          log(ekp1)
        --------------------------------------------------------------------
           1   9.00000000       4.50000000      0.954242527      0.653212488
           2   4.50000000       2.25000000      0.653212488      0.352182508
           3   2.25000000       1.12500012      0.352182508       5.11525720E-02
           4   1.12500012      0.562500060       5.11525720E-02 -0.249877438
           5  0.562500060      0.281250000     -0.249877438     -0.550907493
           6  0.281250000      0.140625000     -0.550907493     -0.851937473
           7  0.140625000       7.03125000E-02 -0.851937473      -1.15296745
           8   7.03125000E-02   3.51562500E-02  -1.15296745      -1.45399749
           9   3.51562500E-02   1.75781250E-02  -1.45399749      -1.75502753
          10   1.75781250E-02   8.78906250E-03  -1.75502753      -2.05605745
          11   8.78906250E-03   4.39453125E-03  -2.05605745      -2.35708737
          12   4.39453125E-03   2.19726562E-03  -2.35708737      -2.65811753
          13   2.19726562E-03   1.09863281E-03  -2.65811753      -2.95914745
          14   1.09863281E-03   5.49316406E-04  -2.95914745      -3.26017737
          15   5.49316406E-04   2.74658203E-04  -3.26017737      -3.56120753
          16   2.74658203E-04   1.37329102E-04  -3.56120753      -3.86223745
          17   1.37329102E-04   6.86645508E-05  -3.86223745      -4.16326761
          18   6.86645508E-05   3.43322754E-05  -4.16326761      -4.46429729
          19   3.43322754E-05   1.71661377E-05  -4.46429729      -4.76532745
          20   1.71661377E-05   8.58306885E-06  -4.76532745      -5.06635761
          21   8.58306885E-06   4.29153442E-06  -5.06635761      -5.36738729
          22   4.29153442E-06   2.14576721E-06  -5.36738729      -5.66841745
          23   2.14576721E-06   1.07288361E-06  -5.66841745      -5.96944761
          24   1.07288361E-06   4.76837158E-07  -5.96944761      -6.32163000
          25   4.76837158E-07   2.38418579E-07  -6.32163000      -6.62265968

      shift constant:     0.741362631
      rate of convergence:      1.00143242

\end{verbatim}
\vskip0.1in\hrule\vskip0.1in
\noindent
It is easy to see that the error is getting smaller and as shown in the size of
the error appears to be headed to zero. With roundoff error, we already know
that the computational error, $e_k$, is actually bounded below. Another question
to be answered is how fast is the convergence?
\vskip0.1in\hrule\vskip0.1in
\newpage
%%%%%%%%%%%%%%%%%%%%%%%%%%%%%%%%%%%%%%%%%%%%%%%%%%%%%%%%%%%%%%%%%%%%%%%%%%%%%%%%
%%%%%%%%%%%%%%%%%%%%%%%%%%%%%%%%%%%%%%%%%%%%%%%%%%%%%%%%%%%%%%%%%%%%%%%%%%%%%%%%
\vskip0.1in\hrule\vskip0.1in
\noindent
{\bf Errors and Rate of Convergence: Analysis of the Output Data}
\vskip0.1in\hrule\vskip0.1in
\noindent
As shown in the output from the code in the previous section show a rate of
convergence. The next step is to determine a way to analyze the results. The
relationship between successive steps is given by
$$
  | e_{k+1} | \leq C | e_k |^r
$$
It should be noted that in the data from the example, there are 25 conditions
on the rate relationship. That is,
\vskip0.1in\hrule\vskip0.1in
\begin{center}
  \begin{tabular}{lcl}
    $| e_1 | \leq C\ | e_0 |^r$ & $\rightarrow$
                              & $9.00000000 \leq C ( 4.50000000 )^r$ \\
    $| e_2 | \leq C\ | e_1 |^r$ & $\rightarrow$ 
                              & $4.50000000 \leq C ( 2.25000000 )^r$ \\
    $| e_3 | \leq C\ | e_1 |^r$ & $\rightarrow$
                              & $2.25000000 \leq C ( 1.25000000 )^r$ \\
    \ & $\vdots$ & \ \\
    $| e_{25} | \leq C\ | e_{24} |^r$ & $\rightarrow$
                              & $4.76837158E-07 \leq C (2.38418579E-07)^r$
  \end{tabular}
\end{center}
\vskip0.1in\hrule\vskip0.1in
The goal is to compute values for $C$ and $r$ that define the relationship. So,
the input and output for the relationship are the values for $e_k$ and $e_{k+1}$
are known and we want to obtain $C$ and $r$ to complete the definition of the
error relationship. For the example above, this means we have 25 conditions for
the two unknowns, $C$ and $r$. This is a classic over-determined system. Due to
roundoff error and other machine precision issues, the best we can do is fit
the errors data to the parameters $C$ and $r$.
\vskip0.1in\hrule\vskip0.1in
\newpage
%%%%%%%%%%%%%%%%%%%%%%%%%%%%%%%%%%%%%%%%%%%%%%%%%%%%%%%%%%%%%%%%%%%%%%%%%%%%%%%%
%%%%%%%%%%%%%%%%%%%%%%%%%%%%%%%%%%%%%%%%%%%%%%%%%%%%%%%%%%%%%%%%%%%%%%%%%%%%%%%%
\vskip0.1in\hrule\vskip0.1in
\noindent
{\bf Errors and Rate of Convergence: Apply A Log-log Transform}
\vskip0.1in\hrule\vskip0.1in
The first step in obtaining a fit is to transform the relationship using
properties of logarithms. So,
$$
  | e_{k+1} | = C | e_k |^r
$$
becomes
$$
  log(| e_{k+1} |) = log(C | e_k |^r)
      = log(C) + log(| e_k |^r)
      = log(C) + r\ log(| e_k |)
      = a + r\ log(| e_k |)
$$
The logarithmic transformation turns the equality into a linear polynomial in
the two unkbown parameters $r$ and $a=log(C)$. This brings us back to the output
from the code for the Bisection method. The last two columns are the logarithmic
values we need in the transformed error relationship. So, we can write
\vskip0.1in\hrule\vskip0.1in
\begin{center}
  \begin{tabular}{lcl}
    $log(| e_1 |) = a + r\ log(| e_0 |)$
          & $\rightarrow$
          & $(0.954242527) = a + r\ (0.653212488)$ \\
    $log(| e_2 |) = a + r log(| e_1 |)$
          & $\rightarrow$ 
          & $(0.653212488) = a + r\ (0.352182508)$ \\
    $| e_3 | \leq C\ | e_1 |^r$
          & $\rightarrow$
          & $(0.352182508) = a + r\ (5.11525720E-02)$ \\
        \ & $\vdots$ 
          & \ \\
    $| e_{25} | \leq C\ | e_{24} |^r$
          & $\rightarrow$
          & $(-6.32163000) = a + r\ (-6.62265968)$
  \end{tabular}
\end{center}
\vskip0.1in\hrule\vskip0.1in
We can write the conditions from the last equations in a matrix form as follows:
\vskip0.1in\hrule\vskip0.1in
\begin{equation}
   \left[
     \begin{array}{cc}
       1.0 & 0.653212488 \\
       1.0 & 0.352182508 \\
       1.0 & 5.11525720E-02 \\
       \vdots & \vdots \\
       1.0 & -6.62265968
     \end{array}
   \right]
   \left[
     \begin{array}{c}
       a \\ 
       r
     \end{array}
   \right]
   =
   \left[
     \begin{array}{c}
       0.954242527 \\
       0.653212488 \\
       0.352182508 \\
       \vdots \\
       -6.32163000
     \end{array}
   \right]
\end{equation}
\vskip0.1in\hrule\vskip0.1in
\noindent
This system is over-determined. We are trying to determine two parameters, $a$
and $r$ using 25 constraints. It is highly unlikely when roundoff and other
errors effect these equations to expect to obtain a system with a unique
solution. We can, however, come up with the next best thing. That is, we can
project the problem into a space where a unique solution exists. This process
goes by the official name of linear regression.

A shorthand symbolic form of our system of equations can be written as the
matrix equation
$$
  A\ x = b
$$ 
The normal equations for this system of equations is obtained by multiplying
both sides of the equation by the transpose of the matrix. That is,
$$
  A^T\ A {\bf x} = A^T\ b
$$
Note that for our problem the result of this projection into the column space
of the matrix, $A$, is a $2\times 2$ linear system of equations. Note that the
coefficient matrix is symmetric and is positive definite is the two columns of
$A$ are independent of each other.

The matrix equation can be written in the form
$$
  A^T\ A = \left[
             \begin{array}{rcl}
               a_{11} & a_{12} \\ 
               a_{21} & a_{22} 
             \end{array}
          \right]
$$
where $a_{21}=a_{12}$. A closed formula for the inverse of the matrix can be
written down as follows.
$$
  \left((A^T\ A\right) = {1\over{(a_{11} a_{22} - a_{12} a_{21})}}
                         \left[
                           \begin{array}{rcl}
                             a_{22} & -a_{21} \\ 
                             -a_{12} & a_{11} 
                           \end{array}
                        \right]
$$
The details of these calculations are embedded in the code segment given above.
The result of the computations in the code and as shown in the output from the
code are the following.
\vskip0.1in\hrule\vskip0.1in
\begin{verbatim}

      shift constant:           0.74136263
      rate of convergence:      1.00143242

\end{verbatim}
\vskip0.1in\hrule\vskip0.1in
The shift constant is the constant, $C$, in the error reduction from one step to
the next. It should be noted that the computational convergence rate for the
bisection method is about one - indicating the convergence is linear.
\vskip0.1in\hrule\vskip0.1in
\newpage
%%%%%%%%%%%%%%%%%%%%%%%%%%%%%%%%%%%%%%%%%%%%%%%%%%%%%%%%%%%%%%%%%%%%%%%%%%%%%%%%
%%%%%%%%%%%%%%%%%%%%%%%%%%%%%%%%%%%%%%%%%%%%%%%%%%%%%%%%%%%%%%%%%%%%%%%%%%%%%%%%
\vskip0.1in\hrule\vskip0.1in
\noindent
{\bf Errors and Rate of Convergence: Summary of Steps to Compute Computational
Convergence Rates}
\vskip0.1in\hrule\vskip0.1in
The steps needed to compute the computational convergence for the case when a
solution is known are the following.
\begin{enumerate}
  \item Write a code and test the code to make sure that a solution is reached.
  \item Embed a computation in the working code to determine the absolute error
        in the approximation using the and store this error in an array.
  \item Compute the log-log transform of the error at successive steps.
  \item Perform a linear regression of the error data to a linear polynomial
        based on the transformed relationship between errors at successive
        iterations of the algorithm.
\end{enumerate}
Note that these instructions can be applied to just about any numerical
algorithm. Any competent computational scienttist should have a self-contained
code for computing convergence rates. In the code given above all the steps are
embedded in the computer code. If there is a need for this type of code in any
other algorithm, the least squares approach would need to be recoded in terms
of the new application or coding environment.

As an example, consider the following method developed for the Java coding
language.
\vskip0.1in\hrule\vskip0.1in
\begin{verbatim}

      public double convergenceRate(double [] errArray)
      {
        //
        // test to see if there is anything in the error array
        // ---------------------------------------------------
        //
        if(errArray == null) return -1.0;
        //
        // initialize the symmetric 2x2 matrix
        // -----------------------------------
        //
        double a11 = ((double) n);
        double a12 = 0.0;
        double a22 = 0.0;
        double b1 = 0.0;
        double b2 = 0.0;
        //
        // get the number of array values
        // ------------------------------
        //
        int n = errArray.length;
        //
        // loop over pairs of errors to do the computation
        // -----------------------------------------------
        //
        for(int i=0; i<(n-1); i++) {
          //
          // get the two successive errors
          // -----------------------------
          //
          double ek = errArray[i];
          double ekp1 = errArray[i+1];
          //
          // update the matrix entries
          // -------------------------
          //
          a12 = a12 + alog10(ek)
          a22 = a22 + alog10(ek) * alog10(ek)
          //
          // update the right side entries
          // -----------------------------
          //
          b1 = b1 + alog10(ekp1)
          b2 = b2 + alog10(ek) * alog10(ekp1)
          //
        }
        //
        // compute the action of the inverse of the 2x2 on the right hand side
        // -------------------------------------------------------------------
        //
        double detval = a11 * a22 - a12 * a12;
        double aval = ( a22 * b1 - a12 * b2 ) / detval;
        double rval = ( a11 * b2 - a12 * b1 ) / detval;
        //
        // return the convergence rate
        // ---------------------------
        //
        return rval;
        //
      }

\end{verbatim}
\vskip0.1in\hrule\vskip0.1in
In a Java program or jar file, it might be a good idea to include this in a
package with its own Application Programming Interface (API). The code can
be translated or modified as needed. In object oriented languages (like Java)
a programming might overload the definition of the method to possibly return
the shift constant, $log(C)$. As a last note on the code above, the method in
Java was harvested and translated out of the Bisection method code. All you need
to do is have a couple of terminal windows and a cut and paste feature on your
desktop. Now on to a more realistic treatment of computational convergence rate.
\newpage
%%%%%%%%%%%%%%%%%%%%%%%%%%%%%%%%%%%%%%%%%%%%%%%%%%%%%%%%%%%%%%%%%%%%%%%%%%%%%%%%
%%%%%%%%%%%%%%%%%%%%%%%%%%%%%%%%%%%%%%%%%%%%%%%%%%%%%%%%%%%%%%%%%%%%%%%%%%%%%%%%
\vskip0.1in\hrule\vskip0.1in
\noindent
{\bf Errors and Rate of Convergence: Case 2. An Exact Solution is Not Known}
\vskip0.1in\hrule\vskip0.1in
In most problems, we will not know the exact solution and may have absolutely
no idea where the solution resides in our mathematical system. One way to get
an estimate of the convergence rate of an algorithm implemented into computer
code involves the following steps.

\begin{enumerate}
  \item Given that a code is working properly, run the code until an approximate
        solution is obtained or at least several approximations have been
        compouted.
  \item If $n$ approximations are collected, use the last approximation as the
        exact value.
  \item Then compute the rate using the first $n-1$ approximations.
\end{enumerate}

Let's consider a modification of the self-contained method written in Java in
the last section. Instead of passing in the errors at each step, the new code
will pass in the approximations obtained in an array.
\vskip0.1in\hrule\vskip0.1in
\begin{verbatim}

      public double convergenceRate(double [] approxArray, double exact)
      {
        //
        // test to see if there is anything in the error array
        // ---------------------------------------------------
        //
        if(errArray == null) return -1.0;
        //
        // initialize the symmetric 2x2 matrix
        // -----------------------------------
        //
        double a11 = ((double) n);
        double a12 = 0.0;
        double a22 = 0.0;
        double b1 = 0.0;
        double b2 = 0.0;
        //
        // get the number of array values and compute the errors between the
        // approximations and the last array element
        // -----------------------------------------
        //
        int n = approxArray.length;
        double [] errArray = new double[n-1];
        for(int i=0; i<(n-2); i++) {
           errArray[i] = Math.abs(exact - approxArray[i]);
        }
        //
        // overload the method to use the code in the previous case where the
        // exact value was known
        // ---------------------
        //
        return convergenceRate(errArray);
        //
      }

\end{verbatim}
\vskip0.1in\hrule\vskip0.1in
The second incarnation of the method actually uses the first through overloading
the definition of the method. The second version uses a value specified in the
second argument that will serve as the solution.  The code computes an error
for each of the approximations based on the inputs and then uses the previous
version to return the computational convergence rate. 

So, why does this work? The justification comes from a simple application of the
triangle inequality. The erro between an approximation, $x$, and the exact
value, $x^*$, can be written
$$
  e_k = | x_k - x^* | = | x_k - x_n + x_n - x^* |
$$
using a cleverly chosen form of the number zero. Next, we can apply the triangle
inequality to obtain
$$
  e_k \leq  | x_k - x_n | + | x_n - x^* |
$$
If the algorithm appears to be converging, then we would expect that
$$
  \lim_{n\rightarrow\infty}  | x_n - x^* | = 0
$$
regardless of how fast this is happening. So, we can use
$$
  e_k \leq  | x_k - x_n |
$$
as an approximation of the successive approximations compared to the exact
value. In matheamtical terms we say that the second term is negligible. The
second version of the convergenceRate() method can be used for either of the
cases.
\vskip0.1in\hrule\vskip0.1in
\newpage
%%%%%%%%%%%%%%%%%%%%%%%%%%%%%%%%%%%%%%%%%%%%%%%%%%%%%%%%%%%%%%%%%%%%%%%%%%%%%%%%
%%%%%%%%%%%%%%%%%%%%%%%%%%%%%%%%%%%%%%%%%%%%%%%%%%%%%%%%%%%%%%%%%%%%%%%%%%%%%%%%
\vskip0.1in\hrule\vskip0.1in
\noindent
{\bf Errors and Rate of Convergence: An Example When the Exact Value is Unknown}
\vskip0.1in\hrule\vskip0.1in
We can use the Bisection method example earlier in this section. Suppose that we
do not have an exact value for the root in the example problem. We can consider
the first 15 iterations in the example as approximations and the $16^{th}$ as a
good approximation as the exact value.  A modified version of the original
Bisection code in our example can be created that (1) computes a total of 16
approximations, (2) sets $x^*=x_{16}$, (3) computes the approximate error
between the first 15 approximations abd the very last approximation.

\begin{verbatim}

      program main
c
c variables for the test
c ----------------------
c
      real a, b, c(1000), pi, tol
      real ek(1000)
      real a11, a12, a22, b1, b2
      real aval, rval
      real detval
      integer n
c
c initialize the interval and tolerance desired
c ---------------------------------------------
c
      a = -2.2
      b = 6.8
      tol = 0.0000001
c
c compute the number of bisections needed for the given tolerance
c ---------------------------------------------------------------
c
      n = 16
c
c do the bisection iterations needed
c ----------------------------------
c
      do 1 i=1, n
         c(i) = 0.5 * ( a + b )
         if(f(a)*f(c(i)) .lt. 0.0) then
            b = c(i)
         else
            a = c(i)
         endif
    1 continue
c
c print out a header for the output
c ---------------------------------
c
      print *," Iter:     ek    ekp1    log(ek)     log(ekp1)"
      print *," ---------------------------------------------"
c
c initialize storage for the linear regression to determine the computational
c rate of convergence - note that the normal equations matrix is symmetric.
c -------------------------------------------------------------------------
c
      a11 = dble(n-1)
      a12 = 0.0
      a22 = 0.0
      b1 = 0.0
      b2 = 0.0
c
c loop over the approximations computed computing the approximate errors
c ----------------------------------------------------------------------
c
      do 2 i=1,n-1
         ek(i) = abs( c(16) - c(i) )
    2 continue
      do 3 i=1,n-2
c
c print out the next line in the table of errors
c ----------------------------------------------
c
         print *, i, ek(i), ek(i+1), alog10(ek(i)), alog10(ek(i+1))
c
c update the error variables in the normal equations
c --------------------------------------------------
c
         a12 = a12 + alog10(ek(i))
         a22 = a22 + alog10(ek(i)) * alog10(ek(i))
         b1 = b1 + alog10(ek(i+1))
         b2 = b2 + alog10(ek(i)) * alog10(ek(i+1))
c
    3 continue
c
c compute the solution using the inverse of the 2x2 matrix for the linear
c reduction of the matrix
c -----------------------
c
      detval = a11 * a22 - a12 * a12
      aval = ( a22 * b1 - a12 * b2 ) / detval
      rval = ( a11 * b2 - a12 * b1 ) / detval
c
c output the results - note that the constant needs to be exponentiated
c ---------------------------------------------------------------------
c
      print *, "shift constant:   ", exp(aval)
      print *, "rate of convergence:   ", rval
c
c this code will be used to test convergence rates for the bisection method
c -------------------------------------------------------------------------
c
      stop
      end
c
c a simple function used to test the bisection method
c ---------------------------------------------------
c
      real function f(x)
      real pi
      pi = 3.141592653589793
      f = exp(x) - pi
      return
      end

\end{verbatim}
\vskip0.1in\hrule\vskip0.1in

\begin{verbatim}

         Iter:     ek              ekp1             log(ek)       log(ekp1)
         ------------------------------------------------------------------
           1   1.15534973       1.09465039       6.27134740E-02   3.92754339E-02
           2   1.09465039       3.03497314E-02   3.92754339E-02  -1.51784515    
           3   3.03497314E-02  0.532150328      -1.51784515     -0.273965687    
           4  0.532150328      0.250900269     -0.273965687     -0.600498915    
           5  0.250900269      0.110275269     -0.600498915     -0.957521856    
           6  0.110275269       3.99627686E-02 -0.957521856      -1.39834440    
           7   3.99627686E-02   4.80651855E-03  -1.39834440      -2.31816936    
           8   4.80651855E-03   1.27716064E-02  -2.31816936      -1.89375448    
           9   1.27716064E-02   3.98254395E-03  -1.89375448      -2.39983940    
          10   3.98254395E-03   4.11987305E-04  -2.39983940      -3.38511610    
          11   4.11987305E-04   1.78527832E-03  -3.38511610      -2.74829412    
          12   1.78527832E-03   6.86645508E-04  -2.74829412      -3.16326737    
          13   6.86645508E-04   1.37329102E-04  -3.16326737      -3.86223745    
          14   1.37329102E-04   1.37329102E-04  -3.86223745      -3.86223745    

 shift constant:     0.639538586    
 rate of convergence:     0.886137068    

\end{verbatim}
\vskip0.1in\hrule\vskip0.1in
There are a few things to notice in the output. First, the computational
convergence rate given in the output is less than $1$. This is due to the fact
that the approximations have not converged and we will always end up with an
approximation. It would be a good idea to take more iterations before computing
the convergence rate.

The modified code in this example uses the midpoint of the interval containing
a root to define the error. The original code used the interval length as an
upper bound on the error in the approximation. This will give a different
approximation of the errors in the approximations.
\vskip0.1in\hrule\vskip0.1in
\newpage
%%%%%%%%%%%%%%%%%%%%%%%%%%%%%%%%%%%%%%%%%%%%%%%%%%%%%%%%%%%%%%%%%%%%%%%%%%%%%%%%
%%%%%%%%%%%%%%%%%%%%%%%%%%%%%%%%%%%%%%%%%%%%%%%%%%%%%%%%%%%%%%%%%%%%%%%%%%%%%%%%
\vskip0.1in\hrule\vskip0.1in
\noindent
{\bf Errors and Rate of Convergence: Error Reduction per Iteration
Perturbation }
\vskip0.1in\hrule\vskip0.1in
There are times when it is important to know about how much the error has been
reduced from one iterate to the next. A typical example in root finding problems
involves using two different methods, say the Bisection method to start the root
finding and then switch to Newton's method once the interval is small enough.
These are called hybrid methods and are handled in another part of these notes.
If we can estimate the reduction in error at each step, we can estimate the
number of steps needed to meet a given tolerance in the error. The Bisection
method is very easily analyzed in this setting. 

Suppose we want to reduce the error by one order of magnitude using Bisection.
This means that we would need to take some number of steps using the Bisection
method to achieve the reduction. We can assume that assumptions for successful
application of the Bisection method are satisfied. The inequality we need is
$$
  | e_{k+p} | = \left( {1\over 2} \right)^p | e_k | \leq 10^{-1}
$$
The last expression states that we need to find $p$ that makes this true - that
is, how many iterations are required to make sure the error is reduced by one
decimal digit. Recall that computers work in binary arithmetic. Rewriting and
using logarithms we find
$$
  log_{10}\left( {1\over 2} \right)^p ) \leq log_{10}(10^{-1}) = -1
$$
Solving for $p$ gives
$$
    p = {{-1}\over{log_{10}({1\over 2})}} = {1\over{log_{10}(2)}} \approx 4
$$
This states that in order to reduce the error in the sequence of approximations
provided by the Bisection method, we need to take just under 4 iterations. Since
the iteration counter is an integer, we will need to round up to 4 iterations.

Consider again the data generated by the Bisection method in the first example
and look at the iterations from $k=8$ through $k=12$ in the second column of
output.
\vskip0.1in\hrule\vskip0.1in
\begin{verbatim}

           8   7.03125000E-02   3.51562500E-02  -1.15296745      -1.45399749
           9   3.51562500E-02   1.75781250E-02  -1.45399749      -1.75502753
          10   1.75781250E-02   8.78906250E-03  -1.75502753      -2.05605745
          11   8.78906250E-03   4.39453125E-03  -2.05605745      -2.35708737
          12   4.39453125E-03   2.19726562E-03  -2.35708737      -2.65811753

\end{verbatim}
\vskip0.1in\hrule\vskip0.1in
\noindent
Notice that for $k=11$ the error is not quite one order of magnitude less than
the error for $k=8$. However, the difference in errors between $k=8$ and $k=12$
is more than one order of magnitude in reduction of the error.

The methods used in the reduction of error can typically be applied aposteriori
meaning after the fact. For example, for Newton's method, there are ways to
analytically compute the convergence rate and thus the error reduction in each
step as long as the iterates are sufficiently close to the root in question.
However, until the iterates are close enough there is no guarantee as to how
much the error has been reduced in one step. Recall that for Newton's method
$$
  | e_{k+1} | \leq C | e_k |^2
$$
where $C$ depends on the second derivative of the function at a root that has
been located. The dependence can be determined using a Taylor series expansion.
If the second derivative is large, the reduction per iteration may be affected
by the constant, $C$. As the iterates get closer to a root, Newton's method will
become more predicatble in terms of the error reduction. Note that in most
cases, Newton's method is very fast.
\vskip0.1in\hrule\vskip0.1in
\newpage
%%%%%%%%%%%%%%%%%%%%%%%%%%%%%%%%%%%%%%%%%%%%%%%%%%%%%%%%%%%%%%%%%%%%%%%%%%%%%%%%
%%%%%%%%%%%%%%%%%%%%%%%%%%%%%%%%%%%%%%%%%%%%%%%%%%%%%%%%%%%%%%%%%%%%%%%%%%%%%%%%
\end{document}
%%%%%%%%%%%%%%%%%%%%%%%%%%%%%%%%%%%%%%%%%%%%%%%%%%%%%%%%%%%%%%%%%%%%%%%%%%%%%%%%
%%%%%%%%%%%%%%%%%%%%%%%%%%%%%%%%%%%%%%%%%%%%%%%%%%%%%%%%%%%%%%%%%%%%%%%%%%%%%%%%
\vskip0.1in\hrule\vskip0.1in
\noindent
{\bf Errors and Rate of Convergence: Stability of an Algorithm under Small
Perturbation }
\vskip0.1in\hrule\vskip0.1in

\vskip0.1in\hrule\vskip0.1in
\newpage
