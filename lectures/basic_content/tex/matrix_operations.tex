\documentclass[10pt,fleqn]{article}
%\usepackage{graphicx}


\setlength{\topmargin}{-.75in}
\addtolength{\textheight}{2.00in}
\setlength{\oddsidemargin}{.00in}
\addtolength{\textwidth}{.75in}

\title{Math 4610 Lecture Notes \\
            \ \\
        Vector and Matrix Operations in Computational Mathematics
  \footnote{These notes are part of an Open Resource Educational project
            sponsored by Utah State University}}

\author{Joe Koebbe}

\nofiles

\pagestyle{empty}

\setlength{\parindent}{0in}

% new math commands


\setlength{\oddsidemargin}{-0.25in}
\setlength{\evensidemargin}{-0.25in}
\setlength{\textwidth}{6.75in}
\setlength{\headheight}{0.0in}
\setlength{\topmargin}{-0.25in}
\setlength{\textheight}{9.00in}

\makeindex

\usepackage{mathrsfs}

%\usepackage[pdftex]{graphicx}
\usepackage{epstopdf}

\newcounter{beans}

\newcommand{\ds}{\displaystyle}
\newcommand{\limit}[2]{\displaystyle\lim_{#1\to#2}}

\newcommand{\binomial}[2]{\ \left( \begin{array}{c}
                                  #1 \\
                                  #2
                                 \end{array}
                            \right) \
                         }
\newcommand{\ExampleRule}[2]
  {
  \noindent
  \rule{\linewidth}{1pt}
  \begin{example}
    #1
    \label{#2}
  \end{example}
  \rule{\linewidth}{1pt}
  \vskip0.125in
  }

\newcommand{\defbox}[1]
  {
   \ \\
   \noindent
   \setlength\fboxrule{1pt}
   \fbox{
        \begin{minipage}{6.5in}
          #1
        \end{minipage}
        }
   \ \\
  }
\newcommand{\verysmallworkbox}[1]
  {
   \ \\
   \noindent
   \setlength\fboxrule{1pt}
   \fbox{
        \begin{minipage}{6.5in}
           #1
           \ \\
           \vskip0.5in \ \\
           \ \\
        \end{minipage}
        }
   \ \\
  }
\newcommand{\smallworkbox}[1]
  {
   \ \\
   \noindent
   \setlength\fboxrule{1pt}
   \fbox{
        \begin{minipage}{6.5in}
           #1
           \ \\
           \vskip2.5in \ \\
           \ \\
        \end{minipage}
        }
   \ \\
  }
\newcommand{\halfworkbox}[1]
  {
   \ \\
   \noindent
   \setlength\fboxrule{1pt}
   \fbox{
        \begin{minipage}{6.5in}
           #1 \hfill
           \ \\
           \vskip3.25in \ \\
           \ \\
        \end{minipage}
        }
   \ \\
  }
\newcommand{\largeworkbox}[1]
  {
   \ \\
   \noindent
   \setlength\fboxrule{1pt}
   \fbox{
        \begin{minipage}{6.5in}
           #1
           \ \\
           \vskip7.5in \ \\
           \ \\
        \end{minipage}
        }
   \ \\
  }
\newcommand{\flexworkbox}[2]
  {
   \ \\
   \noindent
   \setlength\fboxrule{1pt}
   \fbox{
        \begin{minipage}{6.5in}
           #1
           \ \\

           \vskip#2 \ \\
           \ \\
        \end{minipage}
        }
   \ \\
  }


% symbols for sets of numbers

\newcommand{\natnumb}{$\cal N$}
\newcommand{\whonumb}{$\cal W$}
\newcommand{\intnumb}{$\cal Z$}
\newcommand{\ratnumb}{$\cal Q$}
\newcommand{\irrnumb}{$\cal I$}
\newcommand{\realnumb}{$\cal R$}
\newcommand{\cmplxnumb}{$\cal C$}

% misc. commands

\newcommand{\mma}{{\it Mathematica}}
\newcommand{\sech}{\mbox{ sech}}
 
\newtheorem{theorem}{Theorem}
\newtheorem{example}{Example}
\newtheorem{definition}{Definition}
\newtheorem{problem}{Problem}

\setcounter{secnumdepth}{2}
\setcounter{tocdepth}{4}


\begin{document}
\maketitle
\newpage

%%%%%%%%%%%%%%%%%%%%%%%%%%%%%%%%%%%%%%%%%%%%%%%%%%%%%%%%%%%%%%%%%%%%%%%%%%%%%%%%
%%%%%%%%%%%%%%%%%%%%%%%%%%%%%%%%%%%%%%%%%%%%%%%%%%%%%%%%%%%%%%%%%%%%%%%%%%%%%%%%
\vskip0.1in\hrule\vskip0.1in
\noindent
{\bf Vector/Matrix Operations: Basic Operations on Vectors}
\vskip0.1in\hrule\vskip0.1in
\noindent
The central algorithms used in most computer simulations involve computer codes
that perform basic vector and matrix operations. Operations like matrix-vector
multiplications are central to methods for solving linear systems of equations
and determination of eigenvalues and eigenvectors. In this section of the class
notes, we will cover a number of basic vector and matrix operations.

Suppose that we want to compute the solution of a linear system of equations in
matrix form
$$
  A\ {\bf x} = {\bf b}
$$
where $A$ is the coefficient matrix, ${\bf b}$ is an input or right hand side
vector, and we want to determine ${\bf x}$, a vector of unknowns. Due to issues
related to roundoff error and machine precision representation of numbers, any
vector we obtain will most likely be at best an approximation of the solution.

Since we know there will be errors, we will need some way to measure or compute
the error associated with any approximation obtained. As we will see, there are
infinitely many ways to measure errors. However, we will settle for three
standard meausres of error introduced in the next section of the notes. To
implement the error definitions, we will need to write codes that perform basic
vector operations one would learn in a standard linear algebra course in a
college curriculum. We will begin with basic algebraic operations involvinv
vectors such as addition and scalar multiplication and then will proceed with
the development codes for computing the magnitude/length of a vector and will
end up with matrix algebra and operations.
\vskip0.1in\hrule\vskip0.1in
\newpage
%%%%%%%%%%%%%%%%%%%%%%%%%%%%%%%%%%%%%%%%%%%%%%%%%%%%%%%%%%%%%%%%%%%%%%%%%%%%%%%%
%%%%%%%%%%%%%%%%%%%%%%%%%%%%%%%%%%%%%%%%%%%%%%%%%%%%%%%%%%%%%%%%%%%%%%%%%%%%%%%%
\vskip0.1in\hrule\vskip0.1in
\noindent
{\bf Vector/Matrix Operations: Basic Operations on Vectors}
\vskip0.1in\hrule\vskip0.1in
\noindent
For a set of objects to form a vector space a long list of properties must be
satisfied. In these note, we will be working primarily with vectors in $\Re^n$
or possibly the space of vectors with complex entries. As such, we will assume
that we are working with standard vector spaces. Any vector space or subspace
must satisfy closure of addition and closure of scalar multiplication. This
means that if $V$ is a vector space and ${\bf u}$ and ${\bf v}$ are in the
space, then the sum of the vectors is in $V$. That is, ${\bf w}$, defined by
$$
  {\bf w} = {\bf u} + {\bf v}
$$
is also in the vector space. Also, if ${\bf v}\in V$ and $a$ is any 
number/scalar then ${\bf u}$ defined by
$$
  {\bf v} = a\ {\bf u}
$$
is also in $V$. Based on this idea, we should at least implement a pair of
codes. One that computes the sum of two vectors and one that computes a scalar
multiple of a vector. 

Almost all of the work we will do will involve real vectors of length $n$. A
vector will be of the form $(v_1, v_2, \ldots, v_n)^T$ where each component,
$v_i, i=1,\ldots, n$ is a real number. The superscript, $T$, denotes the
transpose. It should be noted that there is a differnce between row and column
vectors in linear algebra. In our vector notation, we will by default assume
that a given vector is a column vector. Transposing a column vector results in
a row vector and vice versa. This convention lends itself to operations like
matrix-vector multiplication. 

So, let's start with the sum of two vectors. The conventional definition of
addition of vectors is a componentwise definition. If ${\bf u}$ and ${\bf v}$
are two vectors of length $n$. Then
$$
  {\bf u} + {\bf v} = 
       \left[
         \begin{array}{c}
           u_1 \\
           u_2 \\
           \vdots \\
           u_n
         \end{array}
       \right]
     + \left[
         \begin{array}{c}
           v_1 \\
           v_2 \\
           \vdots \\
           v_n
         \end{array}
       \right]
     = \left[
         \begin{array}{c}
           u_1 + v_1 \\
           u_2 + v_2 \\
           \vdots \\
           u_n + v_n
         \end{array}
       \right]
$$
For completeness, we can also write down a relationship between row and column
vectors. That is,
$$
  {\bf v}^T = 
       \left[
         \begin{array}{c}
           v_1 \\
           v_2 \\
           \vdots \\
           v_n
         \end{array}
       \right]^T
     = \left[
         v_1, v_2, \cdots v_n
       \right]
$$
and
\noindent
$$
  \left[ v_1, v_2, \cdots v_n \right]^T
     = \left[
         \begin{array}{c}
           v_1 \\
           v_2 \\
           \vdots \\
           v_n
         \end{array}
       \right] = {\bf v}
$$
This is one of the easiest operations that we can code. There are a few things
that might need checking. For example, we might need to check the dimension of
the two vectors. Without any checks on compabilitiy of dimensions, the code,
written in Java might look like the following.
\vskip0.1in\hrule\vskip0.1in
\begin{verbatim}

     public double[] vecadd(double[] u, double[] v) {
       //
       // get the dimension of the first vector and instantiate some storage for
       // the output vector
       // -----------------
       //
       int n = u.length;
       double [] output = new double[n];
       //
       // compute the componentwise sum for the vector
       // --------------------------------------------
       //
       for(int i=0; i<n; i++) output[i] = u[i] + v[i];
       //
       // return the output
       // -----------------
       //
       return output;
       //
     }

\end{verbatim}
\vskip0.1in\hrule\vskip0.1in
Before coding a scalar multiple method for our library, it would be best to
write a code for computing the difference in a pair of vectors. Given two
vectors ${\bf u}$ and ${\bf v}$, the formula for the difference is the
following.
$$
  {\bf u} - {\bf v} = 
       \left[
         \begin{array}{c}
           u_1 \\
           u_2 \\
           \vdots \\
           u_n
         \end{array}
       \right]
     - \left[
         \begin{array}{c}
           v_1 \\
           v_2 \\
           \vdots \\
           v_n
         \end{array}
       \right]
     = \left[
         \begin{array}{c}
           u_1 - v_1 \\
           u_2 - v_2 \\
           \vdots \\
           u_n - v_n
         \end{array}
       \right]
$$
It is an easy exercise to write a code that will compute the difference of two
vectors. All that needs to happen is to give the routine a unique name and
change the addition to a subtraction. For exaxmple,
\vskip0.1in\hrule\vskip0.1in
\begin{verbatim}

     public double[] vecsub(double[] u, double[] v) {
       //
       // get the dimension of the first vector and instantiate some storage for
       // the output vector
       // -----------------
       //
       int n = u.length;
       double [] output = new double[n];
       //
       // compute the componentwise sum for the vector
       // --------------------------------------------
       //
       for(int i=0; i<n; i++) output[i] = u[i] - v[i];
       //
       // return the output
       // -----------------
       //
       return output;
       //
     }

\end{verbatim}
\vskip0.1in\hrule\vskip0.1in
This particular routine can be used as a means to compute the difference between
two vectors as a part of measuring errors in vector approximations. Note that
the dimensions of the two vectors in the subtraction of vectors has not been
checked. This means it is up to the user to ensuere that the two vectors have
the same length.

The last code to write in this section is the scalar multiplication code. The
vector operation is defined as follows. For any vector, ${\bf v}$, and number,
$a$, scalar multiplication is defined by
$$
  a\ {\bf v} = 
       a\ \left[
            \begin{array}{c}
              v_1 \\
              v_2 \\
              \vdots \\
              v_n
            \end{array}
          \right]
     = \left[
         \begin{array}{c}
           a\ v_1 \\
           a\ v_2 \\
           \vdots \\
           a\ v_n
         \end{array}
       \right]
$$
and the code to implement this operation is given by
\vskip0.1in\hrule\vskip0.1in
\begin{verbatim}

     public double[] scalvec(double a, double[] v) {
       //
       // get the dimension of the first vector and instantiate some storage for
       // the output vector
       // -----------------
       //
       int n = u.length;
       double [] output = new double[n];
       //
       // compute the componentwise sum for the vector
       // --------------------------------------------
       //
       for(int i=0; i<n; i++) output[i] = a * v[i];
       //
       // return the output
       // -----------------
       //
       return output;
       //
     }

\end{verbatim}
\vskip0.1in\hrule\vskip0.1in
In the scalar multiplication, the length of the vector can be any integer and as
long as the vector is not null, the method will work. However, it might be
necessary to test to see if the vector actually exists. If a null vector is
passed to any method, there is likely to be a problem. So, in a software manual
you would need to warn the user of this and any other assumptions that might
cause problems.

The methods developed in this section of the notes provides a starting point for
a number of other methods. It will be easy to cut and paste and rename methods
to do most of the basic matrix operations outlined in these notes.
\vskip0.1in\hrule\vskip0.1in
\newpage
%%%%%%%%%%%%%%%%%%%%%%%%%%%%%%%%%%%%%%%%%%%%%%%%%%%%%%%%%%%%%%%%%%%%%%%%%%%%%%%%
%%%%%%%%%%%%%%%%%%%%%%%%%%%%%%%%%%%%%%%%%%%%%%%%%%%%%%%%%%%%%%%%%%%%%%%%%%%%%%%%
\vskip0.1in\hrule\vskip0.1in
\noindent
{\bf Vector/Matrix Operations: The Euclidean Length of a Vector}
\vskip0.1in\hrule\vskip0.1in
\noindent
In many problems it will be necessary to measure the length or magnitude of a
vector (at least so say the villian in Despicable Me). So, we need some way to
determine the length. Actually, there are infinitely many ways to do this. The
approach most students see first involves using the Euclidean distance between
two points used to define a vector. From any standard linear algebra course, the
Euclidean length can be defined as follows.
$$
  || v || = ( v_1^2 + v_2^2 + \cdots + v_n^2 )^{1/2}
$$
where the notation, $||*||$, provides a mathematical notation for the magnitude
of the vector. The notation emphasizes the difference between the absolute 
value of the difference of two real numbers and the length of a vector. This can
be implemented easily into a method or routine that will return this length of a
vector.
\vskip0.1in\hrule\vskip0.1in
\begin{verbatim}

     public double l2norm(double[] v) {
       double sum = 0.0;
       //
       // extract the length of the vector
       // --------------------------------
       //
       int n = v.length;
       //
       // compute the sum of squares
       // --------------------------
       //
       for(int i=0; i<n; i++) sum = sum + v[i] * v[i];
       //
       // return the value
       // ----------------
       //
       return Math.sqrt(sum);
       //
     }

\end{verbatim}
\vskip0.1in\hrule\vskip0.1in
Note that the Java package including the square root function defines the last
step in the code. It is best to use an intrinsic for the square root since
implementing our own version of a square root method would take some time.
\vskip0.1in\hrule\vskip0.1in
\newpage
%%%%%%%%%%%%%%%%%%%%%%%%%%%%%%%%%%%%%%%%%%%%%%%%%%%%%%%%%%%%%%%%%%%%%%%%%%%%%%%%
%%%%%%%%%%%%%%%%%%%%%%%%%%%%%%%%%%%%%%%%%%%%%%%%%%%%%%%%%%%%%%%%%%%%%%%%%%%%%%%%
\vskip0.1in\hrule\vskip0.1in
\noindent
{\bf Vector/Matrix Operations: Definition of the Norm of a Vector}
\vskip0.1in\hrule\vskip0.1in
\noindent
There are many ways to compute the length of a vector. In this section we will
consider several ways to do this. However, we need to have a general definition
of what is meant by the length of a vector. To start, the definition of a norm
is given.
\vskip0.1in\hrule\vskip0.1in
\begin{definition}
   Suppose that $V$ is a vector space and ${\bf u}$ and ${\bf v}$ are any two
   vectors in $V$. Also, assume $a$ is an arbitrary number/scalar. The norm of a
   vector is a function
   $$
     || * || : V \rightarrow \Re
   $$
   such that
   \begin{enumerate}
     \item $||v||=0$ if and only if ${\bf v} = {\bf 0}$,
     \item $||a{\bf v}|| = |a|\ ||{\bf v}||$, and
     \item $||{\bf u} + {\bf v} || \leq ||{\bf u}|| + ||{\bf v}||$.
   \end{enumerate}
\end{definition}
\vskip0.1in\hrule\vskip0.1in
The Euclidean magnitude/length of a vector can be shown to satisfy this
definition of a norm. This means we can use the terms length, magnitude, and
norm interchangeably. The term norm is used in the name, {\bf l2norm}, chosen
for the method in the code above.

There are an infinite number of ways to compute norms on vector spaces. A
general definition for a norm on n-dimensional real space is the following.
$$
  ||v||_p = \left( v_1^p + v_2^p + \cdots + v_n^p \right)^{1/p}
              = \left( \sum_{i=1}^n\ |v_i|^p \right)^{1/p}
$$
for any positive integer $p$. The Euclidean norm is obtained when $p=2$. The
reality in computational mathematics is that there are three norms of interest
including the 2-norm. The other two norms are the 1-norm and the other is the
infinity-norm.  The definitions for these two norms are given below.
$$
  ||v||_1 = | v_1 | + | v_2 | + \cdots + | v_n | = \sum_{i=1}^n\ |v_i|
$$
and
$$
  ||v||_\infty = \max_{1\leq i\leq n} | v_i | 
$$
are the definitions we will use in our work. The gold standard for measuring
length is typically the 2-norm. However, in some problems the 1-norm and
infinity-norm are easier to compute in many problems.

The 1-norm and infinity-norm can easily be coded into methods as was done above
to the 2-norm. Using the 2-norm code, the other two methods just need a couple
of minor changes to implement the other two norms into their own methods.
\vskip0.1in\hrule\vskip0.1in
\newpage
%%%%%%%%%%%%%%%%%%%%%%%%%%%%%%%%%%%%%%%%%%%%%%%%%%%%%%%%%%%%%%%%%%%%%%%%%%%%%%%%
%%%%%%%%%%%%%%%%%%%%%%%%%%%%%%%%%%%%%%%%%%%%%%%%%%%%%%%%%%%%%%%%%%%%%%%%%%%%%%%%
\vskip0.1in\hrule\vskip0.1in
\noindent
{\bf Vector/Matrix Operations: Errors in Vector Approximations}
\vskip0.1in\hrule\vskip0.1in
\noindent
Just as the errors in approximating roots of functions of a single variable,
having a way to compute the magnitude of the error in approximating one vector
${\bf v}$ by another vector ${\bf u}$. We can use the definitions of norms
earlier in this section to define a consistent formula for the vector
approximation error. If we use a generic definition of the norm of a vector, we
can define
$$
  \makebox{absolute error} = || {\bf v} - {\bf u} ||
$$
and
$$
  \makebox{relative error} = {{|| {\bf v} - {\bf u} || }\over{ || {\bf u} || }}
$$
These formulas should look familiar when compared to error measurement in root
finding problems.

A code that will implement the absolute error with the 2-norm might look like
\vskip0.1in\hrule\vskip0.1in
\begin{verbatim}

     public double absl2err(double[] u, double[] v) {
       double sum = 0.0;
       //
       // extract the length of the vector
       // --------------------------------
       //
       int n = v.length;
       double diff = 0.0;
       for(int i=0; i<n; i++) {
         diff = u[i] - v[i];
         sum = sum + diff * diff;
       }
       //
       // return the norm of the difference
       // ---------------------------------
       //
       return Math.sqrt(sum);
       //
     }

\end{verbatim}
\vskip0.1in\hrule\vskip0.1in
Of course, we could reuse code that we have already written as follows. Provided
that the methods have been created and test and inserted in some sort of archive
that is available for our use. A first simpler version is the following.
\vskip0.1in\hrule\vskip0.1in
\begin{verbatim}

     public double absl2err(double[] u, double[] v) {
       double [] diff = null;
       int n = v.length;
       diff = vecsub(u, v);
       return l2norm(diff);
     }

\end{verbatim}
\vskip0.1in\hrule\vskip0.1in
An even more concise version of the method might like the following.
\vskip0.1in\hrule\vskip0.1in
\begin{verbatim}

     public double absl2err(double[] u, double[] v) {
       return l2norm(vecsub(u, v));
     }

\end{verbatim}
\vskip0.1in\hrule\vskip0.1in
Note that in most cases, calling another method or routine requires a bit of
overhead. The first version of the code will be more efficient than the last
version. However, it is a lot easier to maintain one version of a code than to
modify multiple codes all of which have in-lined versions of any reusable code.
In native languages like C or Fortran, you can always compile the code into a
static executable that in-lines all of these types of codes just as if the code
was all there.
\vskip0.1in\hrule\vskip0.1in
\newpage
%%%%%%%%%%%%%%%%%%%%%%%%%%%%%%%%%%%%%%%%%%%%%%%%%%%%%%%%%%%%%%%%%%%%%%%%%%%%%%%%
%%%%%%%%%%%%%%%%%%%%%%%%%%%%%%%%%%%%%%%%%%%%%%%%%%%%%%%%%%%%%%%%%%%%%%%%%%%%%%%%
\vskip0.1in\hrule\vskip0.1in
\noindent
{\bf Vector/Matrix Operations: Dot Products}
\vskip0.1in\hrule\vskip0.1in
\noindent
There are several vector products that should also be included in any discussion
of computational linear algebra. The first of these is the usual dot product or
inner product on $\Re^n$. To start, suppose we have two vectors out of the same
vector space. Then the dot product of the two vectors is given by
$$
  {\bf u} \cdot {\bf v} = u_1\ v_1 + u_2\ v_2 + u_3\ v_3 + \cdots + u_1\ v_1
$$
This product is about as easy to code as anything. In C, the code would look
like
\vskip0.1in\hrule\vskip0.1in
\begin{verbatim}

     //
     // dot product routine written in C
     // --------------------------------
     //
     double dotprd(double u[], double v[], int n)
     {
       //
       // some variable declarations
       // --------------------------
       //
       double sum = 0.0;
       //
       // compute the dot product
       // -----------------------
       //
       for(int i=0; i<n; i++) sum  = sum + u[i] * v[i];
       //
       // return the value
       // ----------------
       //
       return sum;
       //
     }

\end{verbatim}
\vskip0.1in\hrule\vskip0.1in
The code given above passes in the length of the arrays that represent the two
vectors. Note that you can write code that will determine the size of the arrays
internally using the {\it sizeof} operator. In other languages like Java there
are operators for getting the size of the array. In the case above, the user
would need to make sure things match up.

There are weighted inner products that can be defined as modifications to the
standard dot product. For example, the energy norm with respect to a given
symmetric positive definite matrix, $A$. This type of inner product starts with
the definition of an energy norem. So, suppose that the matrix, $A$, is 
symmetric and positive definite. Then the energy norm is defined by
$$
  || {\bf x} ||_A = \sqrt{ {\bf x}^T\ A\ {\bf x} }
$$
The energy norm is used in some applications in the approximation of solutions
of differential equations and in the development of the conjugate gradient
method for iteratively solving linear systems of equations. These topics will be
addressed later in these lectures.
\vskip0.1in\hrule\vskip0.1in
\newpage
%%%%%%%%%%%%%%%%%%%%%%%%%%%%%%%%%%%%%%%%%%%%%%%%%%%%%%%%%%%%%%%%%%%%%%%%%%%%%%%%
%%%%%%%%%%%%%%%%%%%%%%%%%%%%%%%%%%%%%%%%%%%%%%%%%%%%%%%%%%%%%%%%%%%%%%%%%%%%%%%%
\vskip0.1in\hrule\vskip0.1in
\noindent
{\bf Vector/Matrix Operations: The Cross Product}
\vskip0.1in\hrule\vskip0.1in
\noindent
The second product that is used in some computational problems is the cross
product of two vectors. If the cross product of two three dimensional vectors
is computed, the result is another three dimensional vector. The formulas for
the components of the output vector are the following. If ${\bf u}$ and
${\bf v}$ are two three dimensional vectors, then
$$
  {\bf u} \times {\bf v} = 
       \left[
         \begin{array}{c}
           u_1 \\
           u_2 \\
           u_3
         \end{array}
       \right]
  \times \left[
         \begin{array}{c}
           v_1 \\
           v_2 \\
           v_3
         \end{array}
       \right]
     = \left[
         \begin{array}{c}
           ( u_2\ v_3  - u_3\ v_2 ) \\
         - ( u_1\ v_3  - u_3\ v_1 ) \\
           ( u_1\ v_2  - u_2\ v_1 ) 
         \end{array}
       \right]
$$
Again, this is a relatively easy to program, but it should be noted that this
product is valid only in three dimensions. One version of a code might look
like the following. 
\vskip0.1in\hrule\vskip0.1in
\begin{verbatim}

     //
     // the cross product of two vectors
     // --------------------------------
     //
     double [] crsprd(double u[], double v[])
     {
       //
       // some variable declarations
       // --------------------------
       //
       double cross[3];
       cross[0] = u[1] * v[2] - u[2] * v[1];
       cross[1] = u[2] * v[0] - u[0] * v[2];
       cross[2] = u[0] * v[1] - u[1] * v[0];
       //
       // return the value
       // ----------------
       //
       return cross;
       //
     }

\end{verbatim}
\vskip0.1in\hrule\vskip0.1in
Note that the second component is written to absorb the negative in front of
the second component. This will be slightly more efficient.
\vskip0.1in\hrule\vskip0.1in
\newpage
%%%%%%%%%%%%%%%%%%%%%%%%%%%%%%%%%%%%%%%%%%%%%%%%%%%%%%%%%%%%%%%%%%%%%%%%%%%%%%%%
%%%%%%%%%%%%%%%%%%%%%%%%%%%%%%%%%%%%%%%%%%%%%%%%%%%%%%%%%%%%%%%%%%%%%%%%%%%%%%%%
\vskip0.1in\hrule\vskip0.1in
\noindent
{\bf Vector/Matrix Operations: The Vector Outer Product}
\vskip0.1in\hrule\vskip0.1in
\noindent
As we start into this type of vector product, the standard inner product can be
easily written as
$$
   {\bf u}\ \cdot\ {\bf v} = {\bf u}^T\ {\bf v} = {\bf v}^T\ {\bf u} 
$$
That is, the dot product can be couched in terms of matrix/vector
multiplication. This is just an interpretation of the dot product in terms of
matrix multiplication - a $1\times n$ times a $n\times 1$ giving a one by one
result or number.

So, what happens if we reverse the product. That is,
$$
   A = {\bf u}\ {\bf v}^T 
$$
We use a matrix, $A$, as the output from the product. Note that this product is
well defined. That is, $n\times 1$ times a $1\times n$ giving a matrix that is
$n\times n$. This type of product is called an outer product and is a useful
addition to a software library. Some forms of matrix multiplication can be
written in terms of an outer product. In addition, whereas the dot product
requires two vectors of the same length, the outer product is well defined for
any pair of vectors.
\vskip0.1in\hrule\vskip0.1in
\begin{verbatim}

     //
     // routine to compute the outer product of two vectors
     // ---------------------------------------------------
     //
     double [] otrprd(double u[], double v[], int m, int n)
     {
       double A[m][n];
       //
       // a double loop is needed to compute the components
       // -------------------------------------------------
       //
       for(int i=0; i<m; i++) {
         for(int j=0; j<n; j++) {
           a[i][j] = u[i] * v[j];
         }
       }
       //
       // return the value
       // ----------------
       //
       return A;
       //
     }

\end{verbatim}
\vskip0.1in\hrule\vskip0.1in
The result of this product is a matrix. It should be noted that the resulting
matrix is a rank-1 matrix. This means that the output is a matrix with only
one independdent vector. The rest of the vectors are a constant multiple of
the first nonzero vector. If any of the components of either of the vectors is
zero, then at least one row or column will be the zero vector.
\vskip0.1in\hrule\vskip0.1in
\newpage
%%%%%%%%%%%%%%%%%%%%%%%%%%%%%%%%%%%%%%%%%%%%%%%%%%%%%%%%%%%%%%%%%%%%%%%%%%%%%%%%
%%%%%%%%%%%%%%%%%%%%%%%%%%%%%%%%%%%%%%%%%%%%%%%%%%%%%%%%%%%%%%%%%%%%%%%%%%%%%%%%
\vskip0.1in\hrule\vskip0.1in
\noindent
{\bf Vector/Matrix Operations: Matrix Addition}
\vskip0.1in\hrule\vskip0.1in
\noindent
The next step is to work through matrix algebra that help define sets of
matrices that form linear vector spaces. The first operation is matrix
addition. So, given two matrices, $A$ and $B$, that are the same size - same
number of rows and columns in both matrices, matrix multiplication is defined
using the following. 
$$
  A + B =
   \left[
     \begin{array}{cccc}
       a_{1,1} & a_{1,2} & \cdots & a_{1,n} \\
       a_{2,1} & a_{2,2} & \cdots & a_{2,n} \\
       \vdots  & \vdots  &  \    & \vdots  \\
       a_{m,1} & a_{m,2} & \cdots & a_{m,n}
     \end{array}
   \right] 
 + \left[
     \begin{array}{cccc}
       b_{1,1} & b_{1,2} & \cdots & b_{1,n} \\
       b_{2,1} & b_{2,2} & \cdots & b_{2,n} \\
       \vdots  & \vdots  &  \    & \vdots  \\
       b_{m,1} & b_{m,2} & \cdots & b_{m,n}
     \end{array}
   \right] 
$$
$$
 \ \ \ \ \ \ \ 
 = \left[
     \begin{array}{cccc}
       ( a_{1,1} + b_{1,1} ) & ( a_{1,2} + b_{1,2} )
                             & \cdots & ( a_{1,n} + b_{1,n} ) \\
       ( a_{2,1} + b_{2,1} ) & ( a_{2,2} + b_{2,2} )
                             & \cdots & ( a_{2,n} + b_{2,n} ) \\
       \vdots  & \vdots  &  \ 
                         & \vdots  \\
       ( a_{m,1} + b_{m,1} ) & ( a_{m,2} + b_{m,2} )
                             & \cdots & ( a_{m,n} + b_{m,n} )
     \end{array}
   \right] 
$$
The formula is just an extension of vector addition. The only restriction on the
operation of matrix addition is that the two matrices, $A$ and $B$, must be the
same size. We can rewrite the formula for addition as follows: If $C$ is a
matrix that is the result of adding matrices, $A$ and $B$, then the components
of $C$ satisfy
$$
  (C)_{i,j} = c_{i,j} = a_{i,j} + b_{i,j} = (A)_{i,j} + (B)_{i,j}
$$
In the previous equation, there are a couple of things going on. First, the
notation using $(A)_{i,j}$ refers to the component in the $i^{th}$ row and
$j^{th}$ column of $A$. We will also use lower cased letters to indicate the
same thing. The second thing to notice is that the component-wise formula just
given provides the formula we need in a routine that will add two matrices. A
code that provides the sum of two matrices is the following.
\vskip0.1in\hrule\vskip0.1in
\begin{verbatim}

     //
     // matrix addition code in C
     // -------------------------
     //
     double [][] matadd(double A[][], double B[][], int m, int n)
     {
       double C[m][n];
       //
       // a double loop is needed to compute the components of the sum of two
       // matrices
       // --------
       //
       for(int i=0; i<m; i++) {
         for(int j=0; j<n; j++) {
           C[i][j] = A[i][j] + B[i][j];
         }
       }
       //
       // return the value
       // ----------------
       //
       return C;
       //
     }

\end{verbatim}
\vskip0.1in\hrule\vskip0.1in
\noindent
Keep in mind that C/C++ and linux and Unix are in general case-sensittive. In
many applications, capitalized letters are assumed to be environment or global
variables. So, it might be a good idea to rewrite the code with different
variable names. For example,
\vskip0.1in\hrule\vskip0.1in
\begin{verbatim}

     //
     // an alternate version of doing a matrix addition
     // -----------------------------------------------
     //
     double [][] matadd(double amat[][], double bmat[][], int m, int n)
     {
       double cmat[m][n];
       //
       // a double loop is needed to compute the components of the sum of two
       // matrices
       // --------
       //
       for(int i=0; i<m; i++) {
         for(int j=0; j<n; j++) {
           cmat[i][j] = amat[i][j] + bmat[i][j];
         }
       }
       //
       // return the value
       // ----------------
       //
       return cmat;
       //
     }

\end{verbatim}
\vskip0.1in\hrule\vskip0.1in
\noindent
There are also some things that can be done to speed these types of operations
up through parallelism or using the ordering of values in the rows and columns
of a matrix stored in a specific programming language like C or Fortran. 
\vskip0.1in\hrule\vskip0.1in
\newpage
%%%%%%%%%%%%%%%%%%%%%%%%%%%%%%%%%%%%%%%%%%%%%%%%%%%%%%%%%%%%%%%%%%%%%%%%%%%%%%%%
%%%%%%%%%%%%%%%%%%%%%%%%%%%%%%%%%%%%%%%%%%%%%%%%%%%%%%%%%%%%%%%%%%%%%%%%%%%%%%%%
\vskip0.1in\hrule\vskip0.1in
\noindent
{\bf Vector/Matrix Operations: Matrix Subtraction}
\vskip0.1in\hrule\vskip0.1in
\noindent
Matrix subtraction involves a trivial modification of the matrix addition method
given in the previous section of these notes. The shorthand notation is the
following. A matrix, $C$, is the difference of two matrices $A$ and $B$ if we
write $c_{i,j}$ is the entry in $i^{th}$ row and $j^{th}$ column of $C$, then
$$
  c_{i,j} = a_{i,j} - b_{i,j}
$$
A code that will work using lower case variable names is the following.
\vskip0.1in\hrule\vskip0.1in
\begin{verbatim}

     //
     // code to compute the difference of two matrices of the same dimension
     // --------------------------------------------------------------------
     //
     double [][] matsub(double amat[][], double bmat[][], int m, int n)
     {
       double cmat[m][n];
       //
       // a double loop is needed to compute the components of the sum of two
       // matrices
       // --------
       //
       for(int i=0; i<m; i++) {
         for(int j=0; j<n; j++) {
           cmat[i][j] = amat[i][j] - bmat[i][j];
         }
       }
       //
       // return the value
       // ----------------
       //
       return cmat;
       //
     }

\end{verbatim}
\vskip0.1in\hrule\vskip0.1in
\noindent
The modifications to matrix addition involve a name change and a single change
in the code.
\vskip0.1in\hrule\vskip0.1in
\newpage
%%%%%%%%%%%%%%%%%%%%%%%%%%%%%%%%%%%%%%%%%%%%%%%%%%%%%%%%%%%%%%%%%%%%%%%%%%%%%%%%
%%%%%%%%%%%%%%%%%%%%%%%%%%%%%%%%%%%%%%%%%%%%%%%%%%%%%%%%%%%%%%%%%%%%%%%%%%%%%%%%
\vskip0.1in\hrule\vskip0.1in
\noindent
{\bf Vector/Matrix Operations: Scalar Matrix Multiplication}
\vskip0.1in\hrule\vskip0.1in
\noindent
This operation multiplies a scalar into a matrix by multiplying the number into
each entry in the matrix as follows.
$$
  \alpha\ A =
   \alpha\ \left[
     \begin{array}{cccc}
       a_{1,1} & a_{1,2} & \cdots & a_{1,n} \\
       a_{2,1} & a_{2,2} & \cdots & a_{2,n} \\
       \vdots  & \vdots  &  \    & \vdots  \\
       a_{m,1} & a_{m,2} & \cdots & a_{m,n}
     \end{array}
   \right] 
 = \left[
     \begin{array}{cccc}
       \alpha\ a_{1,1} & \alpha\ a_{1,2} & \cdots & \alpha\ a_{1,n} \\
       \alpha\ a_{2,1} & \alpha\ a_{2,2} & \cdots & \alpha\ a_{2,n} \\
       \vdots  & \vdots  &  \ & \vdots  \\
       \alpha\ a_{m,1} & \alpha\ a_{m,2} & \cdots & \alpha\ a_{m,n}
     \end{array}
   \right] 
$$
In terms of the components of the output, assume that a matrix, $B$, is the
result of multipling a matrix $A$ by a number. This implies
$$
  b_{i,j} = \alpha\ a_{i,j}
$$
and a code to complete this task might look like the following.
\vskip0.1in\hrule\vskip0.1in
\begin{verbatim}

     //
     // routine to compute a constant/scalar multiple of a matrix
     // ---------------------------------------------------------
     //
     double [][] scalmat(double alpha, double amat[][], int m, int n)
     {
       double bmat[m][n];
       //
       // a double loop is needed to compute the components of the sum of two
       // matrices
       // --------
       //
       for(int i=0; i<m; i++) {
         for(int j=0; j<n; j++) {
           bmat[i][j] = alpha * amat[i][j];
         }
       }
       //
       // return the value
       // ----------------
       //
       return bmat;
       //
     }

\end{verbatim}
\vskip0.1in\hrule\vskip0.1in
\noindent
The mathematical importance of matrix addition and scalar multiplication imply
that the set of all $m\times n$ real matrices is closed under these operations
which gives consistency of these operations.
\vskip0.1in\hrule\vskip0.1in
\newpage
%%%%%%%%%%%%%%%%%%%%%%%%%%%%%%%%%%%%%%%%%%%%%%%%%%%%%%%%%%%%%%%%%%%%%%%%%%%%%%%%
%%%%%%%%%%%%%%%%%%%%%%%%%%%%%%%%%%%%%%%%%%%%%%%%%%%%%%%%%%%%%%%%%%%%%%%%%%%%%%%%
\vskip0.1in\hrule\vskip0.1in
\noindent
{\bf Vector/Matrix Operations: The Determinant of a Matrix}
\vskip0.1in\hrule\vskip0.1in
\noindent
The determinant of a matrix is presented in most linear algebra courses and is
an indispensible tool in analyzing the properties of matrices. The standard way
of computing the determinant of a matrix is to use cofactor expansion. We will
review a couple of examples of computing determinants, but determinants are
computationally intensive and should not be computed unless it is absolutely
necessary. In some cases, the matrix has a special structure which lends itself
to an easy computation of the determinant. For example, the determinant of any
upper or lower triangular or is a diagonal matrix, the determinant is the
product of the diagonal values of the matrix. In general, the work needed
depends on the size of the matrix. If a matrix, $A$, is square with the number
of rows and columns is some integer, $n$. Then, the number of flops needed to
complete the computation of a determinant is $n!$. This is a large number of
flops.

For example, for a $2\times 2$ matrix, say
$$
  A = \left[
        \begin{array}{cc}
          a_{1,1} & a_{1,2}  \\
          a_{2,1} & a_{2,2} 
        \end{array}
      \right]
$$
the determinant is computed as
$$
  det(A) = \left| A \right| = a_{1,1}\ a_{2,2} - a_{1, 2} a_{2,1}
$$
and for the $3\times 3$ case, 
$$
  A = \left[
        \begin{array}{ccc}
          a_{1,1} & a_{1,2} & a_{1,3} \\
          a_{2,1} & a_{2,2} & a_{2,3} \\
          a_{3,1} & a_{3,2} & a_{3,3}
        \end{array}
      \right]
$$
the result is
$$
  det(A) = a_{1,1}\ \left( a_{2,2}\ a_{3, 3} - a_{3,2}\ a_{2,3} \right)
         - a_{2,1}\ \left( a_{1,2}\ a_{3, 3} - a_{3,2}\ a_{2,3} \right)
         + a_{3,1}\ \left( a_{1,2}\ a_{2, 3} - a_{2,2}\ a_{1,3} \right)
$$
The result can be extended to any size matrix, but that is the topic for a
linear algebra course. It is rarely the case that we will need to evaluate or
compute a determinant beyond a $3\times 3$ matrix. The main issue that we are
faced in computing determinants is the sheer amount of work needed to complete
the computation. The number of flops need is $O(n!)$ and we already know that
$n!$ gets large relatively quickly.
\vskip0.1in\hrule\vskip0.1in
\newpage
%%%%%%%%%%%%%%%%%%%%%%%%%%%%%%%%%%%%%%%%%%%%%%%%%%%%%%%%%%%%%%%%%%%%%%%%%%%%%%%%
%%%%%%%%%%%%%%%%%%%%%%%%%%%%%%%%%%%%%%%%%%%%%%%%%%%%%%%%%%%%%%%%%%%%%%%%%%%%%%%%
\vskip0.1in\hrule\vskip0.1in
\noindent
{\bf Vector/Matrix Operations: The Transpose of a Matrix}
\vskip0.1in\hrule\vskip0.1in
\noindent
For any matrix, we can compoute the transpose a matrix, $A\in\Re^{n\times n}$,
by interchanging rows and columns of the matrix. The shorthand definition of the
transpose is if $A\in\Re^{n\times n}$ is any matrix, then $B=A^T$ using the
definition
$$
  b_{i,j} = a_{j,i} 
$$
So, a routine for computing the transpose of a matrix will look like the
following.
\vskip0.1in\hrule\vskip0.1in
\begin{verbatim}

     //
     // computing the transpose of a matrix
     // -----------------------------------
     //
     double [][] mattrns(double amat[][], int m, int n)
     {
       double bmat[n][m];
       //
       // a double loop is needed to compute the components of the sum of two
       // matrices
       // --------
       //
       for(int i=0; i<m; i++) 
         for(int j=0; j<n; j++)
           bmat[j][i] = amat[i][j];
       //
       // return the value
       // ----------------
       //
       return bmat;
       //
     }

\end{verbatim}
\vskip0.1in\hrule\vskip0.1in
The code clearly switches the roles of the rows and columns in the output matrix
by interchanging the indices in the loop. Note also that the dimensions of the
temporary storage, {\bf bmat}, are switched when used to allocate enough space
to do the operation.

Note that this routine returns a matrix with the same number of entries as the
original. This may cause some issues when a code to solve problems with large
matrices is considered. As an example, suppose that we need to compute the
action of the tranpose on a vector. That is, something like
$$
  {\bf y} = A^T\ {\bf x}
$$
Even though it might be tempting to do the transpose first, physically
exchanging the rows and columns is not necessary. A code to do complete the
multiplication operation is the following.
\vskip0.1in\hrule\vskip0.1in
\begin{verbatim}

      double [] atrnx(double A[][], double x[], int m, int n)
      {
        double output[n];
        //
        // loop over the rows in the matrix A - columns of the transpose of A
        // ------------------------------------------------------------------
        //
        for(int i=0; i<n; i++) {
          //
          // initialize the output value
          // ---------------------------
          //
          output[i] = 0.0;
          //
          // loop over the columns of the matrix A.....
          // ------------------------------------------
          //
          for(int j=0; j<m; j++)
          {
            //
            // add the appropriate contribution
            // --------------------------------
            //
            output[i] = output[i] + a[j][i] * x[j];
          }
        }
        //
        return output;
      }

\end{verbatim}
\vskip0.1in\hrule\vskip0.1in
Note the change in the order of the indices used from the input matrix,
{\bf amat}, to the output matrix, {\bf bmat}. The main advantages are that we
do not need to double the storage in the process of explicitly storing the
transpose of the original matrix. In addition, the original matrix is retained
and can be resused for any other computations. Note that we will revisit matrix
multiplication later in this set of notes. You may need to review this after
visiting the section on matrix vector multiplication.
\vskip0.1in\hrule\vskip0.1in
\newpage
%%%%%%%%%%%%%%%%%%%%%%%%%%%%%%%%%%%%%%%%%%%%%%%%%%%%%%%%%%%%%%%%%%%%%%%%%%%%%%%%
%%%%%%%%%%%%%%%%%%%%%%%%%%%%%%%%%%%%%%%%%%%%%%%%%%%%%%%%%%%%%%%%%%%%%%%%%%%%%%%%
\vskip0.1in\hrule\vskip0.1in
\noindent
{\bf Vector/Matrix Operations: The Trace of a Matrix}
\vskip0.1in\hrule\vskip0.1in
\noindent
The trace of a matrix is just the sum of the entries on the main diagonal of
the matrix. This is useful in a limited number of computational codes, but is
presented for completeness of the presentation.
\vskip0.1in\hrule\vskip0.1in
\begin{verbatim}

     //
     // this is a simple routine to add up the diagonal entries in a square
     // matrix
     // ------
     //
     double [][] mattrc(double amat[][], int n)
     {
       double output;
       //
       // a single loop is enough to grab and add the diagonal values
       // -----------------------------------------------------------
       //
       for(int i=0; i<m; i++) output = output + amat[i][i];
       //
       return output;
       //
     }

\end{verbatim}
\vskip0.1in\hrule\vskip0.1in
One property of square matrices is that the trace of the matrix is the same as
the sum of the eigenvalues of the matrix. However, the trace, besides being
trivial to code up, is not used very often in computational mathematics.
\vskip0.1in\hrule\vskip0.1in
\newpage
%%%%%%%%%%%%%%%%%%%%%%%%%%%%%%%%%%%%%%%%%%%%%%%%%%%%%%%%%%%%%%%%%%%%%%%%%%%%%%%%
%%%%%%%%%%%%%%%%%%%%%%%%%%%%%%%%%%%%%%%%%%%%%%%%%%%%%%%%%%%%%%%%%%%%%%%%%%%%%%%%
\vskip0.1in\hrule\vskip0.1in
\noindent
{\bf Vector/Matrix Operations: Matrix-Norms}
\vskip0.1in\hrule\vskip0.1in
\noindent
An important concept in computational linear algebra is that of matrix norms.
Most error estimates in computational linear algebra are stated in terms of
inequalities involving the norm of a matrix. There are an infinite number of
measures of the norm or size of a matrix. One example of a norm is the Frobenius
norm. Given a square matrix, $A\in\Re^{n\times n}$, the definition of the norm
is
$$
  || A || = \left( \sum_{i=1}^n\ \sum_{j=1}^n | a_{i,j} |^2 \right)^{1/2}
$$
The Frobenius norm looks a bit like the $2$-norm for a vector. The similarity is
that you can think of concatenating the rows (or columns) of the matrix and then
apply the vector $2$-norm to the $n\times n$ length vector that results. The
Frobenius norm is used sparingly in computational linear algebra.

The main norms that are used are called induced norms and are defined in terms
of vector norms. Recall that the $p$-norm of a vector of length $n$ is defined
by
$$
  || {\bf v} ||_p = \left( \sum_{i=1}^n | v_i |^p \right)^{1/p}
$$
for $p=1, 2, \ldots$ and any vector, ${\bf v}\in\Re^n$. Note that the definition
changes slightly if we allow $p=\infty$. In this case
$$
  || {\bf v} ||_\infty = \max_{1\leq n} | v_i |
$$
The $p$-norms can be used to define norms of matrices. The induced $p$-norm of a
matrix, $A\in\Re^{m\times n}$, is defined by
$$
  || A ||_p = \max_{ || {\bf x} ||_p \neq 0 }
      {{
        || A\ {\bf x} ||_p
      }\over{
        || {\bf x} ||_p
      }}
$$ 
This definition is based on vector $p$-norms. Note that the $p$-norm in the
numerator of the expression is applied to vectors of length $m$, while the norm
in the denominator is applied to vectors of length $n$.

With a bit of work, we can prove that the induced matrix norms can be written 
as
$$
  || A ||_p = \max_{ || {\bf x} ||_p = 1 } || A\ {\bf x} ||_p
$$ 
This means that matrix norm can be determined on vectors of length one in the
corresponding $p$-norm. The set of vectors of length one identifies the
\lq\lq\ unit ball\rq\rq\ for the norm.

Although any $p\in (0,\infty)$ can be used, the most commonly used norms in
computational work are $p=1$, $p=2$, and $p=\infty$. These will be referred to
as the $1$-norm, the $2$ or Euclidean norm, and the $\infty$ or
\lq\lq\ sup\rq\rq-norm, respectively. For completeness, the $1$-norm defined
by
$$
  || A ||_1 = \max_{ || {\bf x} ||_1 = 1 } || A\ {\bf x} ||_1
$$
the $2$-norm or Euclidean norm is defined by
$$
  || A ||_2 = \max_{ || {\bf x} ||_2 = 1 } || A\ {\bf x} ||_2
$$
and the $\infty$-norm or sup-norm is defined by
$$
  || A ||_\infty = \max_{ || {\bf x} ||_\infty = 1 } || A\ {\bf x} ||_\infty
$$
The definitions of these norms are not necessarily very helpful. However, the
$1$-norm and the $\infty$-norm can be computed for any matrix. The $2$-norm
requires the computation of eigenvalues or singular values of the matrix. This
computation will be taken up in a later section in the lectures. 
\vskip0.1in\hrule\vskip0.1in
\newpage
%%%%%%%%%%%%%%%%%%%%%%%%%%%%%%%%%%%%%%%%%%%%%%%%%%%%%%%%%%%%%%%%%%%%%%%%%%%%%%%%
%%%%%%%%%%%%%%%%%%%%%%%%%%%%%%%%%%%%%%%%%%%%%%%%%%%%%%%%%%%%%%%%%%%%%%%%%%%%%%%%
\vskip0.1in\hrule\vskip0.1in
\noindent
{\bf Vector/Matrix Operations: Two Important Computable Matrix Norms}
\vskip0.1in\hrule\vskip0.1in
\noindent
A formula that allows us to compute the $1$-norm of a matrix, $A$, involves the
determination of the maximum absolute column sum. That is,
$$
  || A ||_1 = \max_{1\leq j\leq n} \sum_{i=1}^m\ | a_{i,j} |
$$
and the formula for the $\infty$-norm of the same matrix involves the
determination of the maximum absolute row sum. That is,
$$
  || A ||_\infty = \max_{1\leq i\leq m} \sum_{j=1}^n\ | a_{i,j} |
$$
Both of these norms can be computed with nothing more than the entries of the
matrix.

A code for computing the $1$-norm of a matrix might look like the following.
\vskip0.1in\hrule\vskip0.1in
\begin{verbatim}

     double l1matnrm(double a[][], int m, int n)
     {
       double output = 0.0;
       //
       // loop over the columns of the matrix
       // -----------------------------------
       //
       for(int j=0; j<n; j++) {
         //
         // provide a temporary storage value
         // ---------------------------------
         //
         double tmpval = 0.0;
         //
         // loop over the entries in the column adding in the absolute value
         // ----------------------------------------------------------------
         //
         for(int i=0; i<m; i++)
            tmpval = tmpval + abs( a[i][j];
         //
         // test this column sum for the maximum value so far
         // -------------------------------------------------
         //
         if(tmpval > output) output = tmpval;
       }
       //
       return output;
       //
     }

\end{verbatim}
\vskip0.1in\hrule\vskip0.1in
The algorithm for the $\infty$-norm is just as easy to write and might look like
the following.
\vskip0.1in\hrule\vskip0.1in
\begin{verbatim}

     double linfmatnrm(double a[][], int m, int n)
     {
       double output = 0.0;
       //
       // loop over the rows of the matrix
       // --------------------------------
       //
       for(int i=0; i<m; i++) {
         //
         // provide a temporary storage value
         // ---------------------------------
         //
         double tmpval = 0.0;
         //
         // loop over the entries in the row adding in the absolute value
         // -------------------------------------------------------------
         //
         for(int j=0; j<m; j++)
            tmpval = tmpval + abs( a[i][j];
         //
         // test this column sum for the maximum value so far
         // -------------------------------------------------
         //
         if(tmpval > output) output = tmpval;
       }
       //
       return output;
       //
     }

\end{verbatim}
\vskip0.1in\hrule\vskip0.1in
All we needed to do is switch the order of the two loops to move from the
matrix $1$-norm to the matrix $\infty$-norm.
\vskip0.1in\hrule\vskip0.1in
\newpage
%%%%%%%%%%%%%%%%%%%%%%%%%%%%%%%%%%%%%%%%%%%%%%%%%%%%%%%%%%%%%%%%%%%%%%%%%%%%%%%%
%%%%%%%%%%%%%%%%%%%%%%%%%%%%%%%%%%%%%%%%%%%%%%%%%%%%%%%%%%%%%%%%%%%%%%%%%%%%%%%%
\vskip0.1in\hrule\vskip0.1in
\noindent
{\bf Vector/Matrix Operations: Formal Matrix Norm Definition, Consistent Norms}
\vskip0.1in\hrule\vskip0.1in
\noindent
In the case of vector norms, we have definition of a norm that requires three
properties must be satisfied. Matrix norms also satisfy these requirements.
\vskip0.1in\hrule\vskip0.1in
\begin{definition}
   Suppose that $V$ is a vector space of matrices and ${\bf A}$ and ${\bf B}$
   are any two matrices in $V$. Also, assume $c$ is an arbitrary number/scalar.
   The norm of a matrix is a function
   $$
     || * || : V \rightarrow \Re
   $$
   such that
   \begin{enumerate}
     \item $||A||=0$ if and only if $A = 0$,
     \item $||c\ A|| = |c|\ ||A||$, and
     \item $||A + B || \leq ||A|| + ||B||$.
   \end{enumerate}
\end{definition}
\vskip0.1in\hrule\vskip0.1in
There are a lot of ways to define a norm for matrices. In many results in linear
algebra an additional property for a matrix norm is helpful. The following
definition adds a fourth property to the norm.
\begin{definition}
   Suppose that $V$ is a vector space of matrices and ${\bf A}$ and ${\bf B}$
   are any two matrices in $V$. Also, assume that the function
   $|| * || : V \rightarrow \Re$ is a norm. The norm is a consistent norm if the
   following additional property is satisfied
   $$
     || A\ B || \leq || A ||\ || B ||
   $$
\end{definition}
\vskip0.1in\hrule\vskip0.1in
\noindent
Note that all matrix $p$-norms as have already been defined are consistent
norms.
\vskip0.1in\hrule\vskip0.1in
\newpage
%%%%%%%%%%%%%%%%%%%%%%%%%%%%%%%%%%%%%%%%%%%%%%%%%%%%%%%%%%%%%%%%%%%%%%%%%%%%%%%%
%%%%%%%%%%%%%%%%%%%%%%%%%%%%%%%%%%%%%%%%%%%%%%%%%%%%%%%%%%%%%%%%%%%%%%%%%%%%%%%%
\vskip0.1in\hrule\vskip0.1in
\noindent
{\bf Vector/Matrix Operations: Comparison/Equivalence of Norms in Finite
Dimensions}
\vskip0.1in\hrule\vskip0.1in
\noindent
In previous sections of these lectures, we learned that the norm of a vector or
matrix is not a unique function. For example, the $p$-norms of vectors all give
a value that represents the length of a vector. In fact, there are infinitely
many norms that we can choose from. A valid question is which one should we use
or maybe, which norm is the best norm to use for a specific problem? In
computational mathematics, we will work in finite dimensional vector spaces. The
finite precision of number representation and finite precision of computer
arithmetic guarantee that this will always be the case. It turns out that the
answer to the question of which norm should be used is that it does not matter
due to the concept of equivalence of norms.

\vskip0.1in\hrule\vskip0.1in
\begin{definition}
  Two norms $||\cdot||_A$ and $||\cdot||_B$ are equivalent if there exist
  positive constants, $C_1$ and $C_2$, such that
  $$
    C_1\ ||{\bf x}||_A \leq ||{\bf x}||_B \leq C_2 ||{\bf x}||_A
  $$
  for all ${\bf x}$ in the underlying vector space. Note that this means that we
  can bound one of the norms in terms of the other. It can be shown that if the
  above inequalities are true, one can write.
  $$
    C_3\ ||{\bf x}||_B \leq ||{\bf x}||_A \leq C_4 ||{\bf x}||_B
  $$
  for positive constants, $C_3$ and $C_4$.
\end{definition}
\vskip0.1in\hrule\vskip0.1in
\noindent
An important result for our work is contained in the following theorem.
\begin{theorem}
  Suppose $V$ is a finite-dimensional matrix and $||\cdot ||_\alpha$ and
  $||\cdot ||_\beta$ are any two norms defined on $V$. Then the two norms are
  equivalent. That is, there exist two positive constants, $C_1$ and $C_2$, such
  that
  $$
    C_1 ||{\bf u}||_\alpha \leq ||{\bf u}||_\beta \leq C_2 ||{\bf u}||_\alpha
  $$
  for all ${\bf u}\in V$.
\end{theorem}
It is important to have some estimate of the relationahips between the norms in
terms of the bounding coefficients, $C_1$ and $C_2$.
\vskip0.1in\hrule\vskip0.1in
\newpage
%%%%%%%%%%%%%%%%%%%%%%%%%%%%%%%%%%%%%%%%%%%%%%%%%%%%%%%%%%%%%%%%%%%%%%%%%%%%%%%%
%%%%%%%%%%%%%%%%%%%%%%%%%%%%%%%%%%%%%%%%%%%%%%%%%%%%%%%%%%%%%%%%%%%%%%%%%%%%%%%%
\vskip0.1in\hrule\vskip0.1in
\noindent
{\bf Vector/Matrix Operations: Bounds Relating $p=1$, $p=\infty$, and $p=2$
Norms}
\vskip0.1in\hrule\vskip0.1in
As mentioned previously, the norms we use are basically the same up to constant
bounds. The $1$-norm, $2$-norm, and $\infty$-norm can be used to bound each
other as follows. In general, for the vector norms and any vector, ${\bf x}$.
$$
  || {\bf x} ||_\infty \leq || {\bf x} ||_2 \leq || {\bf x} ||_1
$$
and equivalence relationships
$$
  || {\bf x} ||_\infty \leq || {\bf x} ||_2 \leq \sqrt{n}\ || {\bf x} ||_\infty
$$
and
$$
  || {\bf x} ||_\infty \leq || {\bf x} ||_1 \leq n\ || {\bf x} ||_\infty
$$
Since the matrix $p$-norms are constructed from the vector $p$-norms, analogous
inequalities can be written down for matrices. For a matrix
$A\in\Re^{m\times n}$ some important inequalties are the following.
\vskip0.1in\hrule\vskip0.1in
\begin{enumerate}
  \item
    $$
      || A ||_2 \leq || A ||_F \leq \sqrt{n}\ || A ||_2
    $$
  \item
    $$
      \max_{i,j} | a_{i,j} | \leq || A ||_2
                         \leq \sqrt{mn}\ | \max_{i,j} | a_{i,j} |
    $$
  \item
    $$
      {1\over{\sqrt{n}}}\ || A ||_\infty\ \leq || A ||_2
                              \leq \sqrt{m}\ || A ||_\infty
    $$
  \item
    $$
      {1\over{\sqrt{m}}}\ || A ||_1\ \leq || A ||_2
                              \leq \sqrt{n}\ || A ||_1
    $$
\end{enumerate}
\vskip0.1in\hrule\vskip0.1in
where the subscript $F$ denotes the Frobenius norm. We should keep in mind that
the matrix $2$-norm is the norm that is being bounded by the matrix $1$-norm and
matrix $\infty$-norm. Since we can compute the other two norms this makes sense.
In addition, the matrix $2$-norm requires computing the eigenvalues or singular
values of the matrix. This take a lot of computational effort.
\vskip0.1in\hrule\vskip0.1in
\newpage
%%%%%%%%%%%%%%%%%%%%%%%%%%%%%%%%%%%%%%%%%%%%%%%%%%%%%%%%%%%%%%%%%%%%%%%%%%%%%%%%
%%%%%%%%%%%%%%%%%%%%%%%%%%%%%%%%%%%%%%%%%%%%%%%%%%%%%%%%%%%%%%%%%%%%%%%%%%%%%%%%
\vskip0.1in\hrule\vskip0.1in
\noindent
{\bf Vector/Matrix Operations: Matrix Condition Number}
\vskip0.1in\hrule\vskip0.1in
\noindent
A great many problems in computational linear algebra involve the solution of a
linear system of equations
$$
  A\ {\bf x} = {\bf b}
$$
where $A\in\Re^{n\times n}$ and ${\bf b}\in\Re^n$ are a given matrix and right
hand side for the linear system. From a theoretical point of view, if the
coefficient matrix, $A$, has an inverse, then a solution exists for the linear
system. Normally, we will use $A^{-1}$ to denote the inverse of $A$ and we can
write the equation
$$
  {\bf x} = A^{-1}\ {\bf b}
$$ 
Computationally, we (almost) never compute the inverse of the coefficient matrix
explicitly. Instead, we compute the action of the inverse of the matrix without
actually computing $A^{-1}$.

We would still like to have a measure how well an algorithm will work in the
solution of some linear algebra problem. Suppose that ${\bf x^*}$ is the exact
solution of the linear system above. That is,
$$
  A\ {\bf x^*} = {\bf b}\ \ \ \ \rightarrow\ \ \ \ {\bf x^*} = A^{-1}\ {\bf b}
$$
and suppose that ${\bf x}$ is an approximate solution for the linear system. 
The residual vector for the approximation is given by
$$
  {\bf r} = {\bf b} - A\ {\bf x}
$$
Subtracting the original linear system equation from the equation for the
residual vector equation just defined gives
$$
  {\bf r} - {\bf 0}
    = \left( {\bf b} - A\ {\bf x} \right)
                           - \left( {\bf b} - A\ {\bf x^*} \right)
    = A\ \left( {\bf x^*} - A\ {\bf x} \right)
$$
or rewriting gives the equation
$$
  A \left( {\bf x^*} - {\bf x} \right) = {\bf r}
   \ \ \ \ \rightarrow \ \ \ \ {\bf x^*} - {\bf x} = A^{-1}\ {\bf r}
$$
This gives a measure of the absolute error when we apply a vector norm to both
sides of the equation. That is,
$$
  \makebox{absolute error }\ \ || {\bf x^*} - {\bf x} || = || A^{-1}\ {\bf r} ||
$$
If we want to use the relative error measure, we will write
$$
  \makebox{relative error }\ \ {{
                                 || {\bf x^*} - {\bf x} ||
                               }\over{
                                 || {\bf x^*} ||
                               }}
$$
We need an estimate of the denominator.

To do this, consider the original system of equations
$$
  A\ {\bf x^*} = {\bf b}
$$
Taking vector norms of both sides of the equation using a consistent matrix 
norm gives
$$
  || {\bf b} || = || A\ {\bf x} || \leq || A ||\ ||{\bf x} ||
$$
Using some arithmetic and the fact the norm is consistnt, we can write
$$
  {1\over{|| {\bf x} ||}} \leq {{ || A || }\over{ || {\bf b} || }}
$$
So, the relative error can be bounded as follows.
$$
  {{ || {\bf x^*} - {\bf x} || }\over{ || {\bf x^*} || }}
    \leq {{
           || A^{-1} ||\ || {\bf r} ||\ || A ||
         }\over{
           || {\bf b} ||
         }}
     = || A^{-1} ||\ || A || \ 
         {{
           || {\bf r} ||
         }\over{
           || {\bf b} ||
         }}
$$
The product of the matrix norms is called the condition number of the matrix,
$A$.
\vskip0.1in\hrule\vskip0.1in
  \begin{definition}
    Suppose that $A$ is an invertible matrix. The condition number, $\kappa$,
    is defined by
    $$
      \kappa_\alpha(A) = || A^{-1} ||_\alpha\ || A ||_\alpha
    $$
    The value of the condition number is dependent on the norm chosen.
  \end{definition}
\vskip0.1in\hrule\vskip0.1in
\noindent
It should be noted that $\kappa_\alpha(A)$ is always greater than or equal to
one. Values of $\kappa\alpha(A)$ closer to one indicate that the matrix is
well-conditioned. The larger $\kappa_\alpha(A)$ is, the more poorly conditioned
the matrix is. One can take a look at the Hilbert matrix as an example of a
poorly conditioned matrix.

As an example, let's compute the condition number of a matrix using the matrix
$\infty$-norm. ..... need a matrix for the example
\vskip0.1in\hrule\vskip0.1in
\newpage
%%%%%%%%%%%%%%%%%%%%%%%%%%%%%%%%%%%%%%%%%%%%%%%%%%%%%%%%%%%%%%%%%%%%%%%%%%%%%%%%
%%%%%%%%%%%%%%%%%%%%%%%%%%%%%%%%%%%%%%%%%%%%%%%%%%%%%%%%%%%%%%%%%%%%%%%%%%%%%%%%
\vskip0.1in\hrule\vskip0.1in
\noindent
{\bf Vector/Matrix Operations: Matrix-Vector Multiplication}
\vskip0.1in\hrule\vskip0.1in
\noindent
Suppose that we are asked to solve a system of three linear equations that can
be written in the fowm
$$
  \begin{array}{rcl}
    a_{1,1}\ x_1\ + a_{1,2}\ x_2 + a_{1,3}\ x_3 & = & b_1 \\
    a_{2,1}\ x_1\ + a_{2,2}\ x_2 + a_{2,3}\ x_3 & = & b_2 \\
    a_{3,1}\ x_1\ + a_{3,2}\ x_2 + a_{3,3}\ x_3 & = & b_3 
  \end{array}
$$
where the coefficients, $a_{i,j}$, are given real numbers, $b_i$, are real
numbers, and the unknown variables are $x_i$ all of these for $i=1,2,3$. The
system of equations can be translated into a matrix-vector equation as follows.
$$
  A\ {\bf x} = {\bf b}
$$
where
$$
  A = \left[
        \begin{array}{ccc}
          a_{1,1}\ & a_{1,2}\ & a_{1,3} \\
          a_{2,1}\ & a_{2,2}\ & a_{2,3} \\
          a_{3,1}\ & a_{3,2}\ & a_{3,3}
        \end{array}
      \right], \ \ \ \ 
  {\bf x} = \left[
        \begin{array}{c}
          x_1 \\
          x_2 \\
          x_3
        \end{array}
      \right], \ \ \ \ 
  {\bf b} = \left[
        \begin{array}{c}
          b_1 \\
          b_2 \\
          b_3
        \end{array}
      \right]
$$
This is a place where a matrix-vector multiplication $A\ {\bf x}$ is part of the
expression.

In this section of the notes, we will see how to compute a matrix-vector
product. The operation basically, computes the (dot) product of each row and the
column vector ${\bf x}$. We can define this operation in terms of the output we
expect. For a general matrix-vector product
$$
  {\bf y} = A\ {\bf x}
$$
where $A\in\Re^{m\times n}$, $x\in\Re^n$, and $y\in\Re^m$. Thus the output is a
vector of length $m$ where
$$
  y_i = a_{i,1}\ x_1 + a_{i,2}\ x_2 + \cdots + a_{i,n}\ x_n = b_i
$$
for $i=1,2,\ldots,m$. This dot product definition can be translated into a
computer code as follows.
\vskip0.1in\hrule\vskip0.1in
\begin{verbatim}

     double matvecmlt(double a[][], double x[], int m, int n)
     {
       double output[m];
       //
       // loop over the rows of the matrix
       // --------------------------------
       //
       for(int i=0; i<m; i++) {
         //
         // iniitalize the component of the output vector, y
         // ------------------------------------------------
         //
         output[i] = 0.0;
         //
         // loop over the entries in the row each new contribution to the output
         // --------------------------------------------------------------------
         //
         for(int j=0; j<m; j++) {
           output[i] = output[i] + a[i][j] * x[j];
         }
         //
       }
       //
       return output;
       //
     }

\end{verbatim}
\vskip0.1in\hrule\vskip0.1in
\noindent
There are a couple of things to notice in this definition. The inner loop
computes the dot product of the $i^{th}$ row in the coefficient matrix with the
column vector, ${\bf x}$. One thing that can be done would be to initialize the
first value of ${\bf y}$ to the initial product. That is, write a code that
looks like the following.
\vskip0.1in\hrule\vskip0.1in
\begin{verbatim}

     double matvecmlt(double a[][], double x[], int m, int n)
     {
       double output[m];
       //
       // loop over the rows of the matrix
       // --------------------------------
       //
       for(int i=0 i<m; i++) {
         //
         // iniitalize the component of the output vector, y
         // ------------------------------------------------
         //
         output[i] = a[i][0] * x[0];
         //
         // loop over the entries in the row each new contribution to the output
         // --------------------------------------------------------------------
         //
         for(int j=1; j<m; j++) {
           output[i] = output[i] + a[i][j] * x[j];
         }
         //
       }
       //
       return output;
       //
     }

\end{verbatim}
\vskip0.1in\hrule\vskip0.1in
\noindent
All that this new routine does is to eliminate one extra computation by using
the first product in the dot product computing each new component. It might save
a little computation time to do this. Now that we have a code written for
computing a matrix vector product, you should go back to the section on the
transpose of a matrix, $A$, and review the example where the matrix of a
transpose was needed to compute a matrix vector product.
\vskip0.1in\hrule\vskip0.1in
\newpage
%%%%%%%%%%%%%%%%%%%%%%%%%%%%%%%%%%%%%%%%%%%%%%%%%%%%%%%%%%%%%%%%%%%%%%%%%%%%%%%%
%%%%%%%%%%%%%%%%%%%%%%%%%%%%%%%%%%%%%%%%%%%%%%%%%%%%%%%%%%%%%%%%%%%%%%%%%%%%%%%%
\vskip0.1in\hrule\vskip0.1in
\noindent
{\bf Vector/Matrix Operations: Matrix-Matrix Multiplication}
\vskip0.1in\hrule\vskip0.1in
\noindent
The ideas in the matrix-vector multiplication can be used easily to define
matrix-matrix multiplication. Suppose $A\in\Re^{m\times n}$ and
$B\in\Re^{n\times r}$ are two matrices, then the product
$$
  C = A\ B
$$
where $C\in\Re^{m\times r}$ is determined by the following formula.
$$
   C_{i,j} = \sum_{k=1}^n\ a_{i,k}\ b_{k,j}
$$
This is a formula that is covered in most linear algebra courses. The product
formula involves the computation of $m\times r$ dot products. As in the
matrix-vector multiplication the formula for $C_{i,j}$ can be used to build a
computer code to do this work.
\vskip0.1in\hrule\vskip0.1in
\begin{verbatim}

     double matmatmlt(double a[][], double b[][], int m, int n, int r)
     {
       double output[m][r];
       //
       // loop over the rows of the matrix
       // --------------------------------
       //
       for(int i=0 i<m; i++) {
         for(int j=0 j<r; j++) {
           //
           // initial this component of the output
           // ------------------------------------
           //
           output[i][j] = 0.0;
           //
           // loop over the entries in the output matrix
           // ------------------------------------------
           //
           for(int k=1; k<n; k++) {
             output[i][j] = output[i][j] + a[i][k] * b[k][j];
           }
         }
         //
       }
       //
       return output;
       //
     }

\end{verbatim}
\vskip0.1in\hrule\vskip0.1in
\noindent
There are a number of ways to structure this code. In fact, there are 6 ways to
write this code. For this section, the code above will provide the product of
two matrices. The next section will reorder the loops in the original,
{\it matmatmlt}, to come up with alternate codes for the same operation that
may or may not improve performance.
\vskip0.1in\hrule\vskip0.1in
\newpage
%%%%%%%%%%%%%%%%%%%%%%%%%%%%%%%%%%%%%%%%%%%%%%%%%%%%%%%%%%%%%%%%%%%%%%%%%%%%%%%%
%%%%%%%%%%%%%%%%%%%%%%%%%%%%%%%%%%%%%%%%%%%%%%%%%%%%%%%%%%%%%%%%%%%%%%%%%%%%%%%%
\vskip0.1in\hrule\vskip0.1in
\noindent
{\bf Vector/Matrix Operations: Some Building Blocks for Computational Linear
Algebra - SAXPY}
\vskip0.1in\hrule\vskip0.1in
\noindent
In the world of computational mathematics, one approaches the construction of
complicated algorithms by starting out with simple building block algorithms and
codes to match. We have already covered a number of these in previous sections.
However, there are some algorithms that come standard with packages designed to
solve computational linear algebra problems. The examples in this and the next
section can be found in just about any document that covers Linpack or LAPACK.
In this section, we will consider the SAXPY operation for updating
a vector within a loop structure.

The moniker \lq\lq\ SAXPY\rq\rq\ stands for {\bf S}calar {\bf A} {\bf X}
{\bf P}lus {\bf Y} and is defined by the assignment operation
$$
  {\bf y} \leftarrow a\ {\bf x} + {\bf y}
$$ 
that can be written component-wise as
$$
  y_i \leftarrow a\ x_i + y_i
$$ 
for $i=1,2,\ldots, n$. This last form gives the inner most statement of a loop
for performing the update. So, a code can be constructed using the last formula
and might look like the following.
\vskip0.1in\hrule\vskip0.1in
\begin{verbatim}

     double * saxpy(double y[], double x[], double a, int n)
     {
       //
       // loop over the rows of the matrix
       // --------------------------------
       //
       for(int i=0; i<n; i++)
         y[i] = a * x[i] + y[i];
       //
       return y;
       //
     }

\end{verbatim}
\vskip0.1in\hrule\vskip0.1in
This may seem very simple. However, this operation can be used in all kinds of
computational linear algorithms and as such this code provides a building block
for other computational algorithms to use. Note that this is an update operation
for the vector ${\bf y}$. This means ${\bf y}$ will be overwritten each and
every time the routine is called. This is useful in algorithms that involve
iterating on a matrix form.
\vskip0.1in\hrule\vskip0.1in
\newpage
%%%%%%%%%%%%%%%%%%%%%%%%%%%%%%%%%%%%%%%%%%%%%%%%%%%%%%%%%%%%%%%%%%%%%%%%%%%%%%%%
%%%%%%%%%%%%%%%%%%%%%%%%%%%%%%%%%%%%%%%%%%%%%%%%%%%%%%%%%%%%%%%%%%%%%%%%%%%%%%%%
\vskip0.1in\hrule\vskip0.1in
\noindent
{\bf Vector/Matrix Operations: The GAXPY Operation}
\vskip0.1in\hrule\vskip0.1in
\noindent
Another standard operation is a generalization of the SAXPY operation that was
discussed in the last section of these notes. A formula that defines this
operation is
$$
  {\bf y} = A\ {\bf x} + {\bf y}
$$
So, instead of updating a vector ${\bf y}$ with a scalar multiple of another
vector, $a{\bf x}$, the formula will update a vector with $A{\bf x}$, output
from a matrix-vector multiplication operation. A component form of the GAXPY
operation is the following.
$$
  y_i = \sum_{j=1}^n\ a_{i,j}\ x_j + y_i
$$
Note that this version of the update does the work row by row in the column
vector, ${\bf y}$. A code to perform a GAXPY updata operation might look like
the following.
\vskip0.1in\hrule\vskip0.1in
\begin{verbatim}

     double * gaxpy(double y[], double a[][], double x[], int m, int n)
     {
       //
       // loop over the rows of the matrix
       // --------------------------------
       //
       for(int i=0; i<n; i++) {
         for(int j=0; j<n; j++) y[i] = a[i][j] * x[j] + y[i];
       }
       //
       return y;
       //
     }

\end{verbatim}
\vskip0.1in\hrule\vskip0.1in
\noindent
Using today's computational hardware, a great deal of emphasis involves the use
of parallel computer hardware. This includes multi-core programming, GPU
programming, and other parallel and vector computer resources. The code listing
above may need some work in order to create a parallel version of the algorithm.
An alternative code mught look like
\vskip0.1in\hrule\vskip0.1in
\begin{verbatim}

     double * gaxpy(double y[], double a[][], double x[], int m, int n)
     {
       //
       // loop over the rows of the matrix
       // --------------------------------
       //
       for(int i=0; i<n; i++) {
         #pragma omp parallel for
         for(int j=0; j<n; j++) y[i] += a[i][j] * x[j];
       }
       //
       return y;
       //
     }

\end{verbatim}
\vskip0.1in\hrule\vskip0.1in
\noindent
There is a minor difference in how the code is written. The increment operator
is used and when combined with the OpenMp construct, the code can be compiled
with multi-core coding features. The extra line in the code,
\vskip0.1in\hrule\vskip0.1in
\begin{verbatim}

         #pragma omp parallel for

\end{verbatim}
\vskip0.1in\hrule\vskip0.1in
\noindent
is a compiler directive for OpenMP. The content on how to implement the code
using OpenMP (and other parallel programming constructs) will be covered in
other places in the notes.
\vskip0.1in\hrule\vskip0.1in
\newpage
%%%%%%%%%%%%%%%%%%%%%%%%%%%%%%%%%%%%%%%%%%%%%%%%%%%%%%%%%%%%%%%%%%%%%%%%%%%%%%%%
%%%%%%%%%%%%%%%%%%%%%%%%%%%%%%%%%%%%%%%%%%%%%%%%%%%%%%%%%%%%%%%%%%%%%%%%%%%%%%%%
\vskip0.1in\hrule\vskip0.1in
\noindent
{\bf Vector/Matrix Operations: Matrix Outer-Product Update}
\vskip0.1in\hrule\vskip0.1in
\noindent
An outer product update for a matrix involves two vectors, ${\bf u}\in\Re^{m}$,
and ${\bf v}\in\Re^{n}$ and computes the update
$$
  A\ \leftarrow\ A + {\bf u}\ {\bf v}^T
$$
The second term in the update creates a matrix of the same size as the matrix
$A$. The result of this operation is a matrix that is stored back into $A$. Of
course, this means that $A$ is overwritten in the update process.

Component-wise, we can write
$$
  A_{i,j} = A_{i,j} + u_i\ v_j
$$
A C code for implementing this algorithm is the following.
\vskip0.1in\hrule\vskip0.1in
\begin{verbatim}

     double * matupdate(double a[][], double u[], double v[], int m, int n)
     {
       double output[m,n];
       //
       // loop over the rows of the matrix
       // --------------------------------
       //
       for(int i=0 i<m; i++) {
         for(int j=0; j<n; j++) output[i][j] = a[i][k] + u[i] * v[j];
       }
       //
       return output;
       //
     }

\end{verbatim}
\vskip0.1in\hrule\vskip0.1in
\noindent
We will use this type of outer-product update for implementing a matrix-matrix
update in the next section of these notes.
\vskip0.1in\hrule\vskip0.1in
\newpage
%%%%%%%%%%%%%%%%%%%%%%%%%%%%%%%%%%%%%%%%%%%%%%%%%%%%%%%%%%%%%%%%%%%%%%%%%%%%%%%%
%%%%%%%%%%%%%%%%%%%%%%%%%%%%%%%%%%%%%%%%%%%%%%%%%%%%%%%%%%%%%%%%%%%%%%%%%%%%%%%%
\vskip0.1in\hrule\vskip0.1in
\noindent
{\bf Vector/Matrix Operations: Matrix Multiplication Update Code}
\vskip0.1in\hrule\vskip0.1in
\noindent
As a last example of how to write codes that are helpful in computational
linear algebra problems, we consider a matrix update formula involving the
computation of a matrix-matrix product. Given matrices $A\in\Re^{m\times p}$,
$B\in\Re^{p\times n}$, and $C\in\Re^{m\times n}$ a useful update formula in
computational mathematics is
$$
  C\ \leftarrow\ A\ B + C
$$
This is a recursive formula for updating the matrix $C$. The number of flops
needed to update $C$ is order $n^3$ for square matrices.

A component-wise formula for the update can be written as
$$
  C_{i,j}\ = \sum{i=1}^k\ A_{i,k}\ B_{k,j} + C_{i,j}
$$
for $i=1,2,\ldots,m$ and $j=1,2,\ldots, n$. A code to perform this operation
might look like the following.
\vskip0.1in\hrule\vskip0.1in
\begin{verbatim}

     double * matupdate(double c[][], double a[][], double b[][],
                        int m, int p, int n)
     {
       double output[m,n];
       //
       // loop over the rows of the matrix
       // --------------------------------
       //
       for(int i=0 i<m; i++) {
         //
         // loop through the columns in the output matrix
         // ---------------------------------------------
         //
         for(int j=0; j<n; j++) {
           //
           // iniitalize the component of the output vector, y
           // ------------------------------------------------
           //
           for(int k=0; k<p; k++)
                       output[i][j] = a[i][k] * b[k][j] + c[i][j];
         }
       }
       //
       return output;
       //
     }

\end{verbatim}
\vskip0.1in\hrule\vskip0.1in
This version of the code is called a dot-product implementation of the
algorithm. This is due to the inner most of the triple-nested loop in the code
is a dot product of a row in $A$ and a column of $B$. There are actually a total
of six  different versions of this code. The different versions of the algorithm
are created by interchanging the three loops. In exact arithmetic, all six of
the code will produce the same result. If the computations are rearranged the
output will change slightly due to round-off error. In addition, due to the way
computer languages set up storage of arrays, most notably C and Fortran, one of
the codes will run faster than another version.

One rearrangement of the loops just interchanges the outer most loops and looks
like
\vskip0.1in\hrule\vskip0.1in
\begin{verbatim}

     double * matupdate(double c[][], double a[][], double b[][],
                        int m, int p, int n)
     {
       double output[m,n];
       //
       // loop over the rows of the matrix
       // --------------------------------
       //
       for(int i=0 i<m; i++) {
         //
         // loop through the columns in the output matrix
         // ---------------------------------------------
         //
         for(int j=0; j<n; j++) {
           //
           // loop over the inner dimension of the matrix-matrix multipliction
           // ----------------------------------------------------------------
           //
           for(int k=0; k<p; k++)
                       output[i][j] = a[i][k] * b[k][j] + c[i][j];
         }
       }
       //
       return output;
       //
     }

\end{verbatim}
\vskip0.1in\hrule\vskip0.1in
\noindent
As we can see, the inner most loop does not change. This gives a second
dot-product version of the algorithm due to the fact that the inner most loop is
still just a dot product operation. The difference is in the order in which the
components of the output matrix are compouted.

Another version of the same code is the following changes the loop structure so
that the inner most loop is a matrix outer-product update operation. 
versions
\vskip0.1in\hrule\vskip0.1in
\begin{verbatim}

     double * matupdate(double c[][], double a[][], double b[][],
                        int m, int p, int n)
     {
       double output[m,n];
       //
       // loop over the inner dimension of the matrix-matrix multiplication
       // -----------------------------------------------------------------
       //
       for(int k=0 k<p; k++) {
         //
         // loop through the columns in the output matrix
         // ---------------------------------------------
         //
         for(int j=0; j<n; j++) {
           //
           // loop over the row entries
           // -------------------------
           //
           for(int i=0; i<m; i++)
                       output[i][j] = a[i][k] * b[k][j] + c[i][j];
         }
       }
       //
       return output;
       //
     }

\end{verbatim}
\vskip0.1in\hrule\vskip0.1in
\noindent
We can see that the only change in the code is the order of the loops. As
mentioned in Matrix Computations by Golub and Van Loan the difference is in the
data flow within the code. How the entries are accessed in the code is the only
feature that can change the code. The same number of operations are performed,
just ordered differently. There are differences in how different hardware and
software access data and thus the different versions may provide different
levels of performance.

A final SAXPY version of the matrix update in this section looks like the
following.
\vskip0.1in\hrule\vskip0.1in
\begin{verbatim}

     double * matupdate(double c[][], double a[][], double b[][],
                        int m, int p, int n)
     {
       double output[m,n];
       //
       // loop over the columns of the matrix
       // -----------------------------------
       //
       for(int j=0; j<n; j++) {
         //
         // loop over the inside dimension in the matrix-matrix multiplication
         // ------------------------------------------------------------------
         //
         for(int k=0 k<p; k++) {
           //
           // loop through the rows in the output matrix
           // ------------------------------------------
           //
           for(int i=0; i<m; i++)
                       output[i][j] = a[i][k] * b[k][j] + c[i][j];
         }
       }
       //
       return output;
       //
     }

\end{verbatim}
\vskip0.1in\hrule\vskip0.1in
\noindent
Note that there are two dot-product versions, two SAXPY versions, and two
outer-product versions of this algorithm. Any of these six codes can be written
to implement the given matrix-matrix multiplication update. However, if
performance increases with one of the forms over the others, it should make
sense to take advantage of that difference. The coding difference is just
swapping three loops in the code.
\vskip0.1in\hrule\vskip0.1in
\newpage
%%%%%%%%%%%%%%%%%%%%%%%%%%%%%%%%%%%%%%%%%%%%%%%%%%%%%%%%%%%%%%%%%%%%%%%%%%%%%%%%
%%%%%%%%%%%%%%%%%%%%%%%%%%%%%%%%%%%%%%%%%%%%%%%%%%%%%%%%%%%%%%%%%%%%%%%%%%%%%%%%
\end{document}
