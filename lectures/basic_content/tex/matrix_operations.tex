\documentclass[10pt,fleqn]{article}
%\usepackage{graphicx}


\setlength{\topmargin}{-.75in}
\addtolength{\textheight}{2.00in}
\setlength{\oddsidemargin}{.00in}
\addtolength{\textwidth}{.75in}

\title{Math 4610 Lecture Notes \\
            \ \\
        Vector and Matrix Operations in Computational Mathematics
  \footnote{These notes are part of an Open Resource Educational project
            sponsored by Utah State University}}

\author{Joe Koebbe}

\nofiles

\pagestyle{empty}

\setlength{\parindent}{0in}

% new math commands


\setlength{\oddsidemargin}{-0.25in}
\setlength{\evensidemargin}{-0.25in}
\setlength{\textwidth}{6.75in}
\setlength{\headheight}{0.0in}
\setlength{\topmargin}{-0.25in}
\setlength{\textheight}{9.00in}

\makeindex

\usepackage{mathrsfs}

%\usepackage[pdftex]{graphicx}
\usepackage{epstopdf}

\newcounter{beans}

\newcommand{\ds}{\displaystyle}
\newcommand{\limit}[2]{\displaystyle\lim_{#1\to#2}}

\newcommand{\binomial}[2]{\ \left( \begin{array}{c}
                                  #1 \\
                                  #2
                                 \end{array}
                            \right) \
                         }
\newcommand{\ExampleRule}[2]
  {
  \noindent
  \rule{\linewidth}{1pt}
  \begin{example}
    #1
    \label{#2}
  \end{example}
  \rule{\linewidth}{1pt}
  \vskip0.125in
  }

\newcommand{\defbox}[1]
  {
   \ \\
   \noindent
   \setlength\fboxrule{1pt}
   \fbox{
        \begin{minipage}{6.5in}
          #1
        \end{minipage}
        }
   \ \\
  }
\newcommand{\verysmallworkbox}[1]
  {
   \ \\
   \noindent
   \setlength\fboxrule{1pt}
   \fbox{
        \begin{minipage}{6.5in}
           #1
           \ \\
           \vskip0.5in \ \\
           \ \\
        \end{minipage}
        }
   \ \\
  }
\newcommand{\smallworkbox}[1]
  {
   \ \\
   \noindent
   \setlength\fboxrule{1pt}
   \fbox{
        \begin{minipage}{6.5in}
           #1
           \ \\
           \vskip2.5in \ \\
           \ \\
        \end{minipage}
        }
   \ \\
  }
\newcommand{\halfworkbox}[1]
  {
   \ \\
   \noindent
   \setlength\fboxrule{1pt}
   \fbox{
        \begin{minipage}{6.5in}
           #1 \hfill
           \ \\
           \vskip3.25in \ \\
           \ \\
        \end{minipage}
        }
   \ \\
  }
\newcommand{\largeworkbox}[1]
  {
   \ \\
   \noindent
   \setlength\fboxrule{1pt}
   \fbox{
        \begin{minipage}{6.5in}
           #1
           \ \\
           \vskip7.5in \ \\
           \ \\
        \end{minipage}
        }
   \ \\
  }
\newcommand{\flexworkbox}[2]
  {
   \ \\
   \noindent
   \setlength\fboxrule{1pt}
   \fbox{
        \begin{minipage}{6.5in}
           #1
           \ \\

           \vskip#2 \ \\
           \ \\
        \end{minipage}
        }
   \ \\
  }


% symbols for sets of numbers

\newcommand{\natnumb}{$\cal N$}
\newcommand{\whonumb}{$\cal W$}
\newcommand{\intnumb}{$\cal Z$}
\newcommand{\ratnumb}{$\cal Q$}
\newcommand{\irrnumb}{$\cal I$}
\newcommand{\realnumb}{$\cal R$}
\newcommand{\cmplxnumb}{$\cal C$}

% misc. commands

\newcommand{\mma}{{\it Mathematica}}
\newcommand{\sech}{\mbox{ sech}}
 
\newtheorem{theorem}{Theorem}
\newtheorem{example}{Example}
\newtheorem{definition}{Definition}
\newtheorem{problem}{Problem}

\setcounter{secnumdepth}{2}
\setcounter{tocdepth}{4}


\begin{document}
\maketitle
\newpage

%%%%%%%%%%%%%%%%%%%%%%%%%%%%%%%%%%%%%%%%%%%%%%%%%%%%%%%%%%%%%%%%%%%%%%%%%%%%%%%%
%%%%%%%%%%%%%%%%%%%%%%%%%%%%%%%%%%%%%%%%%%%%%%%%%%%%%%%%%%%%%%%%%%%%%%%%%%%%%%%%
\vskip0.1in\hrule\vskip0.1in
\noindent
{\bf Vector/Matrix Operations: Basic Operations on Vectors}
\vskip0.1in\hrule\vskip0.1in
\noindent
The central algorithms used in most computer simulations involve computer codes
that perform basic vector and matrix operations. Operations like matrix-vector
multiplications are central to methods for solving linear systems of equations
and determination of eigenvalues and eigenvectors. In this section of the class
notes, we will cover a number of basic vector and matrix operations.

Suppose that we want to compute the solution of a linear system of equations in
matrix form
$$
  A\ {\bf x} = {\bf b}
$$
where $A$ is the coefficient matrix, ${\bf b}$ is an input or right hand side
vector, and we want to determine ${\bf x}$, a vector of unknowns. Due to issues
related to roundoff error and machine precision representation of numbers, any
vector we obtain will most likely be at best an approximation of the solution.

Since we know there will be errors, we will need some way to measure or compute
the error associated with any approximation obtained. As we will see, there are
infinitely many ways to measure errors. However, we will settle for three
standard meausres of error introduced in the next section of the notes. To
implement the error definitions, we will need to write codes that perform basic
vector operations one would learn in a standard linear algebra course in a
college curriculum. We will begin with basic algebraic operations involvinv
vectors such as addition and scalar multiplication and then will proceed with
the development codes for computing the magnitude/length of a vector and will
end up with matrix algebra and operations.
\vskip0.1in\hrule\vskip0.1in
\newpage
%%%%%%%%%%%%%%%%%%%%%%%%%%%%%%%%%%%%%%%%%%%%%%%%%%%%%%%%%%%%%%%%%%%%%%%%%%%%%%%%
%%%%%%%%%%%%%%%%%%%%%%%%%%%%%%%%%%%%%%%%%%%%%%%%%%%%%%%%%%%%%%%%%%%%%%%%%%%%%%%%
\vskip0.1in\hrule\vskip0.1in
\noindent
{\bf Vector/Matrix Operations: Basic Operations on Vectors}
\vskip0.1in\hrule\vskip0.1in
\noindent
For a set of objects to form a vector space a long list of properties must be
satisfied. In these note, we will be working primarily with vectors in $\Re^n$
or possibly the space of vectors with complex entries. As such, we will assume
that we are working with standard vector spaces. Any vector space or subspace
must satisfy closure of addition and closure of scalar multiplication. This
means that if $V$ is a vector space and ${\bf u}$ and ${\bf v}$ are in the
space, then the sum of the vectors is in $V$. That is, ${\bf w}$, defined by
$$
  {\bf w} = {\bf u} + {\bf v}
$$
is also in the vector space. Also, if ${\bf v}\in V$ and $a$ is any 
number/scalar then ${\bf u}$ defined by
$$
  {\bf v} = a\ {\bf u}
$$
is also in $V$. Based on this idea, we should at least implement a pair of
codes. One that computes the sum of two vectors and one that computes a scalar
multiple of a vector. 

Almost all of the work we will do will involve real vectors of length $n$. A
vector will be of the form $(v_1, v_2, \ldots, v_n)^T$ where each component,
$v_i, i=1,\ldots, n$ is a real number. The superscript, $T$, denotes the
transpose. It should be noted that there is a differnce between row and column
vectors in linear algebra. In our vector notation, we will by default assume
that a given vector is a column vector. Transposing a column vector results in
a row vector and vice versa. This convention lends itself to operations like
matrix-vector multiplication. 

So, let's start with the sum of two vectors. The conventional definition of
addition of vectors is a componentwise definition. If ${\bf u}$ and ${\bf v}$
are two vectors of length $n$. Then
$$
  {\bf u} + {\bf v} = 
       \left[
         \begin{array}{c}
           u_1 \\
           u_2 \\
           \vdots \\
           u_n
         \end{array}
       \right]
     + \left[
         \begin{array}{c}
           v_1 \\
           v_2 \\
           \vdots \\
           v_n
         \end{array}
       \right]
     = \left[
         \begin{array}{c}
           u_1 + v_1 \\
           u_2 + v_2 \\
           \vdots \\
           u_n + v_n
         \end{array}
       \right]
$$
For completeness, we can also write down a relationship between row and column
vectors. That is,
$$
  {\bf v}^T = 
       \left[
         \begin{array}{c}
           v_1 \\
           v_2 \\
           \vdots \\
           v_n
         \end{array}
       \right]^T
     = \left[
         v_1, v_2, \cdots v_n
       \right]
$$
and
\noindent
$$
  \left[ v_1, v_2, \cdots v_n \right]^T
     = \left[
         \begin{array}{c}
           v_1 \\
           v_2 \\
           \vdots \\
           v_n
         \end{array}
       \right] = {\bf v}
$$
This is one of the easiest operations that we can code. There are a few things
that might need checking. For example, we might need to check the dimension of
the two vectors. Without any checks on compabilitiy of dimensions, the code,
written in Java might look like the following.
\begin{verbatim}

     public double[] vecadd(double[] u, double[] v) {
       //
       // get the dimension of the first vector and instantiate some storage for
       // the output vector
       // -----------------
       //
       int n = u.length;
       double [] output = new double[n];
       //
       // compute the componentwise sum for the vector
       // --------------------------------------------
       //
       for(int i=0; i<n; i++) output[i] = u[i] + v[i];
       //
       // return the output
       // -----------------
       //
       return output;
       //
     }

\end{verbatim}
Before coding a scalar multiple method for our library, it would be best to
write a code for computing the difference in a pair of vectors. Given two
vectors ${\bf u}$ and ${\bf v}$, the formula for the difference is the
following.
$$
  {\bf u} - {\bf v} = 
       \left[
         \begin{array}{c}
           u_1 \\
           u_2 \\
           \vdots \\
           u_n
         \end{array}
       \right]
     - \left[
         \begin{array}{c}
           v_1 \\
           v_2 \\
           \vdots \\
           v_n
         \end{array}
       \right]
     = \left[
         \begin{array}{c}
           u_1 - v_1 \\
           u_2 - v_2 \\
           \vdots \\
           u_n - v_n
         \end{array}
       \right]
$$
It is an easy exercise to write a code that will compute the difference of two
vectors. All that needs to happen is to give the routine a unique name and
change the addition to a subtraction. For exaxmple,
\begin{verbatim}

     public double[] vecsub(double[] u, double[] v) {
       //
       // get the dimension of the first vector and instantiate some storage for
       // the output vector
       // -----------------
       //
       int n = u.length;
       double [] output = new double[n];
       //
       // compute the componentwise sum for the vector
       // --------------------------------------------
       //
       for(int i=0; i<n; i++) output[i] = u[i] - v[i];
       //
       // return the output
       // -----------------
       //
       return output;
       //
     }

\end{verbatim}
This particular routine can be used as a means to compute the difference between
two vectors as a part of measuring errors in vector approximations. Note that
the dimensions of the two vectors in the subtraction of vectors has not been
checked. This means it is up to the user to ensuere that the two vectors have
the same length.

The last code to write in this section is the scalar multiplication code. The
vector operation is defined as follows. For any vector, ${\bf v}$, and number,
$a$, scalar multiplication is defined by
$$
  a\ {\bf v} = 
       a\ \left[
            \begin{array}{c}
              v_1 \\
              v_2 \\
              \vdots \\
              v_n
            \end{array}
          \right]
     = \left[
         \begin{array}{c}
           a\ v_1 \\
           a\ v_2 \\
           \vdots \\
           a\ v_n
         \end{array}
       \right]
$$
and the code to implement this operation is given by
\begin{verbatim}

     public double[] scalvec(double a, double[] v) {
       //
       // get the dimension of the first vector and instantiate some storage for
       // the output vector
       // -----------------
       //
       int n = u.length;
       double [] output = new double[n];
       //
       // compute the componentwise sum for the vector
       // --------------------------------------------
       //
       for(int i=0; i<n; i++) output[i] = a * v[i];
       //
       // return the output
       // -----------------
       //
       return output;
       //
     }

\end{verbatim}
\vskip0.1in\hrule\vskip0.1in
In the scalar multiplication, the length of the vector can be any integer and as
long as the vector is not null, the method will work. However, it might be
necessary to test to see if the vector actually exists. If a null vector is
passed to any method, there is likely to be a problem. So, in a software manual
you would need to warn the user of this and any other assumptions that might
cause problems.

The methods developed in this section of the notes provides a starting point for
a number of other methods. It will be easy to cut and paste and rename methods
to do most of the basic matrix operations outlined in these notes.
\newpage
%%%%%%%%%%%%%%%%%%%%%%%%%%%%%%%%%%%%%%%%%%%%%%%%%%%%%%%%%%%%%%%%%%%%%%%%%%%%%%%%
%%%%%%%%%%%%%%%%%%%%%%%%%%%%%%%%%%%%%%%%%%%%%%%%%%%%%%%%%%%%%%%%%%%%%%%%%%%%%%%%
\vskip0.1in\hrule\vskip0.1in
\noindent
{\bf Vector/Matrix Operations: The Euclidean Length of a Vector}
\vskip0.1in\hrule\vskip0.1in
\noindent
In many problems it will be necessary to measure the length or magnitude of a
vector (at least so say the villian in Despicable Me). So, we need some way to
determine the length. Actually, there are infinitely many ways to do this. The
approach most students see first involves using the Euclidean distance between
two points used to define a vector. From any standard linear algebra course, the
Euclidean length can be defined as follows.
$$
  || v || = ( v_1^2 + v_2^2 + \cdots + v_n^2 )^{1/2}
$$
where the notation, $||*||$, provides a mathematical notation for the magnitude
of the vector. The notation emphasizes the difference between the absolute 
value of the difference of two real numbers and the length of a vector. This can
be implemented easily into a method or routine that will return this length of a
vector.
\begin{verbatim}

     public double l2norm(double[] v) {
       double sum = 0.0;
       //
       // extract the length of the vector
       // --------------------------------
       //
       int n = v.length;
       //
       // compute the sum of squares
       // --------------------------
       //
       for(int i=0; i<n; i++) sum = sum + v[i] * v[i];
       //
       // return the value
       // ----------------
       //
       return Math.sqrt(sum);
       //
     }

\end{verbatim}
Note that the Java package including the square root function defines the last
step in the code. It is best to use an intrinsic for the square root since
implementing our own version of a square root method would take some time.
\newpage
%%%%%%%%%%%%%%%%%%%%%%%%%%%%%%%%%%%%%%%%%%%%%%%%%%%%%%%%%%%%%%%%%%%%%%%%%%%%%%%%
%%%%%%%%%%%%%%%%%%%%%%%%%%%%%%%%%%%%%%%%%%%%%%%%%%%%%%%%%%%%%%%%%%%%%%%%%%%%%%%%
\vskip0.1in\hrule\vskip0.1in
\noindent
{\bf Vector/Matrix Operations: Definition of the Norm of a Vector}
\vskip0.1in\hrule\vskip0.1in
\noindent
There are many ways to compute the length of a vector. In this section we will
consider several ways to do this. However, we need to have a general definition
of what is meant by the length of a vector. To start, the definition of a norm
is given.
\begin{definition}
   Suppose that $V$ is a vector space and ${\bf u}$ and ${\bf v}$ are any two
   vectors in $V$. Also, assume $a$ is an arbitrary number/scalar. The norm of a
   vector is a function
   $$
     || * || : V \rightarrow \Re
   $$
   such that
   \begin{enumerate}
     \item $||v||=0$ if and only if ${\bf v} = {\bf 0}$,
     \item $||a{\bf v}|| = |a|\ ||{\bf v}||$, and
     \item $||{\bf u} + {\bf v} || \leq ||{\bf u}|| + ||{\bf v}||$.
   \end{enumerate}
\end{definition}
The Euclidean magnitude/length of a vector can be shown to satisfy this
definition of a norm. This means we can use the terms length, magnitude, and
norm interchangeably. The term norm is used in the name, {\bf l2norm}, chosen
for the method in the code above.

There are an infinite number of ways to compute norms on vector spaces. A
general definition for a norm on n-dimensional real space is the following.
$$
  ||v||_p = \left( v_1^p + v_2^p + \cdots + v_n^p \right)^{1/p}
              = \left( \sum_{i=1}^n\ |v_i|^p \right)^{1/p}
$$
for any positive integer $p$. The Euclidean norm is obtained when $p=2$. The
reality in computational mathematics is that there are three norms of interest
including the 2-norm. The other two norms are the 1-norm and the other is the
infinity-norm.  The definitions for these two norms are given below.
$$
  ||v||_1 = | v_1 | + | v_2 | + \cdots + | v_n | = \sum_{i=1}^n\ |v_i|
$$
and
$$
  ||v||_\infty = \max_{1\leq i\leq n} | v_i | 
$$
are the definitions we will use in our work. The gold standard for measuring
length is typically the 2-norm. However, in some problems the 1-norm and
infinity-norm are easier to compute in many problems.

The 1-norm and infinity-norm can easily be coded into methods as was done above
to the 2-norm. Using the 2-norm code, the other two methods just need a couple
of minor changes to implement the other two norms into their own methods.
\newpage
%%%%%%%%%%%%%%%%%%%%%%%%%%%%%%%%%%%%%%%%%%%%%%%%%%%%%%%%%%%%%%%%%%%%%%%%%%%%%%%%
%%%%%%%%%%%%%%%%%%%%%%%%%%%%%%%%%%%%%%%%%%%%%%%%%%%%%%%%%%%%%%%%%%%%%%%%%%%%%%%%
\vskip0.1in\hrule\vskip0.1in
\noindent
{\bf Vector/Matrix Operations: Errors in Vector Approximations}
\vskip0.1in\hrule\vskip0.1in
\noindent
Just as the errors in approximating roots of functions of a single variable,
having a way to compute the magnitude of the error in approximating one vector
${\bf v}$ by another vector ${\bf u}$. We can use the definitions of norms
earlier in this section to define a consistent formula for the vector
approximation error. If we use a generic definition of the norm of a vector, we
can define
$$
  \makebox{absolute error} = || {\bf v} - {\bf u} ||
$$
and
$$
  \makebox{relative error} = {{|| {\bf v} - {\bf u} || }\over{ || {\bf u} || }}
$$
These formulas should look familiar when compared to error measurement in root
finding problems.

A code that will implement the absolute error with the 2-norm might look like
\begin{verbatim}

     public double absl2err(double[] u, double[] v) {
       double sum = 0.0;
       //
       // extract the length of the vector
       // --------------------------------
       //
       int n = v.length;
       double diff = 0.0;
       for(int i=0; i<n; i++) {
         diff = u[i] - v[i];
         sum = sum + diff * diff;
       }
       //
       // return the norm of the difference
       // ---------------------------------
       //
       return Math.sqrt(sum);
       //
     }

\end{verbatim}
Of course, we could reuse code that we have already written as follows. Provided
that the methods have been created and test and inserted in some sort of archive
that is available for our use. A first simpler version is the following.
\begin{verbatim}

     public double absl2err(double[] u, double[] v) {
       double [] diff = null;
       int n = v.length;
       diff = vecsub(u, v);
       return l2norm(diff);
     }

\end{verbatim}
An even more concise version of the method might like the following.
\begin{verbatim}

     public double absl2err(double[] u, double[] v) {
       return l2norm(vecsub(u, v));
     }

\end{verbatim}

\end{document}
