\documentclass[10pt,fleqn]{article}
%\usepackage{graphicx}


\setlength{\topmargin}{-.75in}
\addtolength{\textheight}{2.00in}
\setlength{\oddsidemargin}{.00in}
\addtolength{\textwidth}{.75in}

\title{Math 4610 Lecture Notes \\
            \ \\
      Root Finding Problems for Real Values Function of One Variable
  \footnote{These notes are part of an Open Resource Educational project
            sponsored by Utah State University}}

\author{Joe Koebbe}

\nofiles

\pagestyle{empty}

\setlength{\parindent}{0in}

% new math commands


\setlength{\oddsidemargin}{-0.25in}
\setlength{\evensidemargin}{-0.25in}
\setlength{\textwidth}{6.75in}
\setlength{\headheight}{0.0in}
\setlength{\topmargin}{-0.25in}
\setlength{\textheight}{9.00in}

\makeindex

\usepackage{mathrsfs}

%\usepackage[pdftex]{graphicx}
\usepackage{epstopdf}

\newcounter{beans}

\newcommand{\ds}{\displaystyle}
\newcommand{\limit}[2]{\displaystyle\lim_{#1\to#2}}

\newcommand{\binomial}[2]{\ \left( \begin{array}{c}
                                  #1 \\
                                  #2
                                 \end{array}
                            \right) \
                         }
\newcommand{\ExampleRule}[2]
  {
  \noindent
  \rule{\linewidth}{1pt}
  \begin{example}
    #1
    \label{#2}
  \end{example}
  \rule{\linewidth}{1pt}
  \vskip0.125in
  }

\newcommand{\defbox}[1]
  {
   \ \\
   \noindent
   \setlength\fboxrule{1pt}
   \fbox{
        \begin{minipage}{6.5in}
          #1
        \end{minipage}
        }
   \ \\
  }
\newcommand{\verysmallworkbox}[1]
  {
   \ \\
   \noindent
   \setlength\fboxrule{1pt}
   \fbox{
        \begin{minipage}{6.5in}
           #1
           \ \\
           \vskip0.5in \ \\
           \ \\
        \end{minipage}
        }
   \ \\
  }
\newcommand{\smallworkbox}[1]
  {
   \ \\
   \noindent
   \setlength\fboxrule{1pt}
   \fbox{
        \begin{minipage}{6.5in}
           #1
           \ \\
           \vskip2.5in \ \\
           \ \\
        \end{minipage}
        }
   \ \\
  }
\newcommand{\halfworkbox}[1]
  {
   \ \\
   \noindent
   \setlength\fboxrule{1pt}
   \fbox{
        \begin{minipage}{6.5in}
           #1 \hfill
           \ \\
           \vskip3.25in \ \\
           \ \\
        \end{minipage}
        }
   \ \\
  }
\newcommand{\largeworkbox}[1]
  {
   \ \\
   \noindent
   \setlength\fboxrule{1pt}
   \fbox{
        \begin{minipage}{6.5in}
           #1
           \ \\
           \vskip7.5in \ \\
           \ \\
        \end{minipage}
        }
   \ \\
  }
\newcommand{\flexworkbox}[2]
  {
   \ \\
   \noindent
   \setlength\fboxrule{1pt}
   \fbox{
        \begin{minipage}{6.5in}
           #1
           \ \\

           \vskip#2 \ \\
           \ \\
        \end{minipage}
        }
   \ \\
  }


% symbols for sets of numbers

\newcommand{\natnumb}{$\cal N$}
\newcommand{\whonumb}{$\cal W$}
\newcommand{\intnumb}{$\cal Z$}
\newcommand{\ratnumb}{$\cal Q$}
\newcommand{\irrnumb}{$\cal I$}
\newcommand{\realnumb}{$\cal R$}
\newcommand{\cmplxnumb}{$\cal C$}

% misc. commands

\newcommand{\mma}{{\it Mathematica}}
\newcommand{\sech}{\mbox{ sech}}
 
\newtheorem{theorem}{Theorem}
\newtheorem{example}{Example}
\newtheorem{definition}{Definition}
\newtheorem{problem}{Problem}

\setcounter{secnumdepth}{2}
\setcounter{tocdepth}{4}


\begin{document}
\maketitle
\newpage
%%%%%%%%%%%%%%%%%%%%%%%%%%%%%%%%%%%%%%%%%%%%%%%%%%%%%%%%%%%%%%%%%%%%%%%%%%%%%%%%
%%%%%%%%%%%%%%%%%%%%%%%%%%%%%%%%%%%%%%%%%%%%%%%%%%%%%%%%%%%%%%%%%%%%%%%%%%%%%%%%
\vskip0.1in\hrule\vskip0.1in
\noindent
{\bf Root Finding Problem: Definition of a Root Finding Problem} 
\vskip0.1in\hrule\vskip0.1in
\noindent
There are many mathematical problems are cast in terms of finding a point in
some interval, $(a,b)$, where a function, $f$, is zero. Finding such locations
amounts to the solution of a root-finding problem. For example, in a standard
first semester calculus course, the process of finding extreme values of a
real-valued function, $g$, is presented. The problem in one variable can be
recast or transformed with some work into the problem of determining locations
where the derivative, $g'$, is zero. This is true since a necessary condition
for the existence of a local minimum or local maximum value of a differentiable
function at a point $x^*$ is that the derivative be zero. In this case, the
problem of determining the location of a minimum or maximum value of a function
is rewritten as finding the zeros of the derivative of the function. That is,
find all points, $x^*$, such that
$$
  g'(x^*)=0.
$$
The result is a root finding problem for the derivative of a function.

The following is a general definition of the root finding problem for a
real-valued function of a single real variable.
\begin{definition}
  {\bf The General Root Finding Problem:} Given a real-valued function, $f$, of
  a single real variable find a point or points, $x^*$, in the domain of the
  function such that
  $$
    f(x^*) = 0
  $$
  The value, $x^*$, is called a root or zero of the function $f$.
\end{definition}
Solution of the general root finding problem seems like it should be easy.
However, there are many sources of error and difficulties that are hidden within
the definition of the function.

There are all kinds of issues that arise in solving root finding problems. For
example, the function may have multiple roots that are close together. This is
an issue if, for example, the multiplicity of the root you are looking for is
in question. It might be the case that roots located close together may appear
as multiple roots due to roundoff error or machine precision issues. In this
case, it could be difficult to detect the difference in the locations of the
roots. In searching for a specific root, say the largest or smallest, we may
find other roots that are not of interest. To deal with all of the issues in
this problem, we will develop a number of algorithms that can be used to
overcome the problems that arise.

More often than not, we will need to locate roots that cannot be represented
exactly due to finite precision in number representation. For example, finding
the roots of
$$
  sin(x)=0
$$
is easy from an analytic point of view. This is a problem covered in all
trigonometry courses in high school and college. The zeros are $x_n=n\ \pi$
where $n$ is an arbitrary integer. If $n$ is not equal to zero, the root is an
irrational number and cannot be represented exactly in finite precision. So, we
must be prepared to settle for an approximation of the roots of a function. It
should be noted that an algebraic solution will be available only in cases where
$f(x)$ has a simple definition, say a linear or quadratic polynomial. Also, we
might be able to guarantee a solution exists, but there may be no analytic means
of finding a root or multiple roots for the given function.

As a simple example of proving the existence of roots, consider the function
\[
  p(x) = 1 + 2\ x + 3\ x^2 + 5\ x^3 + \pi\ x^4 + e^{1}\ x^5
\]
This is a polynomial of degree five. For any polynomial of odd degree, we know
from our algebra background there is at least one real root. Since $p(x)$ is a
polynomial of degree five, there must be at least one real root. However, based
on the coefficients, it will likely be the case that there is no analytic method
for computing a root for this problem.

One very complicated root finding problem involves one of the oldest unsolved
problems in all of mathematics. The problem is the Riemann conjecture or Riemann
hypothesis regarding the distribution of prime numbers in amongst all real
numbers. The Riemann-Zeta function is
$$
  \zeta(s) = \sum_{n=1}^\infty {1\over{n^s}}
$$
where $s$ represents an arbitrary complex number. This function is at the heart
of the Riemann-conjecture and the distribution of prime numbers. This innocent
looking formula is still not completely understood and the Riemann conjecture
has elluded all efforts at a solution for more than 100 years. It should be
noted that the distribution of primes is central in the development of data
encryption strategies in cyber-security applications.
\vskip0.1in\hrule\vskip0.1in
\newpage
%%%%%%%%%%%%%%%%%%%%%%%%%%%%%%%%%%%%%%%%%%%%%%%%%%%%%%%%%%%%%%%%%%%%%%%%%%%%%%%%
%%%%%%%%%%%%%%%%%%%%%%%%%%%%%%%%%%%%%%%%%%%%%%%%%%%%%%%%%%%%%%%%%%%%%%%%%%%%%%%%
\vskip0.1in\hrule\vskip0.1in
\noindent
{\bf Root Finding Problems: Using Fixed Point Iteration} 
\vskip0.1in\hrule\vskip0.1in
\noindent
As a first attempt at determining the location of a root for a function, we
might consider a modification of the root finding problem as follows. Given a
function, $f$, we can rewrite the root finding equation
$$
  f(x) = 0
$$
as
$$
  x = x - f(x) = g(x)
$$
The resulting equation is called a fixed point equation or fixed point problem
$$
  x = g(x)
$$
We will use the fixed point problem to define an algorithm for locating roots of
a function. So, suppose we have an initial guess at the solution of the fixed
point equation, $x_0$, that may or may not satisfy the equation. We can
substitute the value into the fixed point function to obtain
$$
  x_1 = g(x_0)
$$
If $x_0=x^*$ then the output will be the same as the input, $x_1=x^*$. If not,
the value can be used as another approximation of $x^*$. We can repeat this
process ad infinitum. A general formula for the iteration starts by providing an
initial guess, $x_0$, and then compute
$$
  x_{k+1} = g(x_k)
$$
for $k=0,1,2,\ldots$\ This iteration will produce an infinite sequence
$$
  \{ x_k \}_{k=0}^\infty = \{ x_0, x_1, x_2, \cdots \}
$$
of approximations to the solution of the fixed point problem. Since the fixed
point problem is equivalent to the root finding problem, we can treat the
sequence as approximations of the root finding problem.

Even though we can generate any number of approximations of the solution of the
fixed point problem in this way, there is no guarantee that any of these
approximations are close to the solution we desire. If a tolerance is specified
apriori there is no guarantee that the sequence will be close to anything. In
mathmatical terms, what we want is
$$
  \lim_{k\rightarrow\infty} x_k = x^*
$$
That is, we would really like the sequence to converge to a root. We will
return to this topic after writing a bit of code and presenting an example.
\vskip0.1in\hrule\vskip0.1in
\newpage
%%%%%%%%%%%%%%%%%%%%%%%%%%%%%%%%%%%%%%%%%%%%%%%%%%%%%%%%%%%%%%%%%%%%%%%%%%%%%%%%
%%%%%%%%%%%%%%%%%%%%%%%%%%%%%%%%%%%%%%%%%%%%%%%%%%%%%%%%%%%%%%%%%%%%%%%%%%%%%%%%
\vskip0.1in\hrule\vskip0.1in
\noindent
{\bf Root Finding Problems: Coding Fixed Point Iteration} 
\vskip0.1in\hrule\vskip0.1in
\noindent
One can easily write a routine or computer code that implements fixed point
iteration. The following code provides a template of how a reusable routine
might be written:
\vskip0.1in\hrule\vskip0.1in
\begin{verbatim}
     //
     // Author: Joe Koebbe
     //
     // Routine Name:         fproot
     // Programming Language: Java
     // Last Modified:        09/10/19
     //
     // Description/Purpose: The routine will generate a sequence of numbers
     // using fixed point iteration.
     //
     // Input:
     //
     // FunctionObject f - the function defined in the root finding problem
     // double x0 - the initial guess at the location of a fixed point
     // double tol - the error tolerance allowed in the approximation of the
     //              root finding problem
     // int maxit - the maximum number of iterations allowed in the fixed point
     //             iteration.
     //
     // Output:
     //
     // double x1 - the last number in the finite sequence that is an
     //             approximation in the root finding problem
     //
     public double fproot(FunctionObject f, double x0, double tol, int maxit) {
       //
       // initialize the error in the routine so that the iteration loop will be
       // executed at least one time
       // --------------------------
       //
       double error = 10.0 * tol;
       //
       // initialize a counter for the number of iterations
       // -------------------------------------------------
       //
       int iter = 0;
       //
       // loop over the fixed point iterations as long as the error is larger
       // than the tolerance and the number of iterations is less than the
       // maximum number allowed
       // ----------------------
       //
       while(error > tol && iter < maxit) {
         //
         // update the number of iterations performed
         // -----------------------------------------
         //
         iter++;
         //
         // compute the next approximation
         // ------------------------------
         //
         double x1 = x0 - f(x0);
         //
         // compute the error using the difference between the iterates in the
         // loop
         // ----
         //
         error = Math.abs(x1 - x0);
         //
         // reset the input value to be the new approximation
         // --------------------------------------------------
         //
         x0 = x1;
         //
       }
       //
       // return the last value computed
       // -------------------------------
       //
       return x1;
       //
     }

\end{verbatim}
\vskip0.1in\hrule\vskip0.1in
\noindent
There are a couple of features in the code that need to be explained.
\begin{list}{$\bullet$}{\usecounter{beans} \parsep=0pt \listparindent=0pt
\topsep=0pt \rightmargin=.35in \leftmargin=.35in \labelsep=5 pt
\itemsep=2pt}
  \item To make this work in the Java programming language, the method would
        need to be embedded in a class. That is, the code is not a standalone
        code.
  \item The first argument is an Java Object that needs to be created. The
        object is used to provide the function evaluation for any real input.
  \item The second argument is the initial guess at the solution of the problem.
  \item Since we know we are going to end up with at best an approximation of
        a root, the third argument in the function is an error tolerance that is
        acceptable to the calling routine.
  \item The final argument passed in limits the number of iterations allowed in
        the method. Note that if you are not careful, an infinite loop might be
        created due to the approximations used everywhere.
\end{list}
If we apply the code to any problem, we are assuming that the solution will pop
out the end. There is no guarantee that this is the case. It is important to
establish conditions that will guarantee the code will produce an approximate
solution of the fixed point problem and thus provide a root for the original
function, $f$.
\vskip0.1in\hrule\vskip0.1in
\newpage
%%%%%%%%%%%%%%%%%%%%%%%%%%%%%%%%%%%%%%%%%%%%%%%%%%%%%%%%%%%%%%%%%%%%%%%%%%%%%%%%
%%%%%%%%%%%%%%%%%%%%%%%%%%%%%%%%%%%%%%%%%%%%%%%%%%%%%%%%%%%%%%%%%%%%%%%%%%%%%%%%
\vskip0.1in\hrule\vskip0.1in
\noindent
{\bf Root Finding Problems: Analysis of Functional Iteration Using Taylor
 Series Expansion} 
\vskip0.1in\hrule\vskip0.1in
\noindent
The general iteration formula, given $x_0$, is the following.
$$
  x_{k+1} = g(x_k)
$$
for $k=0,1,2,\ldots$. We also know that for the fixed point problem, the
solution satisfies the equation
$$
  x^* = g(x^*)
$$
Subtracting the two equations gives
$$
  x_{k+1} - x^* = g(x_k) - g(x^*)
$$
The Taylor expansion of $g(x_k)$ about the solution $x^*$ is given by
$$
  g(x_k) = g(x*) + g'(x^*) ( x_k - x^* ) + {1\over 2} g''(x^*) ( x_k - x^* )^2
              + \ldots
$$
Substituting the expansion into the equation above and truncating the series
gives
$$
  x_{k+1} - x^* \approx g(x*) + g'(x^*) ( x_k - x^* ) - g(x^*)
                       = g'(x^*) ( x_k - x^* )
$$
Taking absolute values the last equation can be written as
$$
  | x_{k+1} - x^* | \leq | g'(x^*) | \cdot | x_k - x^* |
$$
One can read the previous expression as the difference (or error) in $x_{k+1}$
is less than the magnitude of the derivative of the fixed point iteration 
function, $g$, times the difference (or error) in the previous approximation,
$x_k$. Using
$$
  e_{k} = | x_k - x^* |
$$
allows use to relate the error at successive steps as
$$
  e_{k+1} \leq | g'(x^*) | \cdot | e_{k+1} |
$$
To get convergence to the fixed point (or root) we would like the error to be
reduced at each step. This requires the condition
$$
  | g'(x^*) | < 1
$$
For the general fixed point problem, this condition is required for convergence
to the fixed point, $x^*$, or solution of the root finding problem. Note that
this is a significant drawback of fixed point iteration as a means of solving
root finding problems.
\vskip0.1in\hrule\vskip0.1in
\newpage
%%%%%%%%%%%%%%%%%%%%%%%%%%%%%%%%%%%%%%%%%%%%%%%%%%%%%%%%%%%%%%%%%%%%%%%%%%%%%%%%
%%%%%%%%%%%%%%%%%%%%%%%%%%%%%%%%%%%%%%%%%%%%%%%%%%%%%%%%%%%%%%%%%%%%%%%%%%%%%%%%
\vskip0.1in\hrule\vskip0.1in
\noindent
{\bf Root Finding Problems: An Example Using Functional Iteration} 
\vskip0.1in\hrule\vskip0.1in
\noindent
Suppose that we are interested in computing the roots of
$$
  f(x) = e^x - \pi
$$
Analytically we can compute the solution by solving for $x$ in the equation
$$
  e^x - \pi = 0
$$
The value is $x=ln(\pi)\approx 1.144729886$. This is a very simple problem.
However, it is always a good idea to test general methods on simple problems
while developing algorithms and coding these up for use on real problems.

Let's apply functional iteration to this root finding problem. First, we will
need to create an associated function that defines a fixed point problem. One
possibility is to choose
$$
  g_1(x) = x - f(x) = x - ( e^x - \pi ) = x - e^x + \pi
$$
Let's check the condition for convergence by computing the derivative of $g$
near at the solution above.
$$
  g_1'(x) = 1 - e^x = 1 - \pi \approx -2.14159245 \rightarrow 
  | g_1'(x) | \approx 2.14159245
$$
The value is bigger than 1 which means the sequence of iterates is not going to
converge. So, the choice of $g(x)$ will not work.

As another option, consider a modification of the function. If
$$
  f(x) = e^x - \pi = 0
$$
then
$$
  f(x) = {1\over 5} ( e^x - \pi ) = 0
$$
which allows us to write
$$
  g_2(x) = x - {1\over 5} ( e^x - \pi )
$$
with derivative
$$
  g_2'(x) = 1 - {1\over 5} e^x 
$$
and near the solution
$$
  |g_2'(x)| = | 1 - {1\over 5} \pi | < 1.0
$$
So, we can expect better results in this case.
\vskip0.1in\hrule\vskip0.1in
\newpage
%%%%%%%%%%%%%%%%%%%%%%%%%%%%%%%%%%%%%%%%%%%%%%%%%%%%%%%%%%%%%%%%%%%%%%%%%%%%%%%%
%%%%%%%%%%%%%%%%%%%%%%%%%%%%%%%%%%%%%%%%%%%%%%%%%%%%%%%%%%%%%%%%%%%%%%%%%%%%%%%%
\vskip0.1in\hrule\vskip0.1in
\noindent
{\bf Root Finding Problems: Example Results Tabulated} 
\vskip0.1in\hrule\vskip0.1in
\noindent
For the two examples, the output for the two choices of the iteration function
$g_1(x)$ or $g_2(x)$.

\vskip0.1in\hrule\vskip0.1in
\begin{table}[h]
\caption{Results for Functional Iteration for Two Different Iteration Functions}
  \vskip0.1in
  \begin{center}
  \begin{tabular}{|c||c|c||c|c|}
    \hline
    Iteration No. & $g_2(x)=x-(e^x-\pi)$ & error
                              & $g_1(x)=x-(e^x-\pi)$ & error \\
    \hline
        01 &  1.08466220  &  8.46621990E-02  & 1.42331100  & 0.423310995 \\
    \hline
        02 &  1.12129271  &  3.66305113E-02  & 0.41406250  & 1.00924850 \\
    \hline
        03 &  1.13584745  &  1.45547390E-02  & 2.04270363  & 1.62864113 \\
    \hline
        04 &  1.14140379  &  5.55634499E-03  &-2.52713394  & 4.56983757 \\
    \hline
        05 &  1.14349020  &  2.08640099E-03  & 0.53457117  & 3.06170511 \\
    \hline
        06 &  1.14426863  &  7.78436661E-04  & 1.96944773  & 1.43487656 \\
    \hline
        07 &  1.14455843  &  2.89797783E-04  &-2.05567694  & 4.02512455 \\
    \hline
        08 &  1.14466619  &  1.07765198E-04  & 0.95790958  & 3.01358652 \\
    \hline
        09 &  1.14470625  &  4.00543213E-05  & 1.49325967  & 0.535350084 \\
    \hline
        10 &  1.14472115  &  1.49011612E-05  & 0.18326997  & 1.30998969 \\
    \hline
        11 &  1.14472663  &  5.48362732E-06  & 2.12372398  & 1.94045401 \\
    \hline
  \end{tabular}
  \end{center}
\end{table}
\vskip0.1in\hrule\vskip0.1in
So, two completely different results are obtained. One converges with a slight
modification to the first. The first function produces a sequence that does not
converge and the second produces the correct result up to machine precision.
That is, $x^*=1.14472663$ with absolute error $5.48362732E-06$. This is one of
the reasons why functional iteration is not used as much. The problem is that
there are infinitely many choices for the fixed point equation. Some will
provide convergence and others will not come close.
\vskip0.1in\hrule\vskip0.1in
\newpage
%%%%%%%%%%%%%%%%%%%%%%%%%%%%%%%%%%%%%%%%%%%%%%%%%%%%%%%%%%%%%%%%%%%%%%%%%%%%%%%%
%%%%%%%%%%%%%%%%%%%%%%%%%%%%%%%%%%%%%%%%%%%%%%%%%%%%%%%%%%%%%%%%%%%%%%%%%%%%%%%%
\vskip0.1in\hrule\vskip0.1in
\noindent
{\bf Root Finding Problems: Convergence of Functional Iteration}
\vskip0.1in\hrule\vskip0.1in
\noindent
If we end up using functional iteration, it will also pay to know how fast the
sequence converges. Fewer iterations means faster results with few computations.
The convergence of the sequence is determined by the same calculations as in
the convergence justification above.
$$
  | x_{k+1} - x^* | \leq | g'(x^*) | \cdot| x_k - x^* |
$$
For functional iteration the convergence rate is defined by
$$
  \makebox{rate of convergence} = | g'(x^*) | 
$$
The smaller the magnitude of the derivative, $|g'(x^*)|$, the faster the
convergence will be.

As an example, consider changing the parameter ${1\over 5}$ used to modify the
iteration function in the previous section. If the parameter is decreased, what
happens to the convergence? This is covered in the homework tasks.
\vskip0.1in\hrule\vskip0.1in
\newpage
%%%%%%%%%%%%%%%%%%%%%%%%%%%%%%%%%%%%%%%%%%%%%%%%%%%%%%%%%%%%%%%%%%%%%%%%%%%%%%%%
%%%%%%%%%%%%%%%%%%%%%%%%%%%%%%%%%%%%%%%%%%%%%%%%%%%%%%%%%%%%%%%%%%%%%%%%%%%%%%%%
\vskip0.1in\hrule\vskip0.1in
\noindent
{\bf Root Finding Problems: Continuous Functions and the Bisection Method}
\vskip0.1in\hrule\vskip0.1in
\noindent
On the positive side of things, the fixed point approach in the previous section
requires very little of the function in the root finding problem. The only
requirement is that $f$ is a function at every input value. It is usually very
easy to implement fixed point iteration for this type of problem. It may be
difficult if not impossible to come up with a fixed point problem that will
provide convergence to any fixed point. Due to slow convergence and issues
finding a fixed point equation that works, functional iteration is limited in
applicability in the real world.

So, we need to develop alternative algorithms for the root finding problem. In
this section, we will assume that the function, $f$, is continuous on a closed
and bounded interval $[a,b]$ where we expect to find a root. The main
mathematical tool used in this case is the Intermediate Value Theorem for
continuous functions.
\vskip0.1in\hrule\vskip0.1in
{\bf Theorem:} Suppose the function, $f$, is continuous on the closed and
boundaed interval $[a, b]$. If $M$ is any value between $f(a)$ and $f(b)$ then
there exists a value $c\in(a,b)$ such that $f(c)=M$,
\vskip0.1in\hrule\vskip0.1in
\noindent
Now, if $f(a)\geq 0\geq f(b)$ (or vice-versa) then there is at least one value,
$c$ in the interval $(a,b)$ such that $f(c)=0$. If we determine end-points of
an interval such that $f(a)<0$ {\bf and} $0<f(b)$ (or vice versa), we know there
is also a root of the function somewhere in the interval we have selected. There
is a simple condition that can be test to verify an interval contains an
interval. That is,
$$
   f(a)\cdot f(b) < 0
$$
This is enough to determine that the function crosses the horizontal axis at
at least one point in the interval.
\vskip0.1in\hrule\vskip0.1in
\newpage
%%%%%%%%%%%%%%%%%%%%%%%%%%%%%%%%%%%%%%%%%%%%%%%%%%%%%%%%%%%%%%%%%%%%%%%%%%%%%%%%
%%%%%%%%%%%%%%%%%%%%%%%%%%%%%%%%%%%%%%%%%%%%%%%%%%%%%%%%%%%%%%%%%%%%%%%%%%%%%%%%
\vskip0.1in\hrule\vskip0.1in
\noindent
{\bf Root Finding Problems: Bisection and Convergence}
\vskip0.1in\hrule\vskip0.1in
\noindent
Once we have determined an interval $[a,b]$ such that $f(a)\ f(b)<0$ we can
start work to determine the location of a root in the initial interval. We
proceed by bisecting the original interval $[a, b]$ into two equal subintervals
$$
  [a, b] = [a, c] \cup [c, b]
$$
where
$$
  c = {{a+b}\over 2}
$$
Since there is at least one root on $[a, b]$ there are three possibilities that
can occur in the bisection. These are:
\begin{list}{$\bullet$}{\usecounter{beans} \parsep=0pt \listparindent=0pt
\topsep=0pt \rightmargin=.35in \leftmargin=.35in \labelsep=5 pt
\itemsep=2pt}
  \item $f(c) = 0$,
  \item $f(a)\cdot f(c)<0$ which implies there is a root in $[a,c]$, or
  \item $f(c)\cdot f(b)<0$ which implies there is a root in $[c,b]$.
\end{list}
If the first condition is true, we have the root, $x^*=c$ and we are done
searching. In the second case, we can redefine the search interval to $[a,c]$
and in the third case, the search interval will be redefined to be $[c,b]$. Once
we have redefined the search interval, we repeat the bisection on this new
search interval. The bisection will reduce the size of the search interval by
a factor of two. We just need to translate this idea into a computer code in
some language.
\vskip0.1in\hrule\vskip0.1in
\newpage
%%%%%%%%%%%%%%%%%%%%%%%%%%%%%%%%%%%%%%%%%%%%%%%%%%%%%%%%%%%%%%%%%%%%%%%%%%%%%%%%
%%%%%%%%%%%%%%%%%%%%%%%%%%%%%%%%%%%%%%%%%%%%%%%%%%%%%%%%%%%%%%%%%%%%%%%%%%%%%%%%
\vskip0.1in\hrule\vskip0.1in
\noindent
{\bf Root Finding Problems: A Simple Bisection Code in C}
\vskip0.1in\hrule\vskip0.1in
\noindent
The following routine, written in something like C implements the Bisection
Method.
\vskip0.1in\hrule\vskip0.1in
\begin{verbatim}

     double bisectionMethod(typedef'd f, double a, double b, double tol,
                            int maxiter)
     {
       //
       // set up some parameters and local variables to do the work
       // ---------------------------------------------------------
       //
       double c;
       double error;
       int iter;
       //
       // check the endpoints - if either is 0, we already have a root
       // ------------------------------------------------------------
       //
       if(f(a)==0) return a;
       if(f(b)==0) return b;
       //
       // check for a root in the interval
       // --------------------------------
       //
       if(f(a)*f(b) >= 0.0) throw an error or print a message
       //
       // set the error and iteration counter
       // -----------------------------------
       //
       error = 10.0 * tol;
       iter = 0;
       //
       // use a while loop to go until the tolerance is met or the maximum 
       // number of iterations has been exceeded
       // --------------------------------------
       //
       while(error > tol && iter < maxiter) {
         //
         // update the iteration counter and compute the midpoint of the current
         // interval
         // --------
         //
         iter++;
         c = 0.5 * ( a + b );
         //
         // compute the sign change value
         // -----------------------------
         //
         double val = f(a) * f(c);
         //
         // reassign the end point based on the location of the root
         // --------------------------------------------------------
         //
         if(val<0.0) {
           b = c;
         } else {
           a = c;
         }
         //
         // compute the error in the approximation - this assumes a<b
         // ---------------------------------------------------------
         //
         error = b - a
         //
       }
       //
       // return the midpoint as it is more accurate
       // ------------------------------------------
       //
       return c;
       //
     }

\end{verbatim}
\vskip0.1in\hrule\vskip0.1in
\noindent
The first argument in the C method needs to be changed to a pointer to a
function as an input to the method. This is left up to the reader to do.
It should be noted that once an interval has been determined on which the
function value changes sign, the Bisection Method will continue until a root is
found, at least up to machine precision. We can take advantage of this property
to redesign the algorithm to take a specific number of iterations instead of
checking the error.
\vskip0.1in\hrule\vskip0.1in
\newpage
%%%%%%%%%%%%%%%%%%%%%%%%%%%%%%%%%%%%%%%%%%%%%%%%%%%%%%%%%%%%%%%%%%%%%%%%%%%%%%%%
%%%%%%%%%%%%%%%%%%%%%%%%%%%%%%%%%%%%%%%%%%%%%%%%%%%%%%%%%%%%%%%%%%%%%%%%%%%%%%%%
\vskip0.1in\hrule\vskip0.1in
\noindent
{\bf Root Finding Problems: The Bisection Method and Error Reduction}
\vskip0.1in\hrule\vskip0.1in
\noindent
The fact the the interval size is being reduced in each iteration of bisection
can be used as follows. The length of the original interval can be computed and
used to bound the error in any approximation of a root. That is,
$$
  | x - x^* | \leq | b - a |
$$
A sequence of intervals is created by the Bisection method that contains a root.
We can use subscripts to define the intervals as the bisection proceeds. If we
use $[a_i, b_i]$, for $i=0,1,\ldots$ where each new interval is selected after
the previous interval is bisected. Note that if we are assuming
$[a_0, b_0]=[a,b]$ in this argument. So, we can write the following set of
inequalities
$$
  | x - x^* | < b_k - a_k
              < {1\over 2} ( b_{k-1} - a_{k-1} )
              < \cdots
              < {1\over{2^k}} ( b_0 - a_0 ) = 2^{-k} ( b - a )  
$$ 
This means that once the interval $[a, b]$ has been determined, the reduction in
the error between iterations is computable.

Suppose that we specify an error tolerance that is acceptable, say $10^{-d}$
where $d$ is the number of digits of accuracy. Then we can define the number of
iterations to reduce the error to the desired tolerance as follows.
$$
  2^{-k} ( b - a ) < 10^{-d}
$$
Using a bit of algebra
$$
  2^{-k} < {{10^{-d}}\over{(b-a)}}
   \rightarrow -k < log_2\left( {{10^{-d}}\over{(b-a)}}\right)
$$
or flipping the inequality using a negative multiplier
$$
  - log_2\left( {{10^{-d}}\over{(b-a)}}\right) < k 
$$
This gives us the total number of iterations needed to reduce the error to the
desired tolerance. So, we can rewrite the code to take advantage of this
calculation.
\vskip0.1in\hrule\vskip0.1in
\newpage
%%%%%%%%%%%%%%%%%%%%%%%%%%%%%%%%%%%%%%%%%%%%%%%%%%%%%%%%%%%%%%%%%%%%%%%%%%%%%%%%
%%%%%%%%%%%%%%%%%%%%%%%%%%%%%%%%%%%%%%%%%%%%%%%%%%%%%%%%%%%%%%%%%%%%%%%%%%%%%%%%
\vskip0.1in\hrule\vskip0.1in
\noindent
{\bf Root Finding Problems: An Alternative Bisection Method Code}
\vskip0.1in\hrule\vskip0.1in
\noindent
The alternative C code to implement the alternate Bisection method is the
following.
%%%%%%%%%%%%%%%%%%%%%%%%%%%%%%%%%%%%%%%%%%%%%%%%%%%%%%%%%%%%%%%%%%%%%%%%%%%%%%%%
%%%%%%%%%%%%%%%%%%%%%%%%%%%%%%%%%%%%%%%%%%%%%%%%%%%%%%%%%%%%%%%%%%%%%%%%%%%%%%%%
\vskip0.1in\hrule\vskip0.1in
\begin{verbatim}

     double bisectionMethod(typedef'd f, double a, double b, double tol) {
       //
       // set up some parameters and local variables to do the work
       // ---------------------------------------------------------
       //
       double c;
       double error;
       //
       // check the endpoints - if either is 0, we already have a root
       // ------------------------------------------------------------
       //
       if(f(a)==0) return a;
       if(f(b)==0) return b;
       //
       // check for a root in the interval
       // --------------------------------
       //
       if(f(a)*f(b) >= 0.0) throw an error or print a message
       //
       // compute the number iterations needed to meet the tolerance given
       // ----------------------------------------------------------------
       //
       maxiter = - 2.0 * log2( tol / ( b - a ) );
       //
       // compute the iterations
       // ----------------------
       for(int i=0; i<maxiter; i++) {
         //
         // compute the midpoint of the current interval
         // --------------------------------------------
         //
         c = 0.5 * ( a + b );
         //
         // compute the sign change value
         // -----------------------------
         //
         double val = f(a) * f(c);
         //
         // reassign the end point based on the location of the root
         // --------------------------------------------------------
         //
         if(val<0.0) {
           b = c;
         } else {
           a = c;
         }
         //
       }
       //
       // return the midpoint as it is more accurate
       // ------------------------------------------
       //
       return c;
       //
     }

\end{verbatim}
\vskip0.1in\hrule\vskip0.1in
\noindent
Note that the output value will be an approximation of a root in the original
interval, $[a,b]$ that satisfies the desired tolerance.
\vskip0.1in\hrule\vskip0.1in
\newpage
%%%%%%%%%%%%%%%%%%%%%%%%%%%%%%%%%%%%%%%%%%%%%%%%%%%%%%%%%%%%%%%%%%%%%%%%%%%%%%%%
%%%%%%%%%%%%%%%%%%%%%%%%%%%%%%%%%%%%%%%%%%%%%%%%%%%%%%%%%%%%%%%%%%%%%%%%%%%%%%%%
\vskip0.1in\hrule\vskip0.1in
\noindent
{\bf Root Finding Problems: Bisection Method Examples}
\vskip0.1in\hrule\vskip0.1in
\noindent
It is always a good idea to test the code you write. Using the example from our
tests of functional iteration we can detemine whether or not the Bisection 
Method works and how this compares with the fixed point approach.
%%%%%%%%%%%%%%%%%%%%%%%%%%%%%%%%%%%%%%%%%%%%%%%%%%%%%%%%%%%%%%%%%%%%%%%%%%%%%%%%
%%%%%%%%%%%%%%%%%%%%%%%%%%%%%%%%%%%%%%%%%%%%%%%%%%%%%%%%%%%%%%%%%%%%%%%%%%%%%%%%
\end{document}
