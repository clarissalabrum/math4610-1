\documentclass[10pt,fleqn]{article}
%\usepackage{graphicx}


\setlength{\topmargin}{-.75in}
\addtolength{\textheight}{2.00in}
\setlength{\oddsidemargin}{.00in}
\addtolength{\textwidth}{.75in}

\title{Math 4610 Lecture Notes \\
            \ \\
       A Brief Introduction to Github Usage
  \footnote{These notes are part of an Open Resource Educational project
            sponsored by Utah State University}}

\author{Joe Koebbe}

\nofiles

\pagestyle{empty}

\setlength{\parindent}{0in}

% new math commands


\setlength{\oddsidemargin}{-0.25in}
\setlength{\evensidemargin}{-0.25in}
\setlength{\textwidth}{6.75in}
\setlength{\headheight}{0.0in}
\setlength{\topmargin}{-0.25in}
\setlength{\textheight}{9.00in}

\makeindex

\usepackage{mathrsfs}

%\usepackage[pdftex]{graphicx}
\usepackage{epstopdf}

\newcounter{beans}

\newcommand{\ds}{\displaystyle}
\newcommand{\limit}[2]{\displaystyle\lim_{#1\to#2}}

\newcommand{\binomial}[2]{\ \left( \begin{array}{c}
                                  #1 \\
                                  #2
                                 \end{array}
                            \right) \
                         }
\newcommand{\ExampleRule}[2]
  {
  \noindent
  \rule{\linewidth}{1pt}
  \begin{example}
    #1
    \label{#2}
  \end{example}
  \rule{\linewidth}{1pt}
  \vskip0.125in
  }

\newcommand{\defbox}[1]
  {
   \ \\
   \noindent
   \setlength\fboxrule{1pt}
   \fbox{
        \begin{minipage}{6.5in}
          #1
        \end{minipage}
        }
   \ \\
  }
\newcommand{\verysmallworkbox}[1]
  {
   \ \\
   \noindent
   \setlength\fboxrule{1pt}
   \fbox{
        \begin{minipage}{6.5in}
           #1
           \ \\
           \vskip0.5in \ \\
           \ \\
        \end{minipage}
        }
   \ \\
  }
\newcommand{\smallworkbox}[1]
  {
   \ \\
   \noindent
   \setlength\fboxrule{1pt}
   \fbox{
        \begin{minipage}{6.5in}
           #1
           \ \\
           \vskip2.5in \ \\
           \ \\
        \end{minipage}
        }
   \ \\
  }
\newcommand{\halfworkbox}[1]
  {
   \ \\
   \noindent
   \setlength\fboxrule{1pt}
   \fbox{
        \begin{minipage}{6.5in}
           #1 \hfill
           \ \\
           \vskip3.25in \ \\
           \ \\
        \end{minipage}
        }
   \ \\
  }
\newcommand{\largeworkbox}[1]
  {
   \ \\
   \noindent
   \setlength\fboxrule{1pt}
   \fbox{
        \begin{minipage}{6.5in}
           #1
           \ \\
           \vskip7.5in \ \\
           \ \\
        \end{minipage}
        }
   \ \\
  }
\newcommand{\flexworkbox}[2]
  {
   \ \\
   \noindent
   \setlength\fboxrule{1pt}
   \fbox{
        \begin{minipage}{6.5in}
           #1
           \ \\

           \vskip#2 \ \\
           \ \\
        \end{minipage}
        }
   \ \\
  }


% symbols for sets of numbers

\newcommand{\natnumb}{$\cal N$}
\newcommand{\whonumb}{$\cal W$}
\newcommand{\intnumb}{$\cal Z$}
\newcommand{\ratnumb}{$\cal Q$}
\newcommand{\irrnumb}{$\cal I$}
\newcommand{\realnumb}{$\cal R$}
\newcommand{\cmplxnumb}{$\cal C$}

% misc. commands

\newcommand{\mma}{{\it Mathematica}}
\newcommand{\sech}{\mbox{ sech}}
 
\newtheorem{theorem}{Theorem}
\newtheorem{example}{Example}
\newtheorem{definition}{Definition}
\newtheorem{problem}{Problem}

\setcounter{secnumdepth}{2}
\setcounter{tocdepth}{4}


\begin{document}
\maketitle
\newpage
%%%%%%%%%%%%%%%%%%%%%%%%%%%%%%%%%%%%%%%%%%%%%%%%%%%%%%%%%%%%%%%%%%%%%%%%%%%%%%%%
%%%%%%%%%%%%%%%%%%%%%%%%%%%%%%%%%%%%%%%%%%%%%%%%%%%%%%%%%%%%%%%%%%%%%%%%%%%%%%%%

%%%%%%%%%%%%%%%%%%%%%%%%%%%%%%%%%%%%%%%%%%%%%%%%%%%%%%%%%%%%%%%%%%%%%%%%%%%%%%%%
%%%%%%%%%%%%%%%%%%%%%%%%%%%%%%%%%%%%%%%%%%%%%%%%%%%%%%%%%%%%%%%%%%%%%%%%%%%%%%%%
\vskip0.1in\hrule\vskip0.1in
\noindent
{\bf Github Primer for Math 4610 at USU: Get an Account} 
\vskip0.1in\hrule\vskip0.1in
\noindent
To create an account on GitHub, go to the Github site
\begin{verbatim}

    https://github.com

\end{verbatim}
This site will display a place to create an account or sign in to an existin
account.
\vskip0.1in\hrule\vskip0.1in
\vfill
\begin{figure}[h]
\centering
\includegraphics{../images/github_01.png}
\vskip0.1in
\caption{{Screenshot} taken using {\bf Snip \& Sketch}. This is an app on
         my Windows 10 box}
\end{figure}
\eject
%%%%%%%%%%%%%%%%%%%%%%%%%%%%%%%%%%%%%%%%%%%%%%%%%%%%%%%%%%%%%%%%%%%%%%%%%%%%%%%%
%%%%%%%%%%%%%%%%%%%%%%%%%%%%%%%%%%%%%%%%%%%%%%%%%%%%%%%%%%%%%%%%%%%%%%%%%%%%%%%%
\vskip0.1in\hrule\vskip0.1in
\noindent
{{\bf GitHub Primer for Math 4610 at USU:} Setting up a Repository} 
\vskip0.1in\hrule\vskip0.1in
\noindent
Once you log in, you will need to build a repository for use in the class and
to turn in homework and completed tasks and projects. For the course, you will
create a repository named the following:
\begin{verbatim}

    math4610

\end{verbatim}
Use only the characters above and using the following rules:
\begin{enumerate}
  \item Use only lower case characters - github is case senesitive.
  \item Do not put any blanks in the name of the repository.
\end{enumerate}
Note that the instructor will use only this repository name in looking for your
work.
\vskip0.1in\hrule\vskip0.1in
\vfill
\begin{figure}[h]
\centering
\includegraphics{../images/cygwin_02.png}
\caption{{Screenshot} taken using {\bf Snip \& Sketch}. This is an app on
         my Windows 10 box}
\end{figure}
\eject
%%%%%%%%%%%%%%%%%%%%%%%%%%%%%%%%%%%%%%%%%%%%%%%%%%%%%%%%%%%%%%%%%%%%%%%%%%%%%%%%
%%%%%%%%%%%%%%%%%%%%%%%%%%%%%%%%%%%%%%%%%%%%%%%%%%%%%%%%%%%%%%%%%%%%%%%%%%%%%%%%
\vskip0.1in\hrule\vskip0.1in
\noindent
{{\bf Cygwin Primer for Math 4610 at USU:} List the Contents of the Home
    Directory} 
\vskip0.1in\hrule\vskip0.1in
\noindent
If you have an account on GitHub, you will already know a lot about these
things. However, when you are logged in you will see the main screen with any
repositories you may already have created. We will go through the steps to
build and name repositories in the next few pages.
\begin{verbatim}

   % command [options] [input parameters]

\end{verbatim}
\vskip0.1in\hrule\vskip0.1in
\vfill
\begin{figure}[h]
\centering
\includegraphics{../images/cygwin_03.png}
\caption{{Screenshot} taken using {\bf Snip \& Sketch}. This is an app on
         my Windows 10 box}
\end{figure}
\eject
%%%%%%%%%%%%%%%%%%%%%%%%%%%%%%%%%%%%%%%%%%%%%%%%%%%%%%%%%%%%%%%%%%%%%%%%%%%%%%%%
%%%%%%%%%%%%%%%%%%%%%%%%%%%%%%%%%%%%%%%%%%%%%%%%%%%%%%%%%%%%%%%%%%%%%%%%%%%%%%%%
\vskip0.1in\hrule\vskip0.1in
\noindent
{{\bf Cygwin Primer for Math 4610 at USU:} Directory Commands} 
\vskip0.1in\hrule\vskip0.1in
\noindent
You will need to create, move, remove, and other things to directories to keep
work organized. The {\bf mkdir} command allows a persion to create a new
directory in the current working directory. This is the same thing tha Windows
Explorer allows you to do with a popup menu. There will be many places where
a directory structure will be required. You can remove a directory with the
{\bf rmdir} command. The {\bf cd} command can be used to navigate through a
directory structure. Finally, on this screen capture, the {\bf pwd} command is
used to determine the current working directory. This can be used to figure out
where you are in a directory structure.
\begin{verbatim}

    % pwd                 current working directory
    % cd                  change working directory
    % mkdir               make a new directory
    % rmdir               remove an existing directory

\end{verbatim}
\vskip0.1in\hrule\vskip0.1in
\vfill
\begin{figure}[h]
\centering
\includegraphics{../images/cygwin_03.png}
\caption{{Screenshot} taken using {\bf Snip \& Sketch}. This is an app on
         my Windows 10 box}
\end{figure}
\eject
%%%%%%%%%%%%%%%%%%%%%%%%%%%%%%%%%%%%%%%%%%%%%%%%%%%%%%%%%%%%%%%%%%%%%%%%%%%%%%%%
%%%%%%%%%%%%%%%%%%%%%%%%%%%%%%%%%%%%%%%%%%%%%%%%%%%%%%%%%%%%%%%%%%%%%%%%%%%%%%%%
\vskip0.1in\hrule\vskip0.1in
\noindent
{{\bf Cygwin Primer for Math 4610 at USU:} Which Command} 
\vskip0.1in\hrule\vskip0.1in
\noindent
You will want to know what is available for doing work within Cygwin or any
other platform. The which command will let you know if apps or other executables
are available on your version of Cygwin. In particular, it is important to know
if certain compilers (e.g, javac, gcc, f77) are available. A significant number
of tasks you will be asked to complete will require the use of a compiler and
Cygwin has a number of (good) standard compilers for C, C++, and fortran. The
syntax for the command is the following.
\begin{verbatim}

    % which command

\end{verbatim}
\vskip0.1in\hrule\vskip0.1in
\vfill
\begin{figure}[h]
\centering
\includegraphics{../images/cygwin_04.png}
\caption{{Screenshot} taken using {\bf Snip \& Sketch}. This is an app on
         my Windows 10 box}
\end{figure}
\eject
%%%%%%%%%%%%%%%%%%%%%%%%%%%%%%%%%%%%%%%%%%%%%%%%%%%%%%%%%%%%%%%%%%%%%%%%%%%%%%%%
%%%%%%%%%%%%%%%%%%%%%%%%%%%%%%%%%%%%%%%%%%%%%%%%%%%%%%%%%%%%%%%%%%%%%%%%%%%%%%%%
\vskip0.1in\hrule\vskip0.1in
\noindent
{{\bf Cygwin Primer for Math 4610 at USU:} A Simple Editing Program} 
\vskip0.1in\hrule\vskip0.1in
\noindent
You will need an editor to create text files. There are a number of editors that
can be downloaded and used in any Cygwin installation. The standard editor that
is always available for linux and unix boxes is \lq vi\rq\. This editor is a bit
rudimentary, but works. Another editor which will be used in by the instructor
in the course is \lq vim\rq. The syntax for starting the editor in a window is
the following.
\begin{verbatim}

    % vim filename

\end{verbatim}
There are a lot of escape sequences it insert text, write a file and so on. If
you are new to vim, you will need to learn at least a few of these editing
commands.
\vskip0.1in\hrule\vskip0.1in
\vfill
\begin{figure}[h]
\centering
\includegraphics{../images/cygwin_05.png}
\caption{{Screenshot} taken using {\bf Snip \& Sketch}. This is an app on
         my Windows 10 box}
\end{figure}
\eject
%%%%%%%%%%%%%%%%%%%%%%%%%%%%%%%%%%%%%%%%%%%%%%%%%%%%%%%%%%%%%%%%%%%%%%%%%%%%%%%%
%%%%%%%%%%%%%%%%%%%%%%%%%%%%%%%%%%%%%%%%%%%%%%%%%%%%%%%%%%%%%%%%%%%%%%%%%%%%%%%%
\vskip0.1in\hrule\vskip0.1in
\noindent
{{\bf Cygwin Primer for Math 4610 at USU:} First View of the vim Editor} 
\vskip0.1in\hrule\vskip0.1in
\noindent
Below is what the terminal will turn into when you start up the editor on a new
file. To get out of the editor, you can use any of the following commands inside
the editor.
\begin{verbatim}

    :x                  write and exit the editor - saves changes in the file
    :q                  exit the editor if no changes have been made
    :x!                 force a write and exit the editor - saves the file
    :q!                 force an exit of the editor - no changes are saved

\end{verbatim}
Note that there are a few other commands that can be used to save changes. For
example
\begin{verbatim}

    :w                  write and stay in the editor
    :w!                 force a write and stay in the editor

\end{verbatim}
\vskip0.1in\hrule\vskip0.1in
\vfill
\begin{figure}[h]
\centering
\includegraphics{../images/cygwin_06.png}
\caption{{Screenshot} taken using {\bf Snip \& Sketch}. This is an app on
         my Windows 10 box}
\end{figure}
\eject
%%%%%%%%%%%%%%%%%%%%%%%%%%%%%%%%%%%%%%%%%%%%%%%%%%%%%%%%%%%%%%%%%%%%%%%%%%%%%%%%
%%%%%%%%%%%%%%%%%%%%%%%%%%%%%%%%%%%%%%%%%%%%%%%%%%%%%%%%%%%%%%%%%%%%%%%%%%%%%%%%
\vskip0.1in\hrule\vskip0.1in
\noindent
{{\bf Cygwin Primer for Math 4610 at USU:} An Example of a Text File/Program} 
\vskip0.1in\hrule\vskip0.1in
\noindent
The following screenshot shows a few lines that have been typed into vim that
deinfes a standard hello world example for C. To insert/append characters in the
text file, you can use the following commands to do this. Note that the commands
below do not show up on the screen and the chnages are made where the cursor is
currently located.
\begin{verbatim}

    a                     append text at this point in the file
    o                     open a line after the current line
    O                     open a line before the current line

\end{verbatim}
To end adding or inserting text, use the escape character. Again, the commands
will not show up on screen. Learning everything about vi or vim is a time
consuming process. It is one of those things that you figure out as you go.
\vskip0.1in\hrule\vskip0.1in
\vfill
\begin{figure}[h]
\centering
\includegraphics{../images/cygwin_07.png}
\caption{{Screenshot} taken using {\bf Snip \& Sketch}. This is an app on
         my Windows 10 box}
\end{figure}
\eject
%%%%%%%%%%%%%%%%%%%%%%%%%%%%%%%%%%%%%%%%%%%%%%%%%%%%%%%%%%%%%%%%%%%%%%%%%%%%%%%%
%%%%%%%%%%%%%%%%%%%%%%%%%%%%%%%%%%%%%%%%%%%%%%%%%%%%%%%%%%%%%%%%%%%%%%%%%%%%%%%%
\vskip0.1in\hrule\vskip0.1in
\noindent
{{\bf Cygwin Primer for Math 4610 at USU:} Compiling a Program} 
\vskip0.1in\hrule\vskip0.1in
\noindent
Compiling a program is relatively easy at this point in time. To compile the
program on the previous page you should type in
\begin{verbatim}

    % gcc hello.c

\end{verbatim}
The result is an exeutable as seen below. If you want to name the executable
something besides \lq a\rq\ then type the following.
\begin{verbatim}

    % gcc -o hello hello.c

\end{verbatim}
\vskip0.1in\hrule\vskip0.1in
\vfill
\begin{figure}[h]
\centering
\includegraphics{../images/cygwin_08.png}
\caption{{Screenshot} taken using {\bf Snip \& Sketch}. This is an app on
         my Windows 10 box}
\end{figure}
\eject
%%%%%%%%%%%%%%%%%%%%%%%%%%%%%%%%%%%%%%%%%%%%%%%%%%%%%%%%%%%%%%%%%%%%%%%%%%%%%%%%
%%%%%%%%%%%%%%%%%%%%%%%%%%%%%%%%%%%%%%%%%%%%%%%%%%%%%%%%%%%%%%%%%%%%%%%%%%%%%%%%
\vskip0.1in\hrule\vskip0.1in
\noindent
{{\bf Cygwin Primer for Math 4610 at USU:} Keeping Track of Working Code}
\vskip0.1in\hrule\vskip0.1in
\noindent
It is a good idea to organize your work within assignments and projects. There
is a standard set of folders/directories in linux and unix that most have
adopted. Your instructor follows this idea and usually creates a list of
folders including /src, /data, /bin, and /doc. When computer literate folks see
these folders, they know what is stored in the folders. As an example,
\begin{verbatim}

    % mkdir src
    % mkdir bin

\end{verbatim}
can be used and then the executable the text file can be put into /src and the
binary can be copied into /bin. 
\vskip0.1in\hrule\vskip0.1in
\vfill
\begin{figure}[h]
\centering
\includegraphics{../images/cygwin_09.png}
\caption{{Screenshot} taken using {\bf Snip \& Sketch}. This is an app on
         my Windows 10 box}
\end{figure}
\eject
%%%%%%%%%%%%%%%%%%%%%%%%%%%%%%%%%%%%%%%%%%%%%%%%%%%%%%%%%%%%%%%%%%%%%%%%%%%%%%%%
%%%%%%%%%%%%%%%%%%%%%%%%%%%%%%%%%%%%%%%%%%%%%%%%%%%%%%%%%%%%%%%%%%%%%%%%%%%%%%%%
\end{document}
