\documentclass[10pt,fleqn]{article}
%\usepackage{graphicx}


\setlength{\topmargin}{-.75in}
\addtolength{\textheight}{2.00in}
\setlength{\oddsidemargin}{.00in}
\addtolength{\textwidth}{.75in}

\title{Math 4610 Lecture Notes \\
            \ \\
      Measurement of Error
  \footnote{These notes are part of an Open Resource Educational project
            sponsored by Utah State University}}

\author{Joe Koebbe}

\nofiles

\pagestyle{empty}

\setlength{\parindent}{0in}

% new math commands


\setlength{\oddsidemargin}{-0.25in}
\setlength{\evensidemargin}{-0.25in}
\setlength{\textwidth}{6.75in}
\setlength{\headheight}{0.0in}
\setlength{\topmargin}{-0.25in}
\setlength{\textheight}{9.00in}

\makeindex

\usepackage{mathrsfs}

%\usepackage[pdftex]{graphicx}
\usepackage{epstopdf}

\newcounter{beans}

\newcommand{\ds}{\displaystyle}
\newcommand{\limit}[2]{\displaystyle\lim_{#1\to#2}}

\newcommand{\binomial}[2]{\ \left( \begin{array}{c}
                                  #1 \\
                                  #2
                                 \end{array}
                            \right) \
                         }
\newcommand{\ExampleRule}[2]
  {
  \noindent
  \rule{\linewidth}{1pt}
  \begin{example}
    #1
    \label{#2}
  \end{example}
  \rule{\linewidth}{1pt}
  \vskip0.125in
  }

\newcommand{\defbox}[1]
  {
   \ \\
   \noindent
   \setlength\fboxrule{1pt}
   \fbox{
        \begin{minipage}{6.5in}
          #1
        \end{minipage}
        }
   \ \\
  }
\newcommand{\verysmallworkbox}[1]
  {
   \ \\
   \noindent
   \setlength\fboxrule{1pt}
   \fbox{
        \begin{minipage}{6.5in}
           #1
           \ \\
           \vskip0.5in \ \\
           \ \\
        \end{minipage}
        }
   \ \\
  }
\newcommand{\smallworkbox}[1]
  {
   \ \\
   \noindent
   \setlength\fboxrule{1pt}
   \fbox{
        \begin{minipage}{6.5in}
           #1
           \ \\
           \vskip2.5in \ \\
           \ \\
        \end{minipage}
        }
   \ \\
  }
\newcommand{\halfworkbox}[1]
  {
   \ \\
   \noindent
   \setlength\fboxrule{1pt}
   \fbox{
        \begin{minipage}{6.5in}
           #1 \hfill
           \ \\
           \vskip3.25in \ \\
           \ \\
        \end{minipage}
        }
   \ \\
  }
\newcommand{\largeworkbox}[1]
  {
   \ \\
   \noindent
   \setlength\fboxrule{1pt}
   \fbox{
        \begin{minipage}{6.5in}
           #1
           \ \\
           \vskip7.5in \ \\
           \ \\
        \end{minipage}
        }
   \ \\
  }
\newcommand{\flexworkbox}[2]
  {
   \ \\
   \noindent
   \setlength\fboxrule{1pt}
   \fbox{
        \begin{minipage}{6.5in}
           #1
           \ \\

           \vskip#2 \ \\
           \ \\
        \end{minipage}
        }
   \ \\
  }


% symbols for sets of numbers

\newcommand{\natnumb}{$\cal N$}
\newcommand{\whonumb}{$\cal W$}
\newcommand{\intnumb}{$\cal Z$}
\newcommand{\ratnumb}{$\cal Q$}
\newcommand{\irrnumb}{$\cal I$}
\newcommand{\realnumb}{$\cal R$}
\newcommand{\cmplxnumb}{$\cal C$}

% misc. commands

\newcommand{\mma}{{\it Mathematica}}
\newcommand{\sech}{\mbox{ sech}}
 
\newtheorem{theorem}{Theorem}
\newtheorem{example}{Example}
\newtheorem{definition}{Definition}
\newtheorem{problem}{Problem}

\setcounter{secnumdepth}{2}
\setcounter{tocdepth}{4}


\begin{document}
\maketitle
\newpage
%%%%%%%%%%%%%%%%%%%%%%%%%%%%%%%%%%%%%%%%%%%%%%%%%%%%%%%%%%%%%%%%%%%%%%%%%%%%%%%%
%%%%%%%%%%%%%%%%%%%%%%%%%%%%%%%%%%%%%%%%%%%%%%%%%%%%%%%%%%%%%%%%%%%%%%%%%%%%%%%%
\vskip0.1in\hrule\vskip0.1in
\noindent
{\bf Absolute and Relative Error: Introduction.} 
\vskip0.1in\hrule\vskip0.1in
\noindent
In most cases, the solution of a mathematical problem can only be approximated.
So, it would be a good idea to have a way to measure the error between the exact
solution and the approximation of that solution. We will define two types of
error. These are absolute error and relative error. The {\bf absolute error} is
the absolute value of the difference between the approximation and the exact
value for the solution. That is, if $x^*$ is the exact value approximated by
$x$, then
$$ e_{abs} = | x - x^{*} | $$
defines the absolute error. The {\bf relative error} is a scaled error defined
by
$$ e_{rel} = {{| x - x^{*} |}\over{|x^*|}} $$
So, the relative error is a scaled or percent error based on the magnitude of
the exact value.
%%%%%%%%%%%%%%%%%%%%%%%%%%%%%%%%%%%%%%%%%%%%%%%%%%%%%%%%%%%%%%%%%%%%%%%%%%%%%%%%
%%%%%%%%%%%%%%%%%%%%%%%%%%%%%%%%%%%%%%%%%%%%%%%%%%%%%%%%%%%%%%%%%%%%%%%%%%%%%%%%

%%%%%%%%%%%%%%%%%%%%%%%%%%%%%%%%%%%%%%%%%%%%%%%%%%%%%%%%%%%%%%%%%%%%%%%%%%%%%%%%
%%%%%%%%%%%%%%%%%%%%%%%%%%%%%%%%%%%%%%%%%%%%%%%%%%%%%%%%%%%%%%%%%%%%%%%%%%%%%%%%
\vskip0.1in\hrule\vskip0.1in
\noindent
{\bf Absolute and Relative Error: Examples.} 
\vskip0.1in\hrule\vskip0.1in
\noindent
If we are trying to find the roots of the polynomial
$$
  p(x) = x^5 + x^3 - 2\ x^2 + 5\ x
$$
we can see that $x=0$ is one solution. To find other roots we can use Newton's
method to generate a sequence of approximations given a starting point. That is,
we will generate a sequence
$$
   S = \left\lbrace x_k \right\rbrace_{k=0}^{\infty}
$$
Newton's method will be covered a bit later in this course. However, what we
will want is for the sequence to converge to a root, say $x^*$. This can be
rephrased as
$$ | x_k - x^* | \rightarrow 0 $$
which implies the absolute error will tend to zero as $k$ tends to $\infty$.
Using the relative error we want
$$ {{ | x_k - x^* | }\over{ | x^* | }} \rightarrow 0 $$
For the polynomial define in this example, there will be problems in using the
relative error near the zero roots. So, if the sequence starts to converg to the
zero root, we would need to use the absolute error as a measure.
%%%%%%%%%%%%%%%%%%%%%%%%%%%%%%%%%%%%%%%%%%%%%%%%%%%%%%%%%%%%%%%%%%%%%%%%%%%%%%%%
%%%%%%%%%%%%%%%%%%%%%%%%%%%%%%%%%%%%%%%%%%%%%%%%%%%%%%%%%%%%%%%%%%%%%%%%%%%%%%%%

%%%%%%%%%%%%%%%%%%%%%%%%%%%%%%%%%%%%%%%%%%%%%%%%%%%%%%%%%%%%%%%%%%%%%%%%%%%%%%%%
%%%%%%%%%%%%%%%%%%%%%%%%%%%%%%%%%%%%%%%%%%%%%%%%%%%%%%%%%%%%%%%%%%%%%%%%%%%%%%%%
\vskip0.1in\hrule\vskip0.1in
\noindent
{\bf Absolute and Relative Error: A Numerical Example.} 
\vskip0.1in\hrule\vskip0.1in
\noindent
As an illustration of how absolute and relative errors compare, we can consider
some numerical examples. The following table gives a pair of numbers along with
both the absolute and relative errors.

\begin{table}[h!]
   \caption{Absolute and Relative Error Values}
   \begin{center}
   \begin{tabular}{c|c|c|c}
     \hline
     \textbf{$x$} & \textbf{$x^*$} & \textbf{abs. err.} & \textbf{rel. err.} \\
     \hline
     0.01 & 0.1 & 0.09 & 0.9 \\
     \hline
     1.01 & 1.0 & 0.01 & 0.01 \\
     \hline
     2.0   & 3.0  & 1.0 & 0.5 \\
     \hline
     10.0  & 9.0  & 1.0 & 0.1 \\
     \hline
     100.0 & 99.0 & 1.0 & 0.01 \\
     \hline
   \end{tabular}
   \end{center}
\end{table}

Note that when both the approximation and exact value are close to zero, the
absolute error becomes a better measure of error than the relative error and
for large values (at least greater than one) one should expect that the better
measure of the error is the relative error. We will use both in the development
of algorithms to numerically solve mathematical problems.






The following is a list of sources for error that need to be taken into account
by compoutational scientists.
\begin{enumerate}
  \item {\bf Modeling Errors} These errors can occur when assumptions are made
        about the phenomena being studied. For example, one may consider a model
        of the solar system where the planets are assumed to be spheres, which
        is not the case.
  \item {\bf Measurement Errors:} These errors occur when instruments are used
        to measure physical quantities. For example, the temperature of molten
        lava might be measured to within one or two degrees based on the
        magnitude of the exact temperature. The fractional part of the 
        measurement would characterize the error.
  \item {\bf Discretization Error:} In order to computer solutions to
        mathematical problems using computers, it necessary that the model be
        finite and discrete. For example, weather models based on systems of
        partial differential equations require a discretization of the continuous model to fit in the
        discrete framework of a computer simulation.
\end{enumerate}
Note that Github will not allow you the luxury of creating empty folders. This
is an advantage in using \lq\lq git\rq\rq\ on a local machine. When changes are
\lq\lq pushed\rq\rq\ to Github, empty folders are ignored. So, let's get started
on the formatting the homework solutions portion of the repository for the
class.
\newpage
%%%%%%%%%%%%%%%%%%%%%%%%%%%%%%%%%%%%%%%%%%%%%%%%%%%%%%%%%%%%%%%%%%%%%%%%%%%%%%%%
%%%%%%%%%%%%%%%%%%%%%%%%%%%%%%%%%%%%%%%%%%%%%%%%%%%%%%%%%%%%%%%%%%%%%%%%%%%%%%%%
\vskip0.1in\hrule\vskip0.1in
\noindent
{\bf Homework Repository for Math 4610: Get to Github.} 
\vskip0.1in\hrule\vskip0.1in
The first step is to login to your Github account. So, in a web browser, type in
\begin{verbatim}

    https://github.com/

\end{verbatim}
and log on to your account. By this point you should already have an account on
Github.
\vfill
\begin{figure}[h]
\centering
\includegraphics[width=0.75\textwidth]{../images/github_01.png}
\caption{{Screenshot} taken using {\bf Snip \& Sketch}. This is an app on
         my Windows 10 box}
\end{figure}
\eject
%%%%%%%%%%%%%%%%%%%%%%%%%%%%%%%%%%%%%%%%%%%%%%%%%%%%%%%%%%%%%%%%%%%%%%%%%%%%%%%%
%%%%%%%%%%%%%%%%%%%%%%%%%%%%%%%%%%%%%%%%%%%%%%%%%%%%%%%%%%%%%%%%%%%%%%%%%%%%%%%%
\vskip0.1in\hrule\vskip0.1in
\noindent
{\bf Homework Repository for Math 4610: Login to Your Account.} 
\vskip0.1in\hrule\vskip0.1in
Use the popup to login to Github with your user name and password.
\vfill
\begin{figure}[h]
\centering
\includegraphics[width=0.75\textwidth]{../images/github_02.png}
\caption{{Screenshot} taken using {\bf Snip \& Sketch}. This is an app on
         my Windows 10 box}
\end{figure}
\eject
%%%%%%%%%%%%%%%%%%%%%%%%%%%%%%%%%%%%%%%%%%%%%%%%%%%%%%%%%%%%%%%%%%%%%%%%%%%%%%%%
%%%%%%%%%%%%%%%%%%%%%%%%%%%%%%%%%%%%%%%%%%%%%%%%%%%%%%%%%%%%%%%%%%%%%%%%%%%%%%%%
\vskip0.1in\hrule\vskip0.1in
\noindent
{\bf Homework Repository for Math 4610: Starting Point for Working on a
Repository.} 
\vskip0.1in\hrule\vskip0.1in
Once the following pops up, we can navigate through the repositories if you
have more than one. In any case, you should have a repository named
\begin{verbatim}

    math4610

\end{verbatim}
Click on this repository to start creating files and the like. Once the
repository is created, students can click on the name to work on the repository
or use files in the repository.
\vfill
\begin{figure}[h]
\centering
\includegraphics[width=0.75\textwidth]{../images/github_03.png}
\caption{{Screenshot} taken using {\bf Snip \& Sketch}. This is an app on
         my Windows 10 box}
\end{figure}
\eject
%%%%%%%%%%%%%%%%%%%%%%%%%%%%%%%%%%%%%%%%%%%%%%%%%%%%%%%%%%%%%%%%%%%%%%%%%%%%%%%%
%%%%%%%%%%%%%%%%%%%%%%%%%%%%%%%%%%%%%%%%%%%%%%%%%%%%%%%%%%%%%%%%%%%%%%%%%%%%%%%%
\vskip0.1in\hrule\vskip0.1in
\noindent
{\bf Homework Repository for Math 4610: Creating a File.} 
\vskip0.1in\hrule\vskip0.1in
To start putting together the repository for submitting task solutions, students
can create a new file by clicking on the button as shown below. In particular,
students should create the table of contents file for the task sheets. The goal
is to get to the repository and create the table of contents for the solutions
to the tasks that will be completed.
\vfill
\begin{figure}[h]
\centering
\includegraphics[width=0.75\textwidth]{../images/github_04.png}
\caption{{Screenshot} taken using {\bf Snip \& Sketch}. This is an app on
         my Windows 10 box}
\end{figure}
\eject
%%%%%%%%%%%%%%%%%%%%%%%%%%%%%%%%%%%%%%%%%%%%%%%%%%%%%%%%%%%%%%%%%%%%%%%%%%%%%%%%
%%%%%%%%%%%%%%%%%%%%%%%%%%%%%%%%%%%%%%%%%%%%%%%%%%%%%%%%%%%%%%%%%%%%%%%%%%%%%%%%
\vskip0.1in\hrule\vskip0.1in
\noindent
{\bf Homework Repository for Math 4610: Type In Lines to Create the Table of
Contents.} 
\vskip0.1in\hrule\vskip0.1in
In the following figure, two things should be noted. The first is the name of
the file
\begin{verbatim}

    hw_toc.md

\end{verbatim}
Note that the extension, \lq\lq .md\rq\rq\ indicates to a browser that this is a
MarkDown file. We will spend more time on using MarkDown in this and other
lectures in the course. The second piece of the puzzle is the circled region
where you can type in lines that will be used in the file. All you need to do is
click to the right of the numbered line and start typing.
\vfill
\begin{figure}[h]
\centering
\includegraphics[width=0.75\textwidth]{../images/github_05.png}
\caption{{Screenshot} taken using {\bf Snip \& Sketch}. This is an app on
         my Windows 10 box}
\end{figure}
\eject
%%%%%%%%%%%%%%%%%%%%%%%%%%%%%%%%%%%%%%%%%%%%%%%%%%%%%%%%%%%%%%%%%%%%%%%%%%%%%%%%
%%%%%%%%%%%%%%%%%%%%%%%%%%%%%%%%%%%%%%%%%%%%%%%%%%%%%%%%%%%%%%%%%%%%%%%%%%%%%%%%
\vskip0.1in\hrule\vskip0.1in
\noindent
{\bf Homework Repository for Math 4610: A Start to a Table of Contents.} 
\vskip0.1in\hrule\vskip0.1in
To give an idea of how to use MarkDown to set up a table of contents. Each of
the lines serve a function in the table of contents.
\begin{verbatim}

    # Math 4610 Homework Solutions

\end{verbatim}
This line is a header line due to the pound sign. The second nonblank line is a
header line with a smaller font size. Note that these lines are short hand in
HTML for $<h1>$ and $<h2>$. The next two lines provide links to other files that
will contain your homework solutions. Note that the asterisk preceding the text
in these lines indicates a bullet should be placed in front of the text. There
are tasks in the homework that will walk you through at least some subset of
MarkDown syntax. Note that one of the tasks requires students work through a
tutorial on Markdown syntax.
\vfill
\begin{figure}[h]
\centering
\includegraphics[width=0.75\textwidth]{../images/github_06.png}
\caption{{Screenshot} taken using {\bf Snip \& Sketch}. This is an app on
         my Windows 10 box}
\end{figure}
\eject
%%%%%%%%%%%%%%%%%%%%%%%%%%%%%%%%%%%%%%%%%%%%%%%%%%%%%%%%%%%%%%%%%%%%%%%%%%%%%%%%
%%%%%%%%%%%%%%%%%%%%%%%%%%%%%%%%%%%%%%%%%%%%%%%%%%%%%%%%%%%%%%%%%%%%%%%%%%%%%%%%
\vskip0.1in\hrule\vskip0.1in
\noindent
{\bf Homework Repository for Math 4610: Adding and Committing the File.} 
\vskip0.1in\hrule\vskip0.1in
Version Control Systems (VCS) like git do not make changes to a repository 
unless a commit has been made. If you scroll down to the bottom of the webpage
there are a couple of boxes and buttons to consider. The textboxes allow the
user to enter comments about the changes being made to the repository. It is
strongly recommended that students add comments about the changes being made.
Finally, the is a button that will seal the deal on the modifications. To
push the changes, click on the
\begin{verbatim}

    Commit new file

\end{verbatim}
button. If you do not want to keep the changes, click on the
\begin{verbatim}

    Cancel

\end{verbatim}
If you cancel the changes, a popup will appear that will allow you to reconsider
the choice. So, click on the Commit button.
\vfill
\begin{figure}[h]
\centering
\includegraphics[width=0.75\textwidth]{../images/github_07.png}
\caption{{Screenshot} taken using {\bf Snip \& Sketch}. This is an app on
         my Windows 10 box}
\end{figure}
\eject
%%%%%%%%%%%%%%%%%%%%%%%%%%%%%%%%%%%%%%%%%%%%%%%%%%%%%%%%%%%%%%%%%%%%%%%%%%%%%%%%
%%%%%%%%%%%%%%%%%%%%%%%%%%%%%%%%%%%%%%%%%%%%%%%%%%%%%%%%%%%%%%%%%%%%%%%%%%%%%%%%
\vskip0.1in\hrule\vskip0.1in
\noindent
{\bf Homework Repository for Math 4610: Another Example of File Creation.} 
\vskip0.1in\hrule\vskip0.1in
There are a couple of ways to create folders in a repository. One way is to go
back into the editor function on Github and modify the name of an existing file.
For this part of the lecture, a different repository will be used since the
instructor already has a repository named \lq\lq math4610\rq\rq. Students should
use their math4610 repository to follow along with this example. To start the
process, create a file
\begin{verbatim}

    harmless.md

\end{verbatim}
You can include some text as shown and then commit the change as shown above.
\vfill
\begin{figure}[h]
\centering
\includegraphics[width=0.75\textwidth]{../images/github_08.png}
\caption{{Screenshot} taken using {\bf Snip \& Sketch}. This is an app on
         my Windows 10 box}
\end{figure}
\eject
%%%%%%%%%%%%%%%%%%%%%%%%%%%%%%%%%%%%%%%%%%%%%%%%%%%%%%%%%%%%%%%%%%%%%%%%%%%%%%%%
%%%%%%%%%%%%%%%%%%%%%%%%%%%%%%%%%%%%%%%%%%%%%%%%%%%%%%%%%%%%%%%%%%%%%%%%%%%%%%%%
\vskip0.1in\hrule\vskip0.1in
\noindent
{\bf Homework Repository for Math 4610: Find the Filename in the Repository.} 
\vskip0.1in\hrule\vskip0.1in
Next, click on the repository name at the top of the webpage. Students should
see the file name
\begin{verbatim}

    harmless.md

\end{verbatim}
in the list of files. Click on the filename in the list to show the contents of
the file. 
\vfill
\begin{figure}[h]
\centering
\includegraphics[width=0.75\textwidth]{../images/github_09.png}
\caption{{Screenshot} taken using {\bf Snip \& Sketch}. This is an app on
         my Windows 10 box}
\end{figure}
\eject
%%%%%%%%%%%%%%%%%%%%%%%%%%%%%%%%%%%%%%%%%%%%%%%%%%%%%%%%%%%%%%%%%%%%%%%%%%%%%%%%
%%%%%%%%%%%%%%%%%%%%%%%%%%%%%%%%%%%%%%%%%%%%%%%%%%%%%%%%%%%%%%%%%%%%%%%%%%%%%%%%
\vskip0.1in\hrule\vskip0.1in
\noindent
{\bf Homework Repository for Math 4610: Rename the File while Creating a
 Subdirectory.} 
\vskip0.1in\hrule\vskip0.1in
Click on the file
\begin{verbatim}

    harmless.md

\end{verbatim}
to open up the contents of the file for editing. To modify the file name, click
on the little pencil to start editing the file. This will show the file name in
a box and allow changes to be made in the file.
\vfill
\begin{figure}[h]
\centering
\includegraphics[width=0.75\textwidth]{../images/github_10.png}
\caption{{Screenshot} taken using {\bf Snip \& Sketch}. This is an app on
         my Windows 10 box}
\end{figure}
\eject
%%%%%%%%%%%%%%%%%%%%%%%%%%%%%%%%%%%%%%%%%%%%%%%%%%%%%%%%%%%%%%%%%%%%%%%%%%%%%%%%
%%%%%%%%%%%%%%%%%%%%%%%%%%%%%%%%%%%%%%%%%%%%%%%%%%%%%%%%%%%%%%%%%%%%%%%%%%%%%%%%
\vskip0.1in\hrule\vskip0.1in
\noindent
{\bf Homework Repository for Math 4610: Rename the File and Create a
 Subdirectory.} 
\vskip0.1in\hrule\vskip0.1in
The last step is to click at the beginning of the file name. The box that the
file name appears in is editable. So, you can change the name or include the
file separator that is used to create a new folder. Note that since there is
a folder in the file name, Github will be glad to create the folder for you.
In this example, the subdirectory
\begin{verbatim}

    temp

\end{verbatim}
is created as an example. You can name the folder anything you like.
\vfill
\begin{figure}[h]
\centering
\includegraphics[width=0.75\textwidth]{../images/github_11.png}
\caption{{Screenshot} taken using {\bf Snip \& Sketch}. This is an app on
         my Windows 10 box}
\end{figure}
\eject
%%%%%%%%%%%%%%%%%%%%%%%%%%%%%%%%%%%%%%%%%%%%%%%%%%%%%%%%%%%%%%%%%%%%%%%%%%%%%%%%
%%%%%%%%%%%%%%%%%%%%%%%%%%%%%%%%%%%%%%%%%%%%%%%%%%%%%%%%%%%%%%%%%%%%%%%%%%%%%%%%
\vskip0.1in\hrule\vskip0.1in
\noindent
{\bf Homework Repository for Math 4610: Commit The New File and Folder.} 
\vskip0.1in\hrule\vskip0.1in
You must commit the changes using the
\begin{verbatim}

    Commit changes

\end{verbatim}
button near the bottom of the page.
\vfill
\begin{figure}[h]
\centering
\includegraphics[width=0.75\textwidth]{../images/github_12.png}
\caption{{Screenshot} taken using {\bf Snip \& Sketch}. This is an app on
         my Windows 10 box}
\end{figure}
\eject
%%%%%%%%%%%%%%%%%%%%%%%%%%%%%%%%%%%%%%%%%%%%%%%%%%%%%%%%%%%%%%%%%%%%%%%%%%%%%%%%
%%%%%%%%%%%%%%%%%%%%%%%%%%%%%%%%%%%%%%%%%%%%%%%%%%%%%%%%%%%%%%%%%%%%%%%%%%%%%%%%
\vskip0.1in\hrule\vskip0.1in
\noindent
{\bf Homework Repository for Math 4610: Finding the new File and Folder in the
Repository.} 
\vskip0.1in\hrule\vskip0.1in
The folder name will appear in the list of files and folders for the repository.
Note that folders are listed first. Clicking on the folder
\begin{verbatim}

    temp

\end{verbatim}
will open the folder and show the file created. Recall that Github does not
like or allow empty folders. So, there must be something in each and every
folder.
\vfill
\begin{figure}[h]
\centering
\includegraphics[width=0.75\textwidth]{../images/github_13.png}
\caption{{Screenshot} taken using {\bf Snip \& Sketch}. This is an app on
         my Windows 10 box}
\end{figure}
\eject
%%%%%%%%%%%%%%%%%%%%%%%%%%%%%%%%%%%%%%%%%%%%%%%%%%%%%%%%%%%%%%%%%%%%%%%%%%%%%%%%
%%%%%%%%%%%%%%%%%%%%%%%%%%%%%%%%%%%%%%%%%%%%%%%%%%%%%%%%%%%%%%%%%%%%%%%%%%%%%%%%
\vskip0.1in\hrule\vskip0.1in
\noindent
{\bf Homework Repository for Math 4610: Deleting Folders
Repository.} 
\vskip0.1in\hrule\vskip0.1in
\vfill
To start the process, click on the folder you want to get rid of and click on
the folder name. This will show the content of the folder.
\begin{figure}[h]
\centering
\includegraphics[width=0.75\textwidth]{../images/github_14.png}
\caption{{Screenshot} taken using {\bf Snip \& Sketch}. This is an app on
         my Windows 10 box}
\end{figure}
\eject
%%%%%%%%%%%%%%%%%%%%%%%%%%%%%%%%%%%%%%%%%%%%%%%%%%%%%%%%%%%%%%%%%%%%%%%%%%%%%%%%
%%%%%%%%%%%%%%%%%%%%%%%%%%%%%%%%%%%%%%%%%%%%%%%%%%%%%%%%%%%%%%%%%%%%%%%%%%%%%%%%
\vskip0.1in\hrule\vskip0.1in
\noindent
{\bf Homework Repository for Math 4610: Deleting the Files in the Folder
Repository.} 
\vskip0.1in\hrule\vskip0.1in
\vfill
Delete the files one by one, by clicking on each file and then choose the
garbage can to delete each of the files. Note that there will be one more step
when all the files have been deleted. That is the Commit change button must be
clicked to make all changes final.
\begin{figure}[h]
\centering
\includegraphics[width=0.75\textwidth]{../images/github_15.png}
\caption{{Screenshot} taken using {\bf Snip \& Sketch}. This is an app on
         my Windows 10 box}
\end{figure}
%%%%%%%%%%%%%%%%%%%%%%%%%%%%%%%%%%%%%%%%%%%%%%%%%%%%%%%%%%%%%%%%%%%%%%%%%%%%%%%%
%%%%%%%%%%%%%%%%%%%%%%%%%%%%%%%%%%%%%%%%%%%%%%%%%%%%%%%%%%%%%%%%%%%%%%%%%%%%%%%%
\vskip0.1in\hrule\vskip0.1in
\noindent
{\bf Homework Repository for Math 4610: Commit the Changes.} 
\vskip0.1in\hrule\vskip0.1in
\vfill
When you are done deleting the files, click the Commit change button. Note that
when all the files have been dumped in the trash, since there is nothing left in
the folder, Github automatically dumps the folder in the trash. In the example,
here when the file 
\begin{verbatim}

    harmless.md

\end{verbatim}
is deleted, students should see the folder contents for the parent directory
show up. The folder
\begin{verbatim}

    temp

\end{verbatim}
will not appear. This will be shown below.
\begin{figure}[h]
\centering
\includegraphics[width=0.75\textwidth]{../images/github_16.png}
\caption{{Screenshot} taken using {\bf Snip \& Sketch}. This is an app on
         my Windows 10 box}
\end{figure}
%%%%%%%%%%%%%%%%%%%%%%%%%%%%%%%%%%%%%%%%%%%%%%%%%%%%%%%%%%%%%%%%%%%%%%%%%%%%%%%%
%%%%%%%%%%%%%%%%%%%%%%%%%%%%%%%%%%%%%%%%%%%%%%%%%%%%%%%%%%%%%%%%%%%%%%%%%%%%%%%%
\vskip0.1in\hrule\vskip0.1in
\noindent
{\bf Homework Repository for Math 4610: Check to See if the File is Deleted.} 
\vskip0.1in\hrule\vskip0.1in
\vfill
The last step is to see if the file has been deleted in the Github repository.
Click on the repository while logged in and look for the
\begin{verbatim}

    temp

\end{verbatim}
folder. Note that by deleting the file
\begin{verbatim}

    harmless

\end{verbatim}
Github will throw away the folder and it will gone.
\begin{figure}[h]
\centering
\includegraphics[width=0.75\textwidth]{../images/github_17.png}
\caption{{Screenshot} taken using {\bf Snip \& Sketch}. This is an app on
         my Windows 10 box}
\end{figure}
%%%%%%%%%%%%%%%%%%%%%%%%%%%%%%%%%%%%%%%%%%%%%%%%%%%%%%%%%%%%%%%%%%%%%%%%%%%%%%%%
%%%%%%%%%%%%%%%%%%%%%%%%%%%%%%%%%%%%%%%%%%%%%%%%%%%%%%%%%%%%%%%%%%%%%%%%%%%%%%%%
\end{document}
