\documentclass[10pt,fleqn]{article}
%\usepackage{graphicx}


\setlength{\topmargin}{-.75in}
\addtolength{\textheight}{2.00in}
\setlength{\oddsidemargin}{.00in}
\addtolength{\textwidth}{.75in}

\nofiles

\pagestyle{empty}

\setlength{\parindent}{0in}

% new math commands


\setlength{\oddsidemargin}{-0.25in}
\setlength{\evensidemargin}{-0.25in}
\setlength{\textwidth}{6.75in}
\setlength{\headheight}{0.0in}
\setlength{\topmargin}{-0.25in}
\setlength{\textheight}{9.00in}

\makeindex

\usepackage{mathrsfs}

%\usepackage[pdftex]{graphicx}
\usepackage{epstopdf}

\newcounter{beans}

\newcommand{\ds}{\displaystyle}
\newcommand{\limit}[2]{\displaystyle\lim_{#1\to#2}}

\newcommand{\binomial}[2]{\ \left( \begin{array}{c}
                                  #1 \\
                                  #2
                                 \end{array}
                            \right) \
                         }
\newcommand{\ExampleRule}[2]
  {
  \noindent
  \rule{\linewidth}{1pt}
  \begin{example}
    #1
    \label{#2}
  \end{example}
  \rule{\linewidth}{1pt}
  \vskip0.125in
  }

\newcommand{\defbox}[1]
  {
   \ \\
   \noindent
   \setlength\fboxrule{1pt}
   \fbox{
        \begin{minipage}{6.5in}
          #1
        \end{minipage}
        }
   \ \\
  }
\newcommand{\verysmallworkbox}[1]
  {
   \ \\
   \noindent
   \setlength\fboxrule{1pt}
   \fbox{
        \begin{minipage}{6.5in}
           #1
           \ \\
           \vskip0.5in \ \\
           \ \\
        \end{minipage}
        }
   \ \\
  }
\newcommand{\smallworkbox}[1]
  {
   \ \\
   \noindent
   \setlength\fboxrule{1pt}
   \fbox{
        \begin{minipage}{6.5in}
           #1
           \ \\
           \vskip2.5in \ \\
           \ \\
        \end{minipage}
        }
   \ \\
  }
\newcommand{\halfworkbox}[1]
  {
   \ \\
   \noindent
   \setlength\fboxrule{1pt}
   \fbox{
        \begin{minipage}{6.5in}
           #1 \hfill
           \ \\
           \vskip3.25in \ \\
           \ \\
        \end{minipage}
        }
   \ \\
  }
\newcommand{\largeworkbox}[1]
  {
   \ \\
   \noindent
   \setlength\fboxrule{1pt}
   \fbox{
        \begin{minipage}{6.5in}
           #1
           \ \\
           \vskip7.5in \ \\
           \ \\
        \end{minipage}
        }
   \ \\
  }
\newcommand{\flexworkbox}[2]
  {
   \ \\
   \noindent
   \setlength\fboxrule{1pt}
   \fbox{
        \begin{minipage}{6.5in}
           #1
           \ \\

           \vskip#2 \ \\
           \ \\
        \end{minipage}
        }
   \ \\
  }


% symbols for sets of numbers

\newcommand{\natnumb}{$\cal N$}
\newcommand{\whonumb}{$\cal W$}
\newcommand{\intnumb}{$\cal Z$}
\newcommand{\ratnumb}{$\cal Q$}
\newcommand{\irrnumb}{$\cal I$}
\newcommand{\realnumb}{$\cal R$}
\newcommand{\cmplxnumb}{$\cal C$}

% misc. commands

\newcommand{\mma}{{\it Mathematica}}
\newcommand{\sech}{\mbox{ sech}}
 
\newtheorem{theorem}{Theorem}
\newtheorem{example}{Example}
\newtheorem{definition}{Definition}
\newtheorem{problem}{Problem}

\setcounter{secnumdepth}{2}
\setcounter{tocdepth}{4}


\begin{document}
%%%%%%%%%%%%%%%%%%%%%%%%%%%%%%%%%%%%%%%%%%%%%%%%%%%%%%%%%%%%%%%%%%%%%%%%%%%%%%%%
%%%%%%%%%%%%%%%%%%%%%%%%%%%%%%%%%%%%%%%%%%%%%%%%%%%%%%%%%%%%%%%%%%%%%%%%%%%%%%%%
\vskip0.1in\hrule\vskip0.1in
\noindent
{\bf Math 4610 Fundamentals of Computational Mathematics  -  Floating Point 
    Representation of Numbers.}  
\vskip0.1in\hrule\vskip0.1in
\noindent
Any work that is done on a computer boils down to manipulating numbers. A
problem with this is that computers have finite resources and the representation
of many numbers requires the use of an infinite number of decimal digits. For
example, given a circle, the formula for the circumference is
$$C = 2 \times \pi \times r = \pi \times d$$
where $r$ is the radius of the circle and $d$ is the diameter of the circle. The
number $\pi$ is not a rational number. That is, the decimal expansion of this
value has an infinite fractional part. The value can be represented as follows:
$$\pi\approx 3.141592653589793...$$
where the ellipsis notation, $...$, means the digits never repeat. So, to get an
exact representation of $\pi$ it is necessary to have an infinite number of
digits available. Since compouter resources are finite, we must settle for an
approximation.

In this part of the lecture, we will use a few examples that should motivate us
to spend some time on this issue and more fully understand the implications of
finite precision of number representation.

For the first example, we could use the approximation
$$\pi\approx 3.141592653589793$$
without including an infinite number of digits. One question that should arise
is how many digits will provide us with an accurate enough approximation. One
of the programs used over the past few decades to \lq\lq\ burn in machines was
an algorithm to compute more and more digits of $\pi$. This means that is
possible to determine $\pi$ to any degree of accuracy that we want. However, it
is not practical for real problems.

In some cases, a very crude approximation is enough. In some of our United
States, laws have been passed to legally approximate $\pi$ using a rational
number. For example,
$$\pi\approx {{22}\over 7}$$
provides an approximation that will hold up in a courts of law. If you are
pouring a circular concrete slab for a water tank it is a good idea to have an
estimate of the amount of concrete based on an accepted value for the number
$\pi$.

Basically, numbers are best represented on a computer using zeros and ones -
or in a binary number system. Other common number systems used in computer
archetecture/hardware are in octal (or base 8) and hexidecimal (or base 16).
Another issue that arises in the representation of numbers is numbers that are
relatively prime to base 2. As a simple example, consider the representation of
the number $1/3$ in base 2.  The value is
$${1\over 3} = 0.01010101.....$$
where the last pair of digits repeats forever. If a finite number of binary
digits are used to represent $1/3$, the result is an approximation of the exact
value. Note that a base 10 representation of $1/3$ is given by the decimal
representation
$${1\over 3} = 0.3333333333333.........$$
Even if computers worked in a base 10 system, we would necessarily have to
settle for approximate number representation.

Since there are an uncountable number of irrational numbers, it is impossible to
imagine a computer that would not suffer the same issue. So, the best we can
hope for is that there is an accepted number representation that will work on
all computers. There is a standard (IEEE standard reference here) for number
representation that we will look into later in the course. For now, we will
assume that all of the computers we will use will behave the same way. From a
practical point of view, it would be nice to be able to compute the limits of
the accuracy of machine numbers. Fortunately, we can write a little program that
will do the trick for us.
%%%%%%%%%%%%%%%%%%%%%%%%%%%%%%%%%%%%%%%%%%%%%%%%%%%%%%%%%%%%%%%%%%%%%%%%%%%%%%%%
%%%%%%%%%%%%%%%%%%%%%%%%%%%%%%%%%%%%%%%%%%%%%%%%%%%%%%%%%%%%%%%%%%%%%%%%%%%%%%%%
\end{document}
