\documentclass[10pt,fleqn]{article}
%\usepackage{graphicx}


\setlength{\topmargin}{-.75in}
\addtolength{\textheight}{2.00in}
\setlength{\oddsidemargin}{.00in}
\addtolength{\textwidth}{.75in}

\title{Math 4610 Lecture Notes \\
            \ \\
            How to Build a Shared Library
  \footnote{These notes are part of an Open Resource Educational project
            sponsored by Utah State University}}

\author{Joe Koebbe}

\nofiles

\pagestyle{empty}

\setlength{\parindent}{0in}

% new math commands


\setlength{\oddsidemargin}{-0.25in}
\setlength{\evensidemargin}{-0.25in}
\setlength{\textwidth}{6.75in}
\setlength{\headheight}{0.0in}
\setlength{\topmargin}{-0.25in}
\setlength{\textheight}{9.00in}

\makeindex

\usepackage{mathrsfs}

%\usepackage[pdftex]{graphicx}
\usepackage{epstopdf}

\newcounter{beans}

\newcommand{\ds}{\displaystyle}
\newcommand{\limit}[2]{\displaystyle\lim_{#1\to#2}}

\newcommand{\binomial}[2]{\ \left( \begin{array}{c}
                                  #1 \\
                                  #2
                                 \end{array}
                            \right) \
                         }
\newcommand{\ExampleRule}[2]
  {
  \noindent
  \rule{\linewidth}{1pt}
  \begin{example}
    #1
    \label{#2}
  \end{example}
  \rule{\linewidth}{1pt}
  \vskip0.125in
  }

\newcommand{\defbox}[1]
  {
   \ \\
   \noindent
   \setlength\fboxrule{1pt}
   \fbox{
        \begin{minipage}{6.5in}
          #1
        \end{minipage}
        }
   \ \\
  }
\newcommand{\verysmallworkbox}[1]
  {
   \ \\
   \noindent
   \setlength\fboxrule{1pt}
   \fbox{
        \begin{minipage}{6.5in}
           #1
           \ \\
           \vskip0.5in \ \\
           \ \\
        \end{minipage}
        }
   \ \\
  }
\newcommand{\smallworkbox}[1]
  {
   \ \\
   \noindent
   \setlength\fboxrule{1pt}
   \fbox{
        \begin{minipage}{6.5in}
           #1
           \ \\
           \vskip2.5in \ \\
           \ \\
        \end{minipage}
        }
   \ \\
  }
\newcommand{\halfworkbox}[1]
  {
   \ \\
   \noindent
   \setlength\fboxrule{1pt}
   \fbox{
        \begin{minipage}{6.5in}
           #1 \hfill
           \ \\
           \vskip3.25in \ \\
           \ \\
        \end{minipage}
        }
   \ \\
  }
\newcommand{\largeworkbox}[1]
  {
   \ \\
   \noindent
   \setlength\fboxrule{1pt}
   \fbox{
        \begin{minipage}{6.5in}
           #1
           \ \\
           \vskip7.5in \ \\
           \ \\
        \end{minipage}
        }
   \ \\
  }
\newcommand{\flexworkbox}[2]
  {
   \ \\
   \noindent
   \setlength\fboxrule{1pt}
   \fbox{
        \begin{minipage}{6.5in}
           #1
           \ \\

           \vskip#2 \ \\
           \ \\
        \end{minipage}
        }
   \ \\
  }


% symbols for sets of numbers

\newcommand{\natnumb}{$\cal N$}
\newcommand{\whonumb}{$\cal W$}
\newcommand{\intnumb}{$\cal Z$}
\newcommand{\ratnumb}{$\cal Q$}
\newcommand{\irrnumb}{$\cal I$}
\newcommand{\realnumb}{$\cal R$}
\newcommand{\cmplxnumb}{$\cal C$}

% misc. commands

\newcommand{\mma}{{\it Mathematica}}
\newcommand{\sech}{\mbox{ sech}}
 
\newtheorem{theorem}{Theorem}
\newtheorem{example}{Example}
\newtheorem{definition}{Definition}
\newtheorem{problem}{Problem}

\setcounter{secnumdepth}{2}
\setcounter{tocdepth}{4}


\begin{document}
\maketitle
\newpage
%%%%%%%%%%%%%%%%%%%%%%%%%%%%%%%%%%%%%%%%%%%%%%%%%%%%%%%%%%%%%%%%%%%%%%%%%%%%%%%%
%%%%%%%%%%%%%%%%%%%%%%%%%%%%%%%%%%%%%%%%%%%%%%%%%%%%%%%%%%%%%%%%%%%%%%%%%%%%%%%%
\vskip0.1in\hrule\vskip0.1in
\noindent
{\bf Steps for Building a Shared Library:}
\vskip0.1in\hrule\vskip0.1in
\noindent
Shared libraries are essential for reusing code. If there are a number of
routines or methods that you plan to use over and over, it makes sense to write
the code once, debug the code and then keep the code in a central location. Then
when the codes are needed, a code can link to the shared library and access the
code. In this part of the course, we will learn to build and use shared
libraries.
%%%%%%%%%%%%%%%%%%%%%%%%%%%%%%%%%%%%%%%%%%%%%%%%%%%%%%%%%%%%%%%%%%%%%%%%%%%%%%%%
%%%%%%%%%%%%%%%%%%%%%%%%%%%%%%%%%%%%%%%%%%%%%%%%%%%%%%%%%%%%%%%%%%%%%%%%%%%%%%%%
\vskip0.1in\hrule\vskip0.1in
{\large{\bf Setting Things Up}}
\vskip0.1in\hrule\vskip0.1in
\noindent
To complete this work, you will need Cygwin or Linux up on your own laptop or go
to the Engineering Computer Lab on the third floor of the Engineering Building.
Once you have a command line terminal up and running, start by making a
directory in which to develop and keep. So, create a folder in the

     math4610

folder. Note that this is not a repository at this point. If you do not have a
folder named math4610 and then use

     cd math4610
     mkdir src
     cd src

Once these commands have created the proper structure, you can create files with
code we need.

There are four codes that we can put into a shared libraries. These are

* smaceps - single precision machine epsilon computation

* dmaceps - double precision machine epsilon computation

* errabs - compute the absolute error in approximation of one number by another

* errrel - compute the relative error in approximation of one number by another

In the end, this may seem like a lot of work. But imagine having 200 routines
that will be used over and over. Then this gets a bit more interesting. 
%%%%%%%%%%%%%%%%%%%%%%%%%%%%%%%%%%%%%%%%%%%%%%%%%%%%%%%%%%%%%%%%%%%%%%%%%%%%%%%%
%%%%%%%%%%%%%%%%%%%%%%%%%%%%%%%%%%%%%%%%%%%%%%%%%%%%%%%%%%%%%%%%%%%%%%%%%%%%%%%%
\vskip0.1in\hrule\vskip0.1in
{\large{\bf The Codes We Have:}}
\vskip0.1in\hrule\vskip0.1in
\noindent
The first if we write the code in Fortran it might look like the following:
\begin{verbatim}
     c
     c -------------------------------------------------------------------------
     c
     c Author: Joe Koebbe
     c         Department of Mathematics and Statistics
     c         Utah State University
     c
     c Subroutine Name:  abserr
     c
     c Description: This code returns the distance or absolute error between
     c              an exact value and an approximate value
     c
     c Input:
     c
     c       x - the approximate value
     c       xe - the exact value
     c
     c Output:
     c
     c       eabs - the return value containing the absolute error of the two
     c              input values
     c
     c -------------------------------------------------------------------------
     c
           program old_main
     c
     c set up the problem
     c ------------------
     c
           real seps = 1.0
           integer ipow = 0
           call smaceps(seps, ipow)
           tval = 1.0 / 2.0**(ipow+1)
           print *, "Output from smaceps:"
           print *, ""
           print *, tval, ipow
     c
     c stop
     c ----
     c
           stop
           end
     c
     c -------------------------------------------------------------------------
     c -------------------------------------------------------------------------
     c
           subroutine smaceps(seps, ipow)
     c
     c intialize some variables for the iteration
     c -------------------------------------------
     c
           one = 1.0
           eps = 1.0
     c
     c ipow keeps track of the power of two that is determined by machine
     c precision
     c ---------
     c
           ipow = -1
     c
     c set a loop that will kick out when precision is met
     c ----------------------------------------------------
     c
           do 1 i=1,1000
     c
     c add a perturbation to 1
     c -----------------------
     c
              appone = one + eps
     c
     c call the routine for absolute error
     c -----------------------------------
     c
              call abserr(diff, appone, one)
     c
     c check the difference to see if the two are the same to machine
     c precision
     c ---------
     c
              if(diff .eq. 0.0) return
     c
              seps = diff
     c
     c update the power and the value of the perturbation
     c --------------------------------------------------
     c
              eps = 0.5 * eps
              ipow = ipow + 1
     c
         1 continue
     c
     c all done
     c --------
     c
           return
           end
     c
     c -------------------------------------------------------------------------
     c -------------------------------------------------------------------------
     c
           subroutine abserr(eabs, x, xe)
     c
     c just take the absolute value of the two numbers and set the output
     c for the routine
     c ---------------
     c
           eabs = abs( x - xe )
     c
     c done
     c ----
     c
           return
           end
\end{verbatim}
\vskip0.1in\hrule\vskip0.1in
\noindent
If you compile the code in the block above as we have alreay done in the class
it will produce the desired result. So, at the command prompt, typing
\begin{verbatim}

     % gfortran -o old_main old_main.f

\end{verbatim}
will produce the executable {\bf a.exe} and typing the command
\begin{verbatim}

     % ./a.exe

\end{verbatim}
will produce output that will give the number of powers of 2 and the machine
precision value.
\begin{verbatim}

     output from smaceps:
     5.96046448E-08          23

\end{verbatim}
which means single precision will give about 8 decimal digits of accuracy.
\vskip0.1in\hrule\vskip0.1in
{\large{\bf
   Creating Pieces and Stuffing the Object Modules into a Shared Library
}}
\vskip0.1in\hrule\vskip0.1in
The first step is to pull off the two subroutines in the code in the last
section. So, there are two codes to include:
\vskip0.1in\hrule\vskip0.1in
\begin{verbatim}
           subroutine smaceps(seps, ipow)
     c
     c intialize some variables for the iteration
     c -------------------------------------------
     c
           one = 1.0
           eps = 1.0
     c
     c ipow keeps track of the power of two that is determined by machine
     c precision
     c ---------
     c
           ipow = -1
     c
     c set a loop that will kick out when precision is met
     c ----------------------------------------------------
     c
           do 1 i=1,1000
     c
     c add a perturbation to 1
     c -----------------------
     c
              appone = one + eps
     c
     c call the routine for absolute error
     c -----------------------------------
     c
              call abserr(diff, appone, one)
     c
     c check the difference to see if the two are the same to machine
     c precision
     c ---------
     c
              if(diff .eq. 0.0) return
     c
              seps = diff
     c
     c update the power and the value of the perturbation
     c --------------------------------------------------
     c
              eps = 0.5 * eps
              ipow = ipow + 1
     c
         1 continue
     c
     c all done
     c --------
     c
           return
           end
\end{verbatim}
\vskip0.1in\hrule\vskip0.1in
\noindent
and
\vskip0.1in\hrule\vskip0.1in
\begin{verbatim}
           subroutine abserr(eabs, x, xe)
     c
     c just take the absolute value of the two numbers and set the output
     c for the routine
     c ---------------
     c
           eabs = abs( x - xe )
     c
     c done
     c ----
     c
           return
           end
\end{verbatim}
\vskip0.1in\hrule\vskip0.1in
Also, we will need to split off the main program that should look like the
following. 
\vskip0.1in\hrule\vskip0.1in
\begin{verbatim}
     c
     c -------------------------------------------------------------------------
     c
     c Author: Joe Koebbe
     c         Department of Mathematics and Statistics
     c         Utah State University
     c
     c Subroutine Name:  abserr
     c
     c Description: This code returns the distance or absolute error between
     c              an exact value and an approximate value
     c
     c Input:
     c
     c       x - the approximate value
     c       xe - the exact value
     c
     c Output:
     c
     c       eabs - the return value containing the absolute error of the two
     c              input values
     c
     c -------------------------------------------------------------------------
     c
           program old_main
     c
     c set up the problem
     c ------------------
     c
           real seps = 1.0
           integer ipow = 0
           call smaceps(seps, ipow)
           tval = 1.0 / 2.0**(ipow+1)
           print *, "output from smaceps:"
           print *, tval, ipow
     c
     c stop
     c ----
     c
           stop
           end
\end{verbatim}
\vskip0.1in\hrule\vskip0.1in
%%%%%%%%%%%%%%%%%%%%%%%%%%%%%%%%%%%%%%%%%%%%%%%%%%%%%%%%%%%%%%%%%%%%%%%%%%%%%%%%
%%%%%%%%%%%%%%%%%%%%%%%%%%%%%%%%%%%%%%%%%%%%%%%%%%%%%%%%%%%%%%%%%%%%%%%%%%%%%%%%
\vskip0.1in\hrule\vskip0.1in
\noindent
In this example, there will be four files:
\vskip0.1in\hrule\vskip0.1in
\begin{list}{$\bullet$}{\usecounter{beans} \parsep=0pt \listparindent=0pt
\topsep=0pt \rightmargin=.35in \leftmargin=.35in \labelsep=5 pt
\itemsep=2pt}
  \item old\_main.f - all of the code is in one file
  \item main.f - just the main part of the code
  \item smaceps.f - this file contains the code to compute the machine epsilon
  \item abserr.f - the routine that computes the absolute error
\end{list}
\vskip0.1in\hrule\vskip0.1in
%%%%%%%%%%%%%%%%%%%%%%%%%%%%%%%%%%%%%%%%%%%%%%%%%%%%%%%%%%%%%%%%%%%%%%%%%%%%%%%%
%%%%%%%%%%%%%%%%%%%%%%%%%%%%%%%%%%%%%%%%%%%%%%%%%%%%%%%%%%%%%%%%%%%%%%%%%%%%%%%%
{\large{\bf Craating the Object Modules for the Routines}}
\vskip0.1in\hrule\vskip0.1in
\noindent
Use the commands:
\vskip0.1in\hrule\vskip0.1in
\begin{verbatim}

     % gfortran -c abserr.f
     % gfortran -c smaceps.f

\end{verbatim}
\vskip0.1in\hrule\vskip0.1in
\noindent
to create the two files **abserr.o** and **smaceps.o**. The next step will
create the library. Use
\vskip0.1in\hrule\vskip0.1in
\begin{verbatim}

     % ar rcv mylib.a *.o

\end{verbatim}
\vskip0.1in\hrule\vskip0.1in
\noindent
Note that the asterisk is a wildcard character. The output will list the files
as they are added. That is
\vskip0.1in\hrule\vskip0.1in
\begin{verbatim}

     a - abserr.o
     a - smaceps.o

\end{verbatim}
\vskip0.1in\hrule\vskip0.1in
\noindent
If you look at what is in the folder (using ls) the files will make sense. One
last thing needs to be done in the library. The command is
\vskip0.1in\hrule\vskip0.1in
\begin{verbatim}

     ranlib mylib.a

\end{verbatim}
\vskip0.1in\hrule\vskip0.1in
\noindent
which creates a random access library. This basically allows programs to link
to the library more efficiently.
\vskip0.1in\hrule\vskip0.1in
{{\large{\bf Testing the Library:}}
\vskip0.1in\hrule\vskip0.1in
\noindent
To test the work, we need to compile and link the main program file without the
subroutines contained. This can be done using the following command
\vskip0.1in\hrule\vskip0.1in
\begin{verbatim}

     % gfortran main.f mylib.a

\end{verbatim}
\vskip0.1in\hrule\vskip0.1in
\noindent
The result is another version of the {\bf a.exe} file than can be executed as
above.
\vskip0.1in\hrule\vskip0.1in
{\large{\bf What is Next?}}
\vskip0.1in\hrule\vskip0.1in
\noindent
At this point, you can add more object models to your library with the idea of
creating a whole suite of codes.
\vskip0.1in\hrule\vskip0.1in
%%%%%%%%%%%%%%%%%%%%%%%%%%%%%%%%%%%%%%%%%%%%%%%%%%%%%%%%%%%%%%%%%%%%%%%%%%%%%%%%
%%%%%%%%%%%%%%%%%%%%%%%%%%%%%%%%%%%%%%%%%%%%%%%%%%%%%%%%%%%%%%%%%%%%%%%%%%%%%%%%
\end{document}
