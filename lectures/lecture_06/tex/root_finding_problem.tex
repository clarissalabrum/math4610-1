\documentclass[10pt,fleqn]{article}
%\usepackage{graphicx}


\setlength{\topmargin}{-.75in}
\addtolength{\textheight}{2.00in}
\setlength{\oddsidemargin}{.00in}
\addtolength{\textwidth}{.75in}

\title{Math 4610 Lecture Notes \\
            \ \\
      Root Finding Problems for Real Values Function of One Variable
  \footnote{These notes are part of an Open Resource Educational project
            sponsored by Utah State University}}

\author{Joe Koebbe}

\nofiles

\pagestyle{empty}

\setlength{\parindent}{0in}

% new math commands


\setlength{\oddsidemargin}{-0.25in}
\setlength{\evensidemargin}{-0.25in}
\setlength{\textwidth}{6.75in}
\setlength{\headheight}{0.0in}
\setlength{\topmargin}{-0.25in}
\setlength{\textheight}{9.00in}

\makeindex

\usepackage{mathrsfs}

%\usepackage[pdftex]{graphicx}
\usepackage{epstopdf}

\newcounter{beans}

\newcommand{\ds}{\displaystyle}
\newcommand{\limit}[2]{\displaystyle\lim_{#1\to#2}}

\newcommand{\binomial}[2]{\ \left( \begin{array}{c}
                                  #1 \\
                                  #2
                                 \end{array}
                            \right) \
                         }
\newcommand{\ExampleRule}[2]
  {
  \noindent
  \rule{\linewidth}{1pt}
  \begin{example}
    #1
    \label{#2}
  \end{example}
  \rule{\linewidth}{1pt}
  \vskip0.125in
  }

\newcommand{\defbox}[1]
  {
   \ \\
   \noindent
   \setlength\fboxrule{1pt}
   \fbox{
        \begin{minipage}{6.5in}
          #1
        \end{minipage}
        }
   \ \\
  }
\newcommand{\verysmallworkbox}[1]
  {
   \ \\
   \noindent
   \setlength\fboxrule{1pt}
   \fbox{
        \begin{minipage}{6.5in}
           #1
           \ \\
           \vskip0.5in \ \\
           \ \\
        \end{minipage}
        }
   \ \\
  }
\newcommand{\smallworkbox}[1]
  {
   \ \\
   \noindent
   \setlength\fboxrule{1pt}
   \fbox{
        \begin{minipage}{6.5in}
           #1
           \ \\
           \vskip2.5in \ \\
           \ \\
        \end{minipage}
        }
   \ \\
  }
\newcommand{\halfworkbox}[1]
  {
   \ \\
   \noindent
   \setlength\fboxrule{1pt}
   \fbox{
        \begin{minipage}{6.5in}
           #1 \hfill
           \ \\
           \vskip3.25in \ \\
           \ \\
        \end{minipage}
        }
   \ \\
  }
\newcommand{\largeworkbox}[1]
  {
   \ \\
   \noindent
   \setlength\fboxrule{1pt}
   \fbox{
        \begin{minipage}{6.5in}
           #1
           \ \\
           \vskip7.5in \ \\
           \ \\
        \end{minipage}
        }
   \ \\
  }
\newcommand{\flexworkbox}[2]
  {
   \ \\
   \noindent
   \setlength\fboxrule{1pt}
   \fbox{
        \begin{minipage}{6.5in}
           #1
           \ \\

           \vskip#2 \ \\
           \ \\
        \end{minipage}
        }
   \ \\
  }


% symbols for sets of numbers

\newcommand{\natnumb}{$\cal N$}
\newcommand{\whonumb}{$\cal W$}
\newcommand{\intnumb}{$\cal Z$}
\newcommand{\ratnumb}{$\cal Q$}
\newcommand{\irrnumb}{$\cal I$}
\newcommand{\realnumb}{$\cal R$}
\newcommand{\cmplxnumb}{$\cal C$}

% misc. commands

\newcommand{\mma}{{\it Mathematica}}
\newcommand{\sech}{\mbox{ sech}}
 
\newtheorem{theorem}{Theorem}
\newtheorem{example}{Example}
\newtheorem{definition}{Definition}
\newtheorem{problem}{Problem}

\setcounter{secnumdepth}{2}
\setcounter{tocdepth}{4}


\begin{document}
\maketitle
\newpage
%%%%%%%%%%%%%%%%%%%%%%%%%%%%%%%%%%%%%%%%%%%%%%%%%%%%%%%%%%%%%%%%%%%%%%%%%%%%%%%%
%%%%%%%%%%%%%%%%%%%%%%%%%%%%%%%%%%%%%%%%%%%%%%%%%%%%%%%%%%%%%%%%%%%%%%%%%%%%%%%%
\vskip0.1in\hrule\vskip0.1in
\noindent
{\bf Root Finding Problem: Definition of the Problem} 
\vskip0.1in\hrule\vskip0.1in
\noindent
Many problems can be recast in the form of finding places where a function is
zero. In a standard first semester calculus course find extreme values of a
function of one variable amounts to determining locations where the derivative
of the function is zero. That is, a necessary condition for the existence of a
local minimum or local maximum at a point $x^8$ is that the derivative is zero
at $x^*$ or
$$
  f'(x_0)=0.
$$
More often than not, we will need to deal with roots that are not exact in
machine precision. For example, finding the roots of
$$
  sin(x)=0
$$
is easy from an analytic point of view, the zeros are $x_n=n\ \pi$ where $n$ can
be any integer. If $n$ is not equal to zero, the root is an irrational number
and cannot be represented exactly. So, we will need to settle for an
approximation.
%%%%%%%%%%%%%%%%%%%%%%%%%%%%%%%%%%%%%%%%%%%%%%%%%%%%%%%%%%%%%%%%%%%%%%%%%%%%%%%%
%%%%%%%%%%%%%%%%%%%%%%%%%%%%%%%%%%%%%%%%%%%%%%%%%%%%%%%%%%%%%%%%%%%%%%%%%%%%%%%%
\vskip0.1in\hrule\vskip0.1in
\noindent
The general root finding problem can be written as follows: For a given
real-valued function, $f$, of a single real variable find a real number, $x^*$,
such that
$$
  f(x^*) = 0
$$
There are all kinds of issues that arise in solving these types of problems.
For example, the function may have multiple roots. In searching for a specific
root, we may find other roots that are not of interest. To deal with all of the
issues in this problem, we will develop a number of algorithms that can be used
in a variety of root finding problems.
%%%%%%%%%%%%%%%%%%%%%%%%%%%%%%%%%%%%%%%%%%%%%%%%%%%%%%%%%%%%%%%%%%%%%%%%%%%%%%%%
%%%%%%%%%%%%%%%%%%%%%%%%%%%%%%%%%%%%%%%%%%%%%%%%%%%%%%%%%%%%%%%%%%%%%%%%%%%%%%%%
\vskip0.1in\hrule\vskip0.1in
\noindent
{\bf Root Finding Problems: Using Fixed Point Iteration} 
\vskip0.1in\hrule\vskip0.1in
\noindent
As a first attempt at determining the location of a root for a function, we
might consider a modification of the root finding problem as follows. Given a
function, $f$, we can rewrite the equation
$$
  f(x^*) = 0
$$
as
$$
  x = x - f(x^*) = g(x^*)
$$
The resulting equation is called a fixed point equation and the equation
suggests an iteration of the form
$$
  x_1 = g(x_0), x_2 = f(x_1), \ldots
$$
where $x_0$ must be supplied to start the iteration. This iteration formula will
produce a sequence of real numbers. The hope is that the sequence will converge
to a solution of the fixed point equation and also a solution of the root
finding problem.
%%%%%%%%%%%%%%%%%%%%%%%%%%%%%%%%%%%%%%%%%%%%%%%%%%%%%%%%%%%%%%%%%%%%%%%%%%%%%%%%
%%%%%%%%%%%%%%%%%%%%%%%%%%%%%%%%%%%%%%%%%%%%%%%%%%%%%%%%%%%%%%%%%%%%%%%%%%%%%%%%
\vskip0.1in\hrule\vskip0.1in
\noindent
{\bf Root Finding Problems: Coding Fixed Point Iteration} 
\vskip0.1in\hrule\vskip0.1in
\noindent
One can easily write a routine or computer code that implements fixed point
iteration. The following code provides a template of how a reusable routine
might be written:
\vskip0.1in\hrule\vskip0.1in
\begin{verbatim}
     //
     // Author: Joe Koebbe
     //
     // Routine Name:         fproot
     // Programming Language: Java
     // Last Modified:        09/10/19
     //
     // Description/Purpose: The routine will generate a sequence of numbers
     // using fixed point iteration.
     //
     // Input:
     //
     // FunctionObject f - the function defined in the root finding problem
     // double x0 - the initial guess at the location of a fixed point
     // double tol - the error tolerance allowed in the approximation of the
     //              root finding problem
     // int maxit - the maximum number of iterations allowed in the fixed point
     //             iteration.
     //
     // Output:
     //
     // double x1 - the last number in the finite sequence that is an
     //             approximation in the root finding problem
     //
     public double fproot(FunctionObject f, double x0, double tol, int maxit) {
       //
       // initialize the error in the routine so that the iteration loop will be
       // executed at least one time
       // --------------------------
       //
       double error = 10.0 * tol;
       //
       // initialize a counter for the number of iterations
       // -------------------------------------------------
       //
       int iter = 0;
       //
       // loop over the fixed point iterations as long as the error is larger
       // than the tolerance and the number of iterations is less than the
       // maximum number allowed
       // ----------------------
       //
       while(error > tol && iter < maxit) {
         //
         // update the number of iterations performed
         // -----------------------------------------
         //
         iter++;
         //
         // compute the next approximation
         // ------------------------------
         //
         double x1 = x0 - f(x0);
         //
         // compute the error using the difference between the iterates in the
         // loop
         // ----
         //
         error = Math.abs(x1 - x0);
         //
         // reset the input value to be the new approximation
         // --------------------------------------------------
         //
         x0 = x1;
         //
       }
       //
       // return the last value computed
       // -------------------------------
       //
       return x1;
       //
     }

\end{verbatim}
\vskip0.1in\hrule\vskip0.1in
\noindent
There are a couple of features in the code that need to be explained.
\begin{list}{$\bullet$}{\usecounter{beans} \parsep=0pt \listparindent=0pt
\topsep=0pt \rightmargin=.35in \leftmargin=.35in \labelsep=5 pt
\itemsep=2pt}
  \item To make this work in the Java programming language, the method would
        need to be embedded in a class. That is, the code is not a standalone
        code.
  \item The first argument is an Java Object that needs to be created. The
        object is used to provide the function evaluation for any real input.
  \item The second argument is the initial guess at the solution of the problem.
  \item Since we know we are going to end up with at best an approximation of
        a root, the third argument in the function is an error tolerance that is
        acceptable to the calling routine.
  \item The final argument passed in limits the number of iterations allowed in
        the method. Note that if you are not careful, an infinite loop might be
        created due to the approximations used everywhere.
\end{list}
If we apply the code to any problem, we are assuming that the solution will pop
out the end. There is no guarantee that this is the case. It is important to
establish conditions that will guarantee the code will produce an approximate
solution of the fixed point problem and thus provide a root for the original
function, $f$.
%%%%%%%%%%%%%%%%%%%%%%%%%%%%%%%%%%%%%%%%%%%%%%%%%%%%%%%%%%%%%%%%%%%%%%%%%%%%%%%%
%%%%%%%%%%%%%%%%%%%%%%%%%%%%%%%%%%%%%%%%%%%%%%%%%%%%%%%%%%%%%%%%%%%%%%%%%%%%%%%%
\vskip0.1in\hrule\vskip0.1in
\noindent
{\bf Root Finding Problems: Analysis of Functional Iteration Using Taylor
 Series Expansion} 
\vskip0.1in\hrule\vskip0.1in
\noindent
The general iteration formula, given $x_0$, is the following.
$$
  x_{k+1} = g(x_k)
$$
for $k=0,1,2,\ldots$. We also know that for the fixed point problem, the
solution satisfies the equation
$$
  x^* = g(x^*)
$$
Subtracting the two equations gives
$$
  x_{k+1} - x^* = g(x_k) - g(x^*)
$$
The Taylor expansion of $g(x_k)$ about the solution $x^*$ is given by
$$
  g(x_k) = g(x*) + g'(x^*) ( x_k - x^* ) + {1\over 2} g''(x^*) ( x_k - x^* )^2
              + \ldots
$$
Substituting the expansion into the equation above and truncating the series
gives
$$
  x_{k+1} - x^* \approx g(x*) + g'(x^*) ( x_k - x^* ) - g(x^*)
                       = g'(x^*) ( x_k - x^* )
$$
Taking absolute values the last equation can be written as
$$
  | x_{k+1} - x^* | \leq | g'(x^*) | | x_k - x^* |
$$
One can read the previous expression as the difference (or error) in $x_{k+1}$
is less than the magnitude of the derivative of the fixed point iteration 
function, $g$, times the difference (or error) in the previous approximation,
$x_k$. Using
$$
  e_{k} = | x_k - x^* |
$$
allows use to relate the error at successive steps as
$$
  e_{k+1} \leq | g'(x^*) | \cdot | e_{k+1} |
$$
To get convergence to the fixed point (or root) we would like the error to be
reduced at each step. This requires the condition
$$
  | g'(x^*) | < 1
$$
For the general fixed point problem, this condition is required for convergence
to the fixed point, $x^*$, or solution of the root finding problem. Note that
this is a significant drawback of fixed point iteration as a means of solving
root finding problems.
%%%%%%%%%%%%%%%%%%%%%%%%%%%%%%%%%%%%%%%%%%%%%%%%%%%%%%%%%%%%%%%%%%%%%%%%%%%%%%%%
%%%%%%%%%%%%%%%%%%%%%%%%%%%%%%%%%%%%%%%%%%%%%%%%%%%%%%%%%%%%%%%%%%%%%%%%%%%%%%%%
\vskip0.1in\hrule\vskip0.1in
\noindent
{\bf Root Finding Problems: An Example Using Functional Iteration} 
\vskip0.1in\hrule\vskip0.1in
\noindent
Suppose that we are interested in computing the roots of
$$
  f(x) = e^x - \pi
$$
Analytically we can compute the solution by solving for $x$ in the equation
$$
  e^x - \pi = 0
$$
The value is $x=ln(\pi)\approx 1.144729886$. This is a very simple problem.
However, it is always a good idea to test general methods on simple problems
while developing algorithms and coding these up for use on real problems.

Let's apply functional iteration to this root finding problem. First, we will
need to create an associated function that defines a fixed point problem. One
possibility is to choose
$$
  g(x) = x - f(x) = x - ( e^x - \pi ) = x - e^x + \pi
$$
%%%%%%%%%%%%%%%%%%%%%%%%%%%%%%%%%%%%%%%%%%%%%%%%%%%%%%%%%%%%%%%%%%%%%%%%%%%%%%%%
%%%%%%%%%%%%%%%%%%%%%%%%%%%%%%%%%%%%%%%%%%%%%%%%%%%%%%%%%%%%%%%%%%%%%%%%%%%%%%%%

%%%%%%%%%%%%%%%%%%%%%%%%%%%%%%%%%%%%%%%%%%%%%%%%%%%%%%%%%%%%%%%%%%%%%%%%%%%%%%%%
%%%%%%%%%%%%%%%%%%%%%%%%%%%%%%%%%%%%%%%%%%%%%%%%%%%%%%%%%%%%%%%%%%%%%%%%%%%%%%%%
\end{document}
