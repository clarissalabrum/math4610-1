\documentclass[10pt,fleqn]{article}

\setlength{\topmargin}{-.75in}
\addtolength{\textheight}{2.00in}
\setlength{\oddsidemargin}{.00in}
\addtolength{\textwidth}{.75in}

\nofiles

\pagestyle{empty}

\setlength{\parindent}{0in}

% new math commands


\setlength{\oddsidemargin}{-0.25in}
\setlength{\evensidemargin}{-0.25in}
\setlength{\textwidth}{6.75in}
\setlength{\headheight}{0.0in}
\setlength{\topmargin}{-0.25in}
\setlength{\textheight}{9.00in}

\makeindex

\usepackage{mathrsfs}

%\usepackage[pdftex]{graphicx}
\usepackage{epstopdf}

\newcounter{beans}

\newcommand{\ds}{\displaystyle}
\newcommand{\limit}[2]{\displaystyle\lim_{#1\to#2}}

\newcommand{\binomial}[2]{\ \left( \begin{array}{c}
                                  #1 \\
                                  #2
                                 \end{array}
                            \right) \
                         }
\newcommand{\ExampleRule}[2]
  {
  \noindent
  \rule{\linewidth}{1pt}
  \begin{example}
    #1
    \label{#2}
  \end{example}
  \rule{\linewidth}{1pt}
  \vskip0.125in
  }

\newcommand{\defbox}[1]
  {
   \ \\
   \noindent
   \setlength\fboxrule{1pt}
   \fbox{
        \begin{minipage}{6.5in}
          #1
        \end{minipage}
        }
   \ \\
  }
\newcommand{\verysmallworkbox}[1]
  {
   \ \\
   \noindent
   \setlength\fboxrule{1pt}
   \fbox{
        \begin{minipage}{6.5in}
           #1
           \ \\
           \vskip0.5in \ \\
           \ \\
        \end{minipage}
        }
   \ \\
  }
\newcommand{\smallworkbox}[1]
  {
   \ \\
   \noindent
   \setlength\fboxrule{1pt}
   \fbox{
        \begin{minipage}{6.5in}
           #1
           \ \\
           \vskip2.5in \ \\
           \ \\
        \end{minipage}
        }
   \ \\
  }
\newcommand{\halfworkbox}[1]
  {
   \ \\
   \noindent
   \setlength\fboxrule{1pt}
   \fbox{
        \begin{minipage}{6.5in}
           #1 \hfill
           \ \\
           \vskip3.25in \ \\
           \ \\
        \end{minipage}
        }
   \ \\
  }
\newcommand{\largeworkbox}[1]
  {
   \ \\
   \noindent
   \setlength\fboxrule{1pt}
   \fbox{
        \begin{minipage}{6.5in}
           #1
           \ \\
           \vskip7.5in \ \\
           \ \\
        \end{minipage}
        }
   \ \\
  }
\newcommand{\flexworkbox}[2]
  {
   \ \\
   \noindent
   \setlength\fboxrule{1pt}
   \fbox{
        \begin{minipage}{6.5in}
           #1
           \ \\

           \vskip#2 \ \\
           \ \\
        \end{minipage}
        }
   \ \\
  }


% symbols for sets of numbers

\newcommand{\natnumb}{$\cal N$}
\newcommand{\whonumb}{$\cal W$}
\newcommand{\intnumb}{$\cal Z$}
\newcommand{\ratnumb}{$\cal Q$}
\newcommand{\irrnumb}{$\cal I$}
\newcommand{\realnumb}{$\cal R$}
\newcommand{\cmplxnumb}{$\cal C$}

% misc. commands

\newcommand{\mma}{{\it Mathematica}}
\newcommand{\sech}{\mbox{ sech}}
 
\newtheorem{theorem}{Theorem}
\newtheorem{example}{Example}
\newtheorem{definition}{Definition}
\newtheorem{problem}{Problem}

\setcounter{secnumdepth}{2}
\setcounter{tocdepth}{4}


\begin{document}
%%%%%%%%%%%%%%%%%%%%%%%%%%%%%%%%%%%%%%%%%%%%%%%%%%%%%%%%%%%%%%%%%%%%%%%%%%%%%%%%
%%%%%%%%%%%%%%%%%%%%%%%%%%%%%%%%%%%%%%%%%%%%%%%%%%%%%%%%%%%%%%%%%%%%%%%%%%%%%%%%
\vskip0.1in\hrule\vskip0.1in
\noindent
{\bf Description of Material Presented in Lectures:}
\vskip0.1in\hrule\vskip0.1in
\noindent
The following description relates the algorithms presented in Math 4610 to a
variety of computational skills that will be discussed during lectures. The
course will attempt to illustrate the use of High Performance Computation (HPC)
on specific algorithms. The idea is to introduce students to computational
techniques that can be used to increase the performance of computer codes that
implement numerical and computational algorithms.

\vskip0.1in\hrule\vskip0.1in
%%%%%%%%%%%%%%%%%%%%%%%%%%%%%%%%%%%%%%%%%%%%%%%%%%%%%%%%%%%%%%%%%%%%%%%%%%%%%%%%
%%%%%%%%%%%%%%%%%%%%%%%%%%%%%%%%%%%%%%%%%%%%%%%%%%%%%%%%%%%%%%%%%%%%%%%%%%%%%%%%
\noindent
\begin{enumerate}
  \item {\bf Lecture: First Day} 
    \begin{list}{$\bullet$}{\usecounter{beans} \parsep=0pt \listparindent=0pt
    \topsep=0pt \rightmargin=.35in \leftmargin=.35in \labelsep=5 pt
    \itemsep=2pt}
      \item {\bf Linear Systems of Equations with OpenMP:} OpenMP provides
            inline directives, a library of intrinsic functions, and global
            variables that can be used by compilers to optimize executable
            performance. 
      \item {\bf The Bisection Method with Recursion:} Every student of
            computational mathematics should know how to use recursion
            effectively. Some programming languages allow the use of recursive
            constructs.
      \item {\bf Newton's Method and Interval Analysis:} Interval analysis
            provides a means to put bounds on the accuracy of some approximation
            methods. Note that there are a number of people trying to use these
            ideas to provide computer aided proofs in mathematics.
      \item {\bf Bracketing Roots with Parallel Methods:} 
      \item {\bf Monte Carlo Integration:} Using GPUs to ... 
      \item {\bf Numerical Integration:} Use properties of the integral to
            divide and conquer.
      \item {\bf Composite Quadrature Rules: The Trapezoid and Simpson's Rules}
            Rewriting formulas to determine equivalent approximations to
            increase efficiency of the algorithm.
      \item {\bf Power Method for Estimating Largest Eigenvalue:} Using parallel
            algorithms to increase code performance.
      \item {\bf Linear Regression for Overdetermined Systems:} Normal equations
            approach.
      \item {\bf Fast Fourier Transform (FFT):} Matrix multiplication versus
            the FFT.
      \item {\bf Cholesky Factorization:} for symmetric, positive definite (spd)
            linear systems.
      \item {\bf QR-factorization:} matrices for the solution of linear systems
            of equations.
      \item {\bf Estimating the Condition Number of a Matrix}
      \item {\bf Euler's method for solving Initial Value Problems.}
    \end{list}
\end{enumerate}
%%%%%%%%%%%%%%%%%%%%%%%%%%%%%%%%%%%%%%%%%%%%%%%%%%%%%%%%%%%%%%%%%%%%%%%%%%%%%%%%
%%%%%%%%%%%%%%%%%%%%%%%%%%%%%%%%%%%%%%%%%%%%%%%%%%%%%%%%%%%%%%%%%%%%%%%%%%%%%%%%
\end{document}
