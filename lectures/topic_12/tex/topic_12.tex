\documentclass[10pt,fleqn]{article}
\usepackage{hyperref}
\usepackage{graphicx}


\setlength{\topmargin}{-.75in}
\addtolength{\textheight}{2.00in}
\setlength{\oddsidemargin}{.00in}
\addtolength{\textwidth}{.75in}

\nofiles

\pagestyle{empty}

\setlength{\parindent}{0in}

% new math commands


\setlength{\oddsidemargin}{-0.25in}
\setlength{\evensidemargin}{-0.25in}
\setlength{\textwidth}{6.75in}
\setlength{\headheight}{0.0in}
\setlength{\topmargin}{-0.25in}
\setlength{\textheight}{9.00in}

\makeindex

\usepackage{mathrsfs}

%\usepackage[pdftex]{graphicx}
\usepackage{epstopdf}

\newcounter{beans}

\newcommand{\ds}{\displaystyle}
\newcommand{\limit}[2]{\displaystyle\lim_{#1\to#2}}

\newcommand{\binomial}[2]{\ \left( \begin{array}{c}
                                  #1 \\
                                  #2
                                 \end{array}
                            \right) \
                         }
\newcommand{\ExampleRule}[2]
  {
  \noindent
  \rule{\linewidth}{1pt}
  \begin{example}
    #1
    \label{#2}
  \end{example}
  \rule{\linewidth}{1pt}
  \vskip0.125in
  }

\newcommand{\defbox}[1]
  {
   \ \\
   \noindent
   \setlength\fboxrule{1pt}
   \fbox{
        \begin{minipage}{6.5in}
          #1
        \end{minipage}
        }
   \ \\
  }
\newcommand{\verysmallworkbox}[1]
  {
   \ \\
   \noindent
   \setlength\fboxrule{1pt}
   \fbox{
        \begin{minipage}{6.5in}
           #1
           \ \\
           \vskip0.5in \ \\
           \ \\
        \end{minipage}
        }
   \ \\
  }
\newcommand{\smallworkbox}[1]
  {
   \ \\
   \noindent
   \setlength\fboxrule{1pt}
   \fbox{
        \begin{minipage}{6.5in}
           #1
           \ \\
           \vskip2.5in \ \\
           \ \\
        \end{minipage}
        }
   \ \\
  }
\newcommand{\halfworkbox}[1]
  {
   \ \\
   \noindent
   \setlength\fboxrule{1pt}
   \fbox{
        \begin{minipage}{6.5in}
           #1 \hfill
           \ \\
           \vskip3.25in \ \\
           \ \\
        \end{minipage}
        }
   \ \\
  }
\newcommand{\largeworkbox}[1]
  {
   \ \\
   \noindent
   \setlength\fboxrule{1pt}
   \fbox{
        \begin{minipage}{6.5in}
           #1
           \ \\
           \vskip7.5in \ \\
           \ \\
        \end{minipage}
        }
   \ \\
  }
\newcommand{\flexworkbox}[2]
  {
   \ \\
   \noindent
   \setlength\fboxrule{1pt}
   \fbox{
        \begin{minipage}{6.5in}
           #1
           \ \\

           \vskip#2 \ \\
           \ \\
        \end{minipage}
        }
   \ \\
  }


% symbols for sets of numbers

\newcommand{\natnumb}{$\cal N$}
\newcommand{\whonumb}{$\cal W$}
\newcommand{\intnumb}{$\cal Z$}
\newcommand{\ratnumb}{$\cal Q$}
\newcommand{\irrnumb}{$\cal I$}
\newcommand{\realnumb}{$\cal R$}
\newcommand{\cmplxnumb}{$\cal C$}

% misc. commands

\newcommand{\mma}{{\it Mathematica}}
\newcommand{\sech}{\mbox{ sech}}
 
\newtheorem{theorem}{Theorem}
\newtheorem{example}{Example}
\newtheorem{definition}{Definition}
\newtheorem{problem}{Problem}

\setcounter{secnumdepth}{2}
\setcounter{tocdepth}{4}


\begin{document}
%%%%%%%%%%%%%%%%%%%%%%%%%%%%%%%%%%%%%%%%%%%%%%%%%%%%%%%%%%%%%%%%%%%%%%%%%%%%%%%%
%%%%%%%%%%%%%%%%%%%%%%%%%%%%%%%%%%%%%%%%%%%%%%%%%%%%%%%%%%%%%%%%%%%%%%%%%%%%%%%%
\vskip0.1in\hrule\vskip0.1in \noindent
{\bf Math 4610 Fundamentals of Computational Mathematics  - Topic 12.}
\vskip0.1in\hrule\vskip0.1in \noindent
Most mathematical problems require the use of approximations as a part of the
solution process. For example, the solution of the simple ordinary differential
equation
\[
  {{dy}\over{dt}} = -2\ y
\]
with initial condition, \(y(0)=1\), is the function
\[
  y(t) = e^{-2t}
\]
You can check this on your own. This is a single term that seems nice and tidy
for predicting value of the solution to the differential equation. However, due
to the fact that the number \(e=2.71828...\) is an irrational number any
evaluation will be an approximation at best. For certain values like \(t=0\) the
exponential can be evaluated exactly. However, for just about any other number
all we can do is approximate the value based on our mathematica knowledge.

From any standard second semester engineering calculus course, a series 
representation of the exponential function is given by
\[
  e^x = 1 + x + {1\over 2} x^2 + {1\over 6} x^3 + \cdots
      = \sum_{k=0}^\infty {{x^k}\over{k!}}
\]
Of course, this is an infinite series and the best we can do is sum a finite
number of terms and neglect/truncate the infinite series.

Mathematically, we know that the series converges rapidly to an output value
given any reasonable value. It may take a few (or a lot of) terms, but a
truncated series
\[
  e^x \approx \sum_{k=0}^N {{x^k}\over{k!}}
\]
for some, \(N>0\), can be used to approximate the exponential function. In this
example, there are a couple of problems that still need to be addressed. First,
if we truncate the infinite sum to a finite value, \(N>0\), how good is the
approximation? As we will see, the truncation of the series can be analyzed
mathematically using Taylor series.

The second problem involves errors in finite number representation on anu
computer. That is, due to the finite resources (memory/disk space)available on
a computer, we will run into problems for either very small values or very large
input values to the exponential function.

The first problem is due to truncation error which is an artifact of
approximating the mathematical model. Truncation error in any given problem 
needs to be analyzed mathematically. The second problem is really beyond our
control since it is due to the particular computer and operating system we are
using. We will treat the problem of truncation error in this topic and save the
problem of round off error and machine precision for another topic in the near
future.
%%%%%%%%%%%%%%%%%%%%%%%%%%%%%%%%%%%%%%%%%%%%%%%%%%%%%%%%%%%%%%%%%%%%%%%%%%%%%%%%
%%%%%%%%%%%%%%%%%%%%%%%%%%%%%%%%%%%%%%%%%%%%%%%%%%%%%%%%%%%%%%%%%%%%%%%%%%%%%%%%
\vskip0.1in\hrule\vskip0.1in\noindent
{\bf A Brief Review of Taylor Series} 
\vskip0.1in\hrule\vskip0.1in\noindent
In just about any calculus sequence, the topic of Taylor series is discussed. A
definition for this concept is the following.
\begin{definition}
  Suppose that the function, \(f\), is a function with derivatives of all
  orders at a point, \(a\) in the domain of \(f\). Then the Taylor series of
  \(f\) about the point \(a\) is given by
  \[
    f(x) \sim f(a) + f'(a) ( x - a ) + {1\over 2} f''(a) ( x - a )^2 + \cdots
          = \sum_{k=0}^\infty\ {{f^{(k)}}\over{k!}}\ ( x - a )^k
  \]
  where \(f^{(k)}(a)\) is the \(k^{th}\) derivative of \(f\) at \(a\) and
  \(k!=k(k-1)(k-2)\ldots (2)(1)\) with the assumption that \(0!=1\).
\end{definition}
Students should be able to produce the Taylor series of simple functions like
the trigonometric functions or examples like \(f(x)=ln(1+x)\). If you are a
little foggy on the details, there are examples all over the internet or you can
refer to any book the presents topics in engineering calculus.

It is also important to know how to apply the Taylor series with remainder. The
definition we need is the following.
\begin{definition}
  Suppose that the function, \(f\), is a function with \(n+1\) continuous
  derivatives at a point, \(a\), in the domain of the function. Then, the Taylor
  series with remainder is
  \[
    f(x) \sim \sum_{k=0}^n\ {{f^{(k)}(a)}\over{k!}}\ ( x - a )^k
                   + R_f(a,n)
  \]
  where
  \[
    R_f(a,n) = {{f^{n+1}(\xi)}\over{(n+1)!}}\ ( x - a )^{n+1}
  \]
  and \(\xi\) is a point between \(x\) and \(a\).
\end{definition}
The remainder in this definition can be used to establish the truncation error
and upper bounds for the truncation error as will be seen in some examples in
this topic.
%%%%%%%%%%%%%%%%%%%%%%%%%%%%%%%%%%%%%%%%%%%%%%%%%%%%%%%%%%%%%%%%%%%%%%%%%%%%%%%%
%%%%%%%%%%%%%%%%%%%%%%%%%%%%%%%%%%%%%%%%%%%%%%%%%%%%%%%%%%%%%%%%%%%%%%%%%%%%%%%%
\vskip0.1in\hrule\vskip0.1in\noindent
{\bf \(h\) Form of the Taylor Series:} 
\vskip0.1in\hrule\vskip0.1in\noindent
Computational mathematicians should be able to use Taylor series with ease in
the analysis of numerical methods. There are several different, but equivalent
forms of the Taylor series. For the purposes in this course, we will use the
\(h\) or increment form for Taylor series expansions. To get to the appropriate
form, letting \(h=x-a\) doing some simplification gives
\[
  f(a+h) \sim f(a) + f'(a) h + {1\over 2} f''(a) h^2 + \cdots
          = \sum_{k=0}^\infty\ {{f^{(k)}(a)}\over{k!}}\ h^k
\]
or using the Taylor series with remainder
\[
    f(a+h) \sim \sum_{k=0}^n\ {{f^{(k)}(a)}\over{k!}}\ h^k
                   + R_f(a,n)
\]
where
\[
  R_f(a,n) = {{f^{n+1}(\xi)}\over{(n+1)!}}\ h^{n+1}
\]
It should
%%%%%%%%%%%%%%%%%%%%%%%%%%%%%%%%%%%%%%%%%%%%%%%%%%%%%%%%%%%%%%%%%%%%%%%%%%%%%%%%
%%%%%%%%%%%%%%%%%%%%%%%%%%%%%%%%%%%%%%%%%%%%%%%%%%%%%%%%%%%%%%%%%%%%%%%%%%%%%%%%
\vskip0.1in\hrule\vskip0.1in\noindent
{\bf } 
\vskip0.1in\hrule\vskip0.1in\noindent
To create an account on GitHub, go to the Github site on any browser. I you
already have an account, you can skip this step.
\begin{verbatim}

    https://github.com

\end{verbatim}
This site will display a place to create an account or sign in to an existin
account.
%\vskip0.1in\hrule\vskip0.1in
%\vfill
%\begin{figure}[h]
%\centering
%\includegraphics{../images/github_01.png}
%\vskip0.1in
%\caption{{Screenshot} taken using {\bf Snip \& Sketch}. This is an app on
%         my Windows 10 box}
%\end{figure}
%\eject
%%%%%%%%%%%%%%%%%%%%%%%%%%%%%%%%%%%%%%%%%%%%%%%%%%%%%%%%%%%%%%%%%%%%%%%%%%%%%%%%
%%%%%%%%%%%%%%%%%%%%%%%%%%%%%%%%%%%%%%%%%%%%%%%%%%%%%%%%%%%%%%%%%%%%%%%%%%%%%%%%
\vskip0.1in\hrule\vskip0.1in
\noindent
{{\bf GitHub Primer for Math 4610 at USU:} Setting up a Repository} 
\vskip0.1in\hrule\vskip0.1in
\noindent
Once you log in, you will need to build a repository for use in the class and
to turn in homework and completed tasks and projects. For the course, you will
create a repository named the following:
\begin{verbatim}

    math4610

\end{verbatim}
Use only the characters above and using the following rules:
\begin{enumerate}
  \item Use only lower case characters - github is case senesitive.
  \item Do not put any blanks in the name of the repository.
\end{enumerate}
Note that the instructor will use only this repository name in looking for your
work.
%%%%%%%%%%%%%%%%%%%%%%%%%%%%%%%%%%%%%%%%%%%%%%%%%%%%%%%%%%%%%%%%%%%%%%%%%%%%%%%%
%%%%%%%%%%%%%%%%%%%%%%%%%%%%%%%%%%%%%%%%%%%%%%%%%%%%%%%%%%%%%%%%%%%%%%%%%%%%%%%%
\vskip0.1in\hrule\vskip0.1in
\noindent
{{\bf Github Primer for Math 4610 at USU:} List the Contents of the Home
    Directory} 
\vskip0.1in\hrule\vskip0.1in
\noindent
If you have an account on GitHub, you will already know a lot about these
things. However, when you are logged in you will see the main screen with any
repositories you may already have created. We will go through the steps to
build and name repositories in the next few pages.

%%%%%%%%%%%%%%%%%%%%%%%%%%%%%%%%%%%%%%%%%%%%%%%%%%%%%%%%%%%%%%%%%%%%%%%%%%%%%%%%
%%%%%%%%%%%%%%%%%%%%%%%%%%%%%%%%%%%%%%%%%%%%%%%%%%%%%%%%%%%%%%%%%%%%%%%%%%%%%%%%
\end{document}
