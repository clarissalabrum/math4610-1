\documentclass[10pt,fleqn]{article}
% new math commands


\setlength{\oddsidemargin}{-0.25in}
\setlength{\evensidemargin}{-0.25in}
\setlength{\textwidth}{6.75in}
\setlength{\headheight}{0.0in}
\setlength{\topmargin}{-0.25in}
\setlength{\textheight}{9.00in}

\makeindex

\usepackage{mathrsfs}

%\usepackage[pdftex]{graphicx}
\usepackage{epstopdf}

\newcounter{beans}

\newcommand{\ds}{\displaystyle}
\newcommand{\limit}[2]{\displaystyle\lim_{#1\to#2}}

\newcommand{\binomial}[2]{\ \left( \begin{array}{c}
                                  #1 \\
                                  #2
                                 \end{array}
                            \right) \
                         }
\newcommand{\ExampleRule}[2]
  {
  \noindent
  \rule{\linewidth}{1pt}
  \begin{example}
    #1
    \label{#2}
  \end{example}
  \rule{\linewidth}{1pt}
  \vskip0.125in
  }

\newcommand{\defbox}[1]
  {
   \ \\
   \noindent
   \setlength\fboxrule{1pt}
   \fbox{
        \begin{minipage}{6.5in}
          #1
        \end{minipage}
        }
   \ \\
  }
\newcommand{\verysmallworkbox}[1]
  {
   \ \\
   \noindent
   \setlength\fboxrule{1pt}
   \fbox{
        \begin{minipage}{6.5in}
           #1
           \ \\
           \vskip0.5in \ \\
           \ \\
        \end{minipage}
        }
   \ \\
  }
\newcommand{\smallworkbox}[1]
  {
   \ \\
   \noindent
   \setlength\fboxrule{1pt}
   \fbox{
        \begin{minipage}{6.5in}
           #1
           \ \\
           \vskip2.5in \ \\
           \ \\
        \end{minipage}
        }
   \ \\
  }
\newcommand{\halfworkbox}[1]
  {
   \ \\
   \noindent
   \setlength\fboxrule{1pt}
   \fbox{
        \begin{minipage}{6.5in}
           #1 \hfill
           \ \\
           \vskip3.25in \ \\
           \ \\
        \end{minipage}
        }
   \ \\
  }
\newcommand{\largeworkbox}[1]
  {
   \ \\
   \noindent
   \setlength\fboxrule{1pt}
   \fbox{
        \begin{minipage}{6.5in}
           #1
           \ \\
           \vskip7.5in \ \\
           \ \\
        \end{minipage}
        }
   \ \\
  }
\newcommand{\flexworkbox}[2]
  {
   \ \\
   \noindent
   \setlength\fboxrule{1pt}
   \fbox{
        \begin{minipage}{6.5in}
           #1
           \ \\

           \vskip#2 \ \\
           \ \\
        \end{minipage}
        }
   \ \\
  }


% symbols for sets of numbers

\newcommand{\natnumb}{$\cal N$}
\newcommand{\whonumb}{$\cal W$}
\newcommand{\intnumb}{$\cal Z$}
\newcommand{\ratnumb}{$\cal Q$}
\newcommand{\irrnumb}{$\cal I$}
\newcommand{\realnumb}{$\cal R$}
\newcommand{\cmplxnumb}{$\cal C$}

% misc. commands

\newcommand{\mma}{{\it Mathematica}}
\newcommand{\sech}{\mbox{ sech}}
 
\newtheorem{theorem}{Theorem}
\newtheorem{example}{Example}
\newtheorem{definition}{Definition}
\newtheorem{problem}{Problem}

\setcounter{secnumdepth}{2}
\setcounter{tocdepth}{4}

\usepackage[pdftex]{graphicx}
\usepackage{epstopdf}
\begin{document}
%
% Student Syllabus File
%
\vskip0.1in\hrule\vskip0.1in \noindent
{\bf Syllabus: Math 1100 Calculus Techniques} \hfill Fall 2020
\vskip0.1in\hrule\vskip0.1in \noindent
This syllabus provides information to students about the content, format, and
grading of Math 1100 Calculus Techniques. Math 1100 is a course taken by most
undergraduate business and economics majors at Utah State University (USU). This
syllabus is intended for USU students enrolled in Math 1100 Calculus Techniques
either on the Logan campus, online, or in the China Cooperative Academic
Program (CCAP) through the Huntsman School of Business.
%
%   The following documentation is intended for facilitators in the China
% Cooperative Academic Program through Utah State University.
%
\vskip0.1in\hrule\vskip0.1in
\noindent
{\bf Instructor:} \hfill   Joe Koebbe, Associate Professor \\
\smallskip\noindent
{\bf Office:}     \hfill   ANSC 209 \\
\smallskip\noindent
{\bf Office Hours:}     \hfill   MWF 7:30am to 9:20am, MWF 10:30pm-11:20pm \\
\smallskip\noindent
{\bf Phone:}      \hfill   1-435-797-2825 \\
\smallskip\noindent
{\bf email:}      \hfill   joe.koebbe@usu.edu \\
\smallskip\noindent
{\bf webpage:}    \hfill   http://www.math.usu.edu/\~{}koebbe
%%%%%%%%%%%%%%%%%%%%%%%%%%%%%%%%%%%%%%%%%%%%%%%%%%%%%%%%%%%%%%%%%%%%%%%%%%%%%%%%
%%%%%%%%%%%%%%%%%%%%%%%%%%%%%%%%%%%%%%%%%%%%%%%%%%%%%%%%%%%%%%%%%%%%%%%%%%%%%%%%
\vskip0.1in\hrule\vskip0.1in \noindent
{\bf Textbook Information:}
\vskip0.1in\hrule\vskip0.1in
\noindent
The information for the textbook is:
\vskip0.1in\noindent
{\bf Title:} \hfill Calculus \& Its Applications \\
\smallskip\noindent
{\bf Author:} \hfill Goldstein, L., Lay, D., Schneider, D., and Asmar, N.\\
\smallskip\noindent
{\bf Edition:} \hfill Fourteenth \\
\smallskip\noindent
{\bf Publisher:} \hfill  Pearson \\
\smallskip\noindent
{\bf ISBN:} \hfill 13:978-0-134-43777-4 \\
\smallskip\noindent
{\bf Copyright:} \hfill 2018 \\
\smallskip\noindent
The book can be obtained through the USU Bookstore or on Amazon.com. The book
may be purchased or rented for a semester through Amazon.
%%%%%%%%%%%%%%%%%%%%%%%%%%%%%%%%%%%%%%%%%%%%%%%%%%%%%%%%%%%%%%%%%%%%%%%%%%%%%%%%
%%%%%%%%%%%%%%%%%%%%%%%%%%%%%%%%%%%%%%%%%%%%%%%%%%%%%%%%%%%%%%%%%%%%%%%%%%%%%%%%
\vskip0.1in\hrule\vskip0.1in \noindent
{\bf USU Course Catalog Description:}
\vskip0.1in\hrule\vskip0.1in
\noindent
The following is the general course description for Math 1100 Calculus
Techniques in the USU catalog as of Fall Semester 2020. The pertinent
information regarding content, prerequisites, and placement in this course are
given in the description. Check with the Math Advising Office for clarification
of this description.

\vskip0.1in

\noindent
{\bf MATH 1100 - Calculus Techniques} (QL Quantitative Literacy 3 credits)
Techniques of elementary calculus, differentiation, integration, elementary
optimization, and introduction to partial derivatives. Applications in business,
social science, and natural resources. Graphing calculator required.
Prerequisite/Restriction: One of the following within the last year or three
consecutive semesters (including summer): ACT Math score of at least 25or
equivalent SAT Math score, AP Calculus AB score of at least 3, Grade of C- or
better in MATH 1050, or satisfactory score on the Math Placement Exam. Course
Fee: \$75.00
%%%%%%%%%%%%%%%%%%%%%%%%%%%%%%%%%%%%%%%%%%%%%%%%%%%%%%%%%%%%%%%%%%%%%%%%%%%%%%%%
%%%%%%%%%%%%%%%%%%%%%%%%%%%%%%%%%%%%%%%%%%%%%%%%%%%%%%%%%%%%%%%%%%%%%%%%%%%%%%%%
\vskip0.1in\hrule\vskip0.1in \noindent
{\bf Instructor Course Comments:}
\vskip0.1in\hrule\vskip0.1in
\noindent
It is important for students to keep up with the material in this course. Past
experience of this instructor (and other instructors) indicates that students
who do not attend class/lecture do not perform well in mathematics courses. The
following comments should be kept in mind while you are enrolled in Math 1100
or any other course in mathematics and statistics.
%%%%%%%%%%%%%%%%%%%%%%%%%%%%%%%%%%%%%%%%%%%%%%%%%%%%%%%%%%%%%%%%%%%%%%%%%%%%%%%%
%%%%%%%%%%%%%%%%%%%%%%%%%%%%%%%%%%%%%%%%%%%%%%%%%%%%%%%%%%%%%%%%%%%%%%%%%%%%%%%%
\vskip0.1in\hrule\vskip0.1in\noindent
\begin{list}{$\bullet$}{\usecounter{beans} \parsep=0pt \listparindent=0pt
\topsep=0pt \rightmargin=.35in \leftmargin=.35in  \labelsep=5 pt \itemsep=2pt}
  \item It is important that students attend lecture either face to face or
        through web broadcast. Homework assignments will be given in class by
        the instructor for the course. Due dates and sections covered will also
        be announced in class.
  \item TAKE NOTES!!!! Get a pen/pencil and paper out and take notes during
        lectures. This includes web broadcast versions of the class meetings.
        This should help you study the material discussed in class in
        your own words. Taking notes will also help you stay organized and keep
        up with the material discussed in the course. You can also used the
        recorded lectures for the course.
  \item Spend time after each lecture reading and working homework problems to
        keep up with the content.
  \item Students should spend about 2 hours per 50 minute lecture studying the
        material presented. It is important to establish good study habits from
        the beginning of the class.
  \item The course content will include many examples of real world problems.
        Topics involving examples like supply and demand, production models,
        optimization of profit, present/future value, and more will be included
        in the course. This is done to make the course useful in later courses
        taken by business and life science majors.
\end{list}

\vskip0.1in

\noindent
If you have any questions about the course material, course policies, or any
other matters, please contact the instructor via email, before or after class,
or during office hours.
%%%%%%%%%%%%%%%%%%%%%%%%%%%%%%%%%%%%%%%%%%%%%%%%%%%%%%%%%%%%%%%%%%%%%%%%%%%%%%%%
%%%%%%%%%%%%%%%%%%%%%%%%%%%%%%%%%%%%%%%%%%%%%%%%%%%%%%%%%%%%%%%%%%%%%%%%%%%%%%%%
\vskip0.1in\hrule\vskip0.1in
\noindent
{\bf Homework, Exams, and Comprehensive Final Grading:}
\vskip0.1in\hrule\vskip0.1in
\noindent
The grade you earn will be based on percentage as described below.

\vskip0.1in

\noindent
\begin{list}{$\bullet$}{\usecounter{beans} \parsep=0pt \listparindent=0pt
\topsep=0pt \rightmargin=.35in \leftmargin=.35in  \labelsep=5 pt \itemsep=2pt}
  \item {\bf Weekly Homework (7.5\%):} Students will be required to complete 
        weekly homework sets. Homework problems will be assigned from the
        textbook sections and content over the previous week.
  \item {\bf Weekly Quizzes (7.5\%):} A quiz from the homework will be
        administered to students at the end of each week. 
  \item {\bf Midterm Exams (45\%):} There will be 3 midterm exams. Each exam
        will account for 15\% of your grade. The exams will be given based on
        the schedule determined by your instructor and the current semester.
  \item {\bf Comprehensive Final Exam (40\%):} The final is a comprehensive
        final. The date for the final will be determined by the instructor and
        the Academic Calendar at USU. The final is required of all students and
        the date cannot be changed. The only exception for this policy is for
        USU excused absences.
\end{list}

\vskip0.1in

\noindent
Suggested homework problems will be provided at the beginning of the semester.
Homework assignments turned in for grading will be comprised of a subset of the
suggested problems. The suggested problems not assigned for homework can be used
to practice the skills learned in the course. Keep in mind that the homework
problems will prepare you for the weekly quizzes, the quizzes will prepare you
for the midterm exams, and the midterm exams will prepare you for the final
exam. Therefore, it is important to keep up with the homework in the course.
This is the basis for the rest of the graded material in the course.
%%%%%%%%%%%%%%%%%%%%%%%%%%%%%%%%%%%%%%%%%%%%%%%%%%%%%%%%%%%%%%%%%%%%%%%%%%%%%%%%
%%%%%%%%%%%%%%%%%%%%%%%%%%%%%%%%%%%%%%%%%%%%%%%%%%%%%%%%%%%%%%%%%%%%%%%%%%%%%%%%
\vskip0.1in\hrule\vskip0.1in
\noindent
{\bf Final Grading Scale:}
\vskip0.1in\hrule\vskip0.1in
\noindent
The grade you earned will be assigned based on the translation to a letter grade
as given in the following list.
\begin{list}{$\bullet$}{\usecounter{beans} \parsep=0pt \listparindent=0pt
\topsep=0pt \rightmargin=.35in \leftmargin=.35in  \labelsep=5 pt \itemsep=2pt}
  \item 90\% to 100\% A-/A
  \item 80\% to 89\% B-/B/B+
  \item 70\% to 79\% C-/C/C+
  \item 60\% to 69\% D/D+
  \item 0\% to 59\% F
\end{list}
A passing grade is any grade of D or better. However, a C or better may be
required in Math 1100 to complete requirements for an undergraduate degree at
USU.
\vskip0.1in\hrule\vskip0.1in
%%%%%%%%%%%%%%%%%%%%%%%%%%%%%%%%%%%%%%%%%%%%%%%%%%%%%%%%%%%%%%%%%%%%%%%%%%%%%%%%
%%%%%%%%%%%%%%%%%%%%%%%%%%%%%%%%%%%%%%%%%%%%%%%%%%%%%%%%%%%%%%%%%%%%%%%%%%%%%%%%
\noindent
{\bf Instructor General Course Policies:}
\vskip0.1in\hrule\vskip0.1in\noindent
Please read the following policies carefully.
\noindent
\begin{list}{$\bullet$}{\usecounter{beans} \parsep=0pt \listparindent=0pt
\topsep=0pt \rightmargin=.35in \leftmargin=.35in  \labelsep=5 pt \itemsep=2pt}
  \item Late homework will not be accepted. Due to the large number of students
        in the course, all homework must be turned in on time.
  \item Students who are working with the DRC must fill out all paper work
        regarding extra time and any special considerations about taking
        quizzes, exams, and the final.
  \item If a student has an excused absence, the work missed during the absence
        must be completed before the absence occurs. The only exception to this
        rule is in the case of an emergency.
  \item Students are allowed to use computers and notebooks in class. Phones
        must be turned off during class. Computers and any other electronic
        devices are not allowed on exams. This includes phones, notebooks, and
        calculators. If there is an family or other issue that requires the use
        of a phone during class, arrangements must be made prior to class with
        the instructor.
  \item Students will not be allowed to use calculators on quizzes, the midterm
        exams or the final.
  \item Students are responsible for reading material in the textbook and
        keeping up with the lectures.

  \item Any questions about grading of homework, quizzes, exams, and the final 
        will be addressed by the instructor. You can check the web site listed
        on the syllabus for office hours or you can reach the instructor using
        the email address
        \begin{verbatim}

           joe.koebbe@usu.edu

        \end{verbatim}
        to make an appointment. All questions about grading will be addressed by
        the instructor of the course.
  \item Quizzes, midterm exams, and the final are all closed book exams. All you
        will need is a pencil and an eraser to complete the exams and final.
  \item Formulas needed on the exams will be included as a part of the exam at
        the discretion of the instructor. For example, it may be necessary to
        include formulas for applications covered in the course or for some more
        complicated formulas for calculus problems.
\end{list}
\vskip0.1in\hrule\vskip0.1in
\newpage
%%%%%%%%%%%%%%%%%%%%%%%%%%%%%%%%%%%%%%%%%%%%%%%%%%%%%%%%%%%%%%%%%%%%%%%%%%%%%%%%
%%%%%%%%%%%%%%%%%%%%%%%%%%%%%%%%%%%%%%%%%%%%%%%%%%%%%%%%%%%%%%%%%%%%%%%%%%%%%%%%
\vskip0.1in\hrule\vskip0.1in
\noindent
{\bf University Policies:}
\vskip0.1in\hrule\vskip0.1in
\paragraph{\underline{Instructor/Department/University Policy Documents:}}
Any policies in this document are superseded by the online policies at the
address given above.

\noindent
There are a number of Utah State University (USU) policies that apply to
students, faculty, and administrators at USU. Students should realize that any
policies stated in this syllabus are included for convenience. The official
policies applied to this course can be found online at the following address.
\begin{verbatim}

http://catalog.usu.edu/content.php?catoid=12&navoid=3587

\end{verbatim}

\paragraph{\underline{Academic Honesty/Integrity}} The University expects that
students and faculty alike maintain the highest standards of academic honesty.
For the benefit of students who may not be aware of specific standards of the
University concerning academic honesty, the following information is quoted from
The Code of Policies and Procedures for Students at Utah State University
(revised September 2009), Article VI, Section 1:

\paragraph{\underline{Section 1. University Standard: Academic Integrity}}
Students have a responsibility to promote academic integrity at the University
by not participating in or facilitating others' participation in any act of
academic dishonesty and by reporting all violations or suspected violations of
the Academic Integrity Standard to their instructors.

\paragraph{\underline{The Honor Pledge:}} To enhance the learning environment at
Utah State University and to develop student academic integrity, each student
agrees to the following Honor Pledge: "I pledge, on my honor, to conduct myself
with the foremost level of academic integrity."

\paragraph{\underline{Violations of the Academic Integrity Standard (academic
violations):}} This includes, but are not limited to:

\vskip0.1in\noindent
Cheating:
\vskip0.1in\noindent
\begin{enumerate}
  \item using or attempting to use or providing others with any unauthorized
        assistance in taking quizzes, tests, examinations, or in any other
        academic exercise or activity, including working in a group when the
        instructor has designated that the quiz, test, examination, or any other
        academic exercise or activity be done "individually";
  \item depending on the aid of sources beyond those authorized by the
        instructor in writing papers, preparing reports, solving problems, or
        carrying out other assignments;
  \item substituting for another student, or permitting another student to
        substitute for oneself, in taking an examination or preparing academic
        work;
  \item acquiring tests or other academic material belonging to a faculty
        member, staff member, or another student without express permission;
  \item continuing to write after time has been called on a quiz, test,
        examination, or any other academic exercise or activity;
  \item submitting substantially the same work for credit in more than one
        class, except with prior approval of the instructor; or
  \item engaging in any form of research fraud.
\end{enumerate}

\paragraph{\underline{Falsification:}} altering or fabricating any information
or citation in an academic exercise or activity.

\paragraph{\underline{Plagiarism:}} representing, by paraphrase or direct
quotation, the published or unpublished work of another person as one's own in
any academic exercise or activity without full and clear acknowledgment. It
also includes using materials prepared by another person or by an agency engaged
in the sale of term papers or other academic materials.

\paragraph{\underline{Section 2. Reporting Violations of Academic Integrity}}
The Academic Integrity Violation Form (AIVF) provides guidance to instructors
and students, ensures minimum due process requirements are met, and allows
tracking of repeat offenders at the University level. The AIVF is available
through the Office of the Vice President for Student Affairs.

Once an instructor has determined that an academic violation has occurred and
that a sanction is appropriate, an AIVF must be submitted prior to application
of the sanction. The student may appeal the determination that an academic
violation occurred if the AIVF is not filed.

All submitted AIVF forms are kept in the Vice President for Student Affairs
Office for the duration of the student's academic career at Utah State
University. When a resolution has been reached between the student and
instructor, a Resolution Report detailing the action taken and agreement of
both parties on that action shall be submitted to the Office of the Vice
President for Student Affairs. If no Resolution Report has been filed for a
submitted AIVF within the semester, the Campus Judicial Officer will investigate
to determine if a solution was reached and why no Resolution Report was filed.

\paragraph{\underline{Section 3. Discipline Regarding Academic Integrity
Violations}} An instructor has full autonomy to evaluate a student's academic
performance in a course. If a student commits an academic violation, the
instructor may sanction the student. Such sanctions may include:
\begin{enumerate}
  \item requiring the student to rewrite a paper/assignment or to retake a
        test/examination;
  \item adjusting the student's grade—for either an assignment/test or the
        course;
  \item giving the student a failing grade for the course; or
  \item taking actions as appropriate. Additional disciplinary action beyond
        instructor sanction shall be determined by the Judicial Officer and the
        University.
\end{enumerate}

The penalty that the University will impose on a student for the first Academic
Integrity violation is placement on academic integrity probation after the first
offense.

The penalties that the University may impose on a student for multiple or
egregious academic integrity violations are:
\begin{enumerate}
  \item Probation: continued participation in an academic program predicated
        upon the student satisfying certain requirements as specified in a
        written notice of probation. Probation is for a designated period of
        time and includes the probability of more severe disciplinary penalties
        if the student does not comply with the specified requirements or is
        found to be committing academic integrity violations during the
        probationary period. The student must request termination of the
        probation in writing.
  \item Performance of community service.
  \item Suspension: temporary dismissal from an academic program or from the
        University for a specified time, after which the student is eligible to
        continue the program or return to the University. Conditions for
        continuance or readmission may be specified.
  \item Expulsion: permanent dismissal either from an academic program or from
        the University.
  \item Assigning a designation with a course grade indicating an academic
        integrity violation involving academic integrity. Conditions for removal
        may be specified, but the designation remains on the student's
        transcript for a minimum of one year; provided however, that once the
        student's degree is posted to the transcript, the designation may not be
        removed thereafter.
  \item Denial or revocation of degrees.
\end{enumerate}
\vskip0.1in\hrule\vskip0.1in\noindent
The complete Code of Policies and Procedures for Students at Utah State
University can be viewed at: The code of policies and procedures for students at
Utah State University.
\vskip0.1in\hrule\vskip0.1in
\newpage
%%%%%%%%%%%%%%%%%%%%%%%%%%%%%%%%%%%%%%%%%%%%%%%%%%%%%%%%%%%%%%%%%%%%%%%%%%%%%%%%
%%%%%%%%%%%%%%%%%%%%%%%%%%%%%%%%%%%%%%%%%%%%%%%%%%%%%%%%%%%%%%%%%%%%%%%%%%%%%%%%
\vskip0.1in\hrule\vskip0.1in
\noindent
{\bf Policies Due to Covid-19:}
\vskip0.1in\hrule\vskip0.1in \noindent
During the Covid-19 pandemic, and prior to resuming on-site operations, all USU
departments must have an approved department operation plan to minimize the
risk of exposure to employees, students, and the public. 

\paragraph{\underline{Classroom Safety Guidelines}}

\noindent
  \underline{Social Distancing of Six Feet:} Students and faculty should
            maintain 6 feet of separation from each other.
    \begin{enumerate}
      \item Each classroom has a social distancing cap (SD cap), indicating the
            greatest number of students allowed in the room.
      \item Seats will be marked indicating where to sit.
      \item An instructional space will be marked at the front of the room.

      \item Instructors are asked to dismiss their students in an orderly manner
            to ensure distancing while leaving the room.
    \end{enumerate}

\noindent
  \underline{Face Coverings Worn by Everyone} All students and faculty are
            required to wear a face covering in the classroom.

    \begin{enumerate}
      \item If students come to class without face coverings, teachers will be
            encouraged to remind students of the university policy.
      \item If students continue to disregard the policy, teachers can submit
            their names to the Provost's office for further steps.
    \end{enumerate}

\noindent
  \underline{Surface Cleaning Before Each Class} All surfaces are to be cleaned
            at the start of each class.
      \begin{enumerate}
        \item Cleaning materials will be provided in each classroom.
        \item Teachers and students are to wipe down chairs, desks, teaching,
              and work stations at the beginning of each class.
        \item USU Facilities staff will clean surfaces more thoroughly after
              hours of operation. 
      \end{enumerate}

\noindent
    \underline{Attendance Tracking} Teachers are to keep track of classroom
              attendance.
      \begin{enumerate}
        \item Attendance is important for purposes of contact tracing in the
              event of a student receiving a positive Covid-19 test result.
        \item Teachers should not pass around a sign-up sheet, but rather mark
              student attendance.
        \item Automated procedures for collecting attendance are being
              developed.
      \end{enumerate}

\noindent
    \underline{Health and hygiene practices} Students and faculty should stay
              home when ill and use good hygiene habits. 
      \begin{enumerate}
        \item Each classroom will have hand sanitizer.
        \item Students should cover coughs and sneezes and wash and disinfect
              their hands regularly.
        \item Students and faculty exhibiting even mild illness symptoms should
              stay home. 
      \end{enumerate}

\noindent
    \underline{Syllabus Language} The Provost's office will be the main resource
              for providing standardized language to include in the course
              syllabus. That language will be posted on the Provost's web page
              when it is available.
%%%%%%%%%%%%%%%%%%%%%%%%%%%%%%%%%%%%%%%%%%%%%%%%%%%%%%%%%%%%%%%%%%%%%%%%%%%%%%%%
%%%%%%%%%%%%%%%%%%%%%%%%%%%%%%%%%%%%%%%%%%%%%%%%%%%%%%%%%%%%%%%%%%%%%%%%%%%%%%%%
\end{document}
