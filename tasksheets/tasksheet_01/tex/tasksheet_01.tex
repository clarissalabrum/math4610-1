\documentclass[10pt,fleqn]{article}
\usepackage{hyperref}
\usepackage{graphicx}

\setlength{\topmargin}{-.75in}
\addtolength{\textheight}{2.00in}
\setlength{\oddsidemargin}{.00in}
\addtolength{\textwidth}{.75in}

\nofiles

\pagestyle{empty}

\setlength{\parindent}{0in}

% new math commands


\setlength{\oddsidemargin}{-0.25in}
\setlength{\evensidemargin}{-0.25in}
\setlength{\textwidth}{6.75in}
\setlength{\headheight}{0.0in}
\setlength{\topmargin}{-0.25in}
\setlength{\textheight}{9.00in}

\makeindex

\usepackage{mathrsfs}

%\usepackage[pdftex]{graphicx}
\usepackage{epstopdf}

\newcounter{beans}

\newcommand{\ds}{\displaystyle}
\newcommand{\limit}[2]{\displaystyle\lim_{#1\to#2}}

\newcommand{\binomial}[2]{\ \left( \begin{array}{c}
                                  #1 \\
                                  #2
                                 \end{array}
                            \right) \
                         }
\newcommand{\ExampleRule}[2]
  {
  \noindent
  \rule{\linewidth}{1pt}
  \begin{example}
    #1
    \label{#2}
  \end{example}
  \rule{\linewidth}{1pt}
  \vskip0.125in
  }

\newcommand{\defbox}[1]
  {
   \ \\
   \noindent
   \setlength\fboxrule{1pt}
   \fbox{
        \begin{minipage}{6.5in}
          #1
        \end{minipage}
        }
   \ \\
  }
\newcommand{\verysmallworkbox}[1]
  {
   \ \\
   \noindent
   \setlength\fboxrule{1pt}
   \fbox{
        \begin{minipage}{6.5in}
           #1
           \ \\
           \vskip0.5in \ \\
           \ \\
        \end{minipage}
        }
   \ \\
  }
\newcommand{\smallworkbox}[1]
  {
   \ \\
   \noindent
   \setlength\fboxrule{1pt}
   \fbox{
        \begin{minipage}{6.5in}
           #1
           \ \\
           \vskip2.5in \ \\
           \ \\
        \end{minipage}
        }
   \ \\
  }
\newcommand{\halfworkbox}[1]
  {
   \ \\
   \noindent
   \setlength\fboxrule{1pt}
   \fbox{
        \begin{minipage}{6.5in}
           #1 \hfill
           \ \\
           \vskip3.25in \ \\
           \ \\
        \end{minipage}
        }
   \ \\
  }
\newcommand{\largeworkbox}[1]
  {
   \ \\
   \noindent
   \setlength\fboxrule{1pt}
   \fbox{
        \begin{minipage}{6.5in}
           #1
           \ \\
           \vskip7.5in \ \\
           \ \\
        \end{minipage}
        }
   \ \\
  }
\newcommand{\flexworkbox}[2]
  {
   \ \\
   \noindent
   \setlength\fboxrule{1pt}
   \fbox{
        \begin{minipage}{6.5in}
           #1
           \ \\

           \vskip#2 \ \\
           \ \\
        \end{minipage}
        }
   \ \\
  }


% symbols for sets of numbers

\newcommand{\natnumb}{$\cal N$}
\newcommand{\whonumb}{$\cal W$}
\newcommand{\intnumb}{$\cal Z$}
\newcommand{\ratnumb}{$\cal Q$}
\newcommand{\irrnumb}{$\cal I$}
\newcommand{\realnumb}{$\cal R$}
\newcommand{\cmplxnumb}{$\cal C$}

% misc. commands

\newcommand{\mma}{{\it Mathematica}}
\newcommand{\sech}{\mbox{ sech}}
 
\newtheorem{theorem}{Theorem}
\newtheorem{example}{Example}
\newtheorem{definition}{Definition}
\newtheorem{problem}{Problem}

\setcounter{secnumdepth}{2}
\setcounter{tocdepth}{4}


\begin{document}
%%%%%%%%%%%%%%%%%%%%%%%%%%%%%%%%%%%%%%%%%%%%%%%%%%%%%%%%%%%%%%%%%%%%%%%%%%%%%%%%
%%%%%%%%%%%%%%%%%%%%%%%%%%%%%%%%%%%%%%%%%%%%%%%%%%%%%%%%%%%%%%%%%%%%%%%%%%%%%%%%
\vskip0.1in\hrule\vskip0.1in \noindent
{\bf{\Large Math 4610 Fundamentals of Numerical Analysis Tasksheet 1 }}
\vskip0.1in\hrule\vskip0.1in \noindent
The problems for the Tasksheet 01 are included below. The deadline for turning
in your work on these problems will be posted on the repository. In addition,
you will turn in your work through the math4610 repository. A directory will be
constructed that will be used as a place to store your work.
\vskip0.1in\hrule\vskip0.1in \noindent
{\bf{\large Tasks}}
\vskip0.1in\hrule\vskip0.1in \noindent
\begin{trivlist}
  \item[\bf Task 1:] Set up an appointment to meet with your instructor via
        Zoom. There are a couple of ways to do this. During office hours you
        can join a Zoom meeting that will be running. The information, including
        a link and password will be sent to you via your email address. If you
        cannot make office hours, you can set up an appointment at another time
        to hold the meeting. To do this, (1) email your instructor with a time
        that will work, (2) copy the link in the return email into a browser,
        and (3) join the meeting on Zoom that has been set up.
\vskip0.1in\hrule\vskip0.1in \noindent
  \item[\bf Task 2:] Email a link to your Github page. The links should be of
        the form
        \begin{verbatim}

             https://username.github.io/

        \end{verbatim}
        where the \lq\lq username\rq\rq\ is the name you chose when setting up
        your account. Your instructor should be able to see your repositories 
        and any public files that are on the repository.
\vskip0.1in\hrule\vskip0.1in \noindent
  \item[\bf Task 3:] In class we discussed the creation of a repository for
        your work in this course. This must be done as soon as possible during
        or just after the first lecture. Once you have created a private
        repository named
        \begin{verbatim}

             math4610

        \end{verbatim}
        with a initial README.md file, do the following. Make sure that the
        repository is private and that the only other collaborator is the
        instructor for the class. The instructor email is contained in the
        syllabus for the course and can be accessed as follows.
        \begin{verbatim}

             https://jvkoebbe.github.io/math4610/syllabus/md/syllabus

        \end{verbatim}
        or
        \begin{verbatim}

             https://jvkoebbe.github.io/math4610/syllabus/pdf/syllabus.pdf

        \end{verbatim}
        Note, that inviting your instructor to be a collaborator will
        automatically be sent an email to respond to.
\vskip0.1in\hrule\vskip0.1in \noindent
  \item[\bf Task 4:] Email your instructor as to which command line environment
        you will use in the course. Possible choices are: (1) term on Cygwin,
        (2) term or xterm on Linux, (3) the command window on MacOS, or (5)
        PowerShell on Windows. If you plan on emulating your instructor's
        terminal emulator, you should probably install Cygwin. However,
        installation of Cygwin is not required for the course.
\vskip0.1in\hrule\vskip0.1in \noindent
  \item[\bf Task 5:] Email the instructor the link to the table of contents
        for your homework solutions. The link should be something like
        \begin{verbatim}

             https://username.github.io/math4610/hw_toc

        \end{verbatim}
        Notice that the extension \lq\lq .md\rq\rq\ is not included if the file
\vskip0.1in\hrule\vskip0.1in \noindent
  \item[\bf Task 6:] Search the internet for sites that talk about Version
        Control Systems (VCS) like Github. Write a brief paragraph (3 or 4
        sentences) that describe your findings. Include links to the sites you
        cite in your response.
\end{trivlist}
\vskip0.1in\hrule\vskip0.1in \noindent
  \href{../../../README.md}{Previous} |
  \href{../../toc/md/tasksheet_toc.md}{Table of Contents} |
  \href{../../tasksheet_02/html/tasksheet_02.html}{Next}
\vskip0.1in\hrule\vskip0.1in \noindent
%%%%%%%%%%%%%%%%%%%%%%%%%%%%%%%%%%%%%%%%%%%%%%%%%%%%%%%%%%%%%%%%%%%%%%%%%%%%%%%%
%%%%%%%%%%%%%%%%%%%%%%%%%%%%%%%%%%%%%%%%%%%%%%%%%%%%%%%%%%%%%%%%%%%%%%%%%%%%%%%%
\end{document}
