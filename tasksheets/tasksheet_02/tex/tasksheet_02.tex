\documentclass[10pt,fleqn]{article}
\usepackage{hyperref}
\usepackage{graphicx}

\setlength{\topmargin}{-.75in}
\addtolength{\textheight}{2.00in}
\setlength{\oddsidemargin}{.00in}
\addtolength{\textwidth}{.75in}

\nofiles

\pagestyle{empty}

\setlength{\parindent}{0in}

% new math commands


\setlength{\oddsidemargin}{-0.25in}
\setlength{\evensidemargin}{-0.25in}
\setlength{\textwidth}{6.75in}
\setlength{\headheight}{0.0in}
\setlength{\topmargin}{-0.25in}
\setlength{\textheight}{9.00in}

\makeindex

\usepackage{mathrsfs}

%\usepackage[pdftex]{graphicx}
\usepackage{epstopdf}

\newcounter{beans}

\newcommand{\ds}{\displaystyle}
\newcommand{\limit}[2]{\displaystyle\lim_{#1\to#2}}

\newcommand{\binomial}[2]{\ \left( \begin{array}{c}
                                  #1 \\
                                  #2
                                 \end{array}
                            \right) \
                         }
\newcommand{\ExampleRule}[2]
  {
  \noindent
  \rule{\linewidth}{1pt}
  \begin{example}
    #1
    \label{#2}
  \end{example}
  \rule{\linewidth}{1pt}
  \vskip0.125in
  }

\newcommand{\defbox}[1]
  {
   \ \\
   \noindent
   \setlength\fboxrule{1pt}
   \fbox{
        \begin{minipage}{6.5in}
          #1
        \end{minipage}
        }
   \ \\
  }
\newcommand{\verysmallworkbox}[1]
  {
   \ \\
   \noindent
   \setlength\fboxrule{1pt}
   \fbox{
        \begin{minipage}{6.5in}
           #1
           \ \\
           \vskip0.5in \ \\
           \ \\
        \end{minipage}
        }
   \ \\
  }
\newcommand{\smallworkbox}[1]
  {
   \ \\
   \noindent
   \setlength\fboxrule{1pt}
   \fbox{
        \begin{minipage}{6.5in}
           #1
           \ \\
           \vskip2.5in \ \\
           \ \\
        \end{minipage}
        }
   \ \\
  }
\newcommand{\halfworkbox}[1]
  {
   \ \\
   \noindent
   \setlength\fboxrule{1pt}
   \fbox{
        \begin{minipage}{6.5in}
           #1 \hfill
           \ \\
           \vskip3.25in \ \\
           \ \\
        \end{minipage}
        }
   \ \\
  }
\newcommand{\largeworkbox}[1]
  {
   \ \\
   \noindent
   \setlength\fboxrule{1pt}
   \fbox{
        \begin{minipage}{6.5in}
           #1
           \ \\
           \vskip7.5in \ \\
           \ \\
        \end{minipage}
        }
   \ \\
  }
\newcommand{\flexworkbox}[2]
  {
   \ \\
   \noindent
   \setlength\fboxrule{1pt}
   \fbox{
        \begin{minipage}{6.5in}
           #1
           \ \\

           \vskip#2 \ \\
           \ \\
        \end{minipage}
        }
   \ \\
  }


% symbols for sets of numbers

\newcommand{\natnumb}{$\cal N$}
\newcommand{\whonumb}{$\cal W$}
\newcommand{\intnumb}{$\cal Z$}
\newcommand{\ratnumb}{$\cal Q$}
\newcommand{\irrnumb}{$\cal I$}
\newcommand{\realnumb}{$\cal R$}
\newcommand{\cmplxnumb}{$\cal C$}

% misc. commands

\newcommand{\mma}{{\it Mathematica}}
\newcommand{\sech}{\mbox{ sech}}
 
\newtheorem{theorem}{Theorem}
\newtheorem{example}{Example}
\newtheorem{definition}{Definition}
\newtheorem{problem}{Problem}

\setcounter{secnumdepth}{2}
\setcounter{tocdepth}{4}


\begin{document}
%%%%%%%%%%%%%%%%%%%%%%%%%%%%%%%%%%%%%%%%%%%%%%%%%%%%%%%%%%%%%%%%%%%%%%%%%%%%%%%%
%%%%%%%%%%%%%%%%%%%%%%%%%%%%%%%%%%%%%%%%%%%%%%%%%%%%%%%%%%%%%%%%%%%%%%%%%%%%%%%%
\vskip0.1in\hrule\vskip0.1in \noindent
{\bf{\Large Math 4610 Fundamentals of Numerical Analysis Tasksheet 2}}
\vskip0.1in\hrule\vskip0.1in \noindent
The problems for the Tasksheet 02 are included below. The deadline for turning
in your work on these problems will be posted on the repository. In addition,
you will turn in your work through the math4610 repository. A directory will be
constructed that will be used as a place to store your work.
\vskip0.1in\hrule\vskip0.1in \noindent
{\bf{\large Tasks}}
\vskip0.1in\hrule\vskip0.1in \noindent
\begin{trivlist}
  \item[\bf Task 1:] After deciding which programming language you will use,
        write a program that provides a response to the "Hello World!" output.
        Provide details of the compilation and execution of the code you write.
        You can choose a suitable response - say something like "...it's only a
        bunny..."
\vskip0.1in\hrule\vskip0.1in \noindent
  \item[\bf Task 2:] Edit the main README.md file in your Math 4610 repository
        on Github using your favorite browser. In the README.md file, create an
        introduction that describes what the repository is being created for and
        put in a link to the table of contents for the homework problems and a
        link to the software manual you will create. Do this in Markdown - not
        html.
\vskip0.1in\hrule\vskip0.1in \noindent
  \item[\bf Task 3:] Create the table of contents in a file with a name like
        \begin{verbatim}

          task_toc.md

        \end{verbatim}
        or something like this. After completing the tasks to this point, clone
        the repository to a local directory on a local computer using git. If
        you have already cloned the repository, use a
        \begin{verbatim}

           koebbe% git pull

        \end{verbatim}
        command to do this.
\vskip0.1in\hrule\vskip0.1in \noindent
  \item[\bf Task 4:] Write out the analysis for the centered difference
        approximation in a Taylor series expansions. Show that the approximation
        is second order.
\vskip0.1in\hrule\vskip0.1in \noindent
  \item[\bf Task 5:] Determine the order of accuracy of the central difference
        approximation of the second derivative. That is, analyze
        \[
          f''(x) \approx {{f(x+h) - 2 f(x) + f(x-h)}\over{h}}
        \]
        Write a code that approximates the second derivative of the function
        \[
          f(x) = cos(x)
        \] 
        at the point, \(x=2.0\).
\vskip0.1in\hrule\vskip0.1in \noindent
  \item[\bf Task 6:] Search the internet for information regarding finite
        difference approximations of derivatives of different orders. Give
        examples of these types of approximations. You should be able to find
        many such examples. Write a brief paragraph (3 or 4 sentences) that
        describe your findings. Include links to the sites you cite.
\end{trivlist}
\vskip0.1in\hrule\vskip0.1in \noindent
  \href{../../tasksheet_01/html/tasksheet_01.html}{Previous}
  \href{../../toc/md/tasksheet_toc.md}{Table of Contents} |
  \href{../../tasksheet_03/html/tasksheet_03.html}{Next}
\vskip0.1in\hrule\vskip0.1in \noindent
%%%%%%%%%%%%%%%%%%%%%%%%%%%%%%%%%%%%%%%%%%%%%%%%%%%%%%%%%%%%%%%%%%%%%%%%%%%%%%%%
%%%%%%%%%%%%%%%%%%%%%%%%%%%%%%%%%%%%%%%%%%%%%%%%%%%%%%%%%%%%%%%%%%%%%%%%%%%%%%%%
\end{document}
