\documentclass[10pt,fleqn]{article}
\usepackage{hyperref}
\usepackage{graphicx}

\setlength{\topmargin}{-.75in}
\addtolength{\textheight}{2.00in}
\setlength{\oddsidemargin}{.00in}
\addtolength{\textwidth}{.75in}

\nofiles

\pagestyle{empty}

\setlength{\parindent}{0in}

\input{/cygdrive/m/tex/commands/commands}

\begin{document}
%%%%%%%%%%%%%%%%%%%%%%%%%%%%%%%%%%%%%%%%%%%%%%%%%%%%%%%%%%%%%%%%%%%%%%%%%%%%%%%%
%%%%%%%%%%%%%%%%%%%%%%%%%%%%%%%%%%%%%%%%%%%%%%%%%%%%%%%%%%%%%%%%%%%%%%%%%%%%%%%%
\vskip0.1in\hrule\vskip0.1in \noindent
{\bf{\Large Math 4610 Fundamentals of Numerical Analysis Tasksheet 4}}
\vskip0.1in\hrule\vskip0.1in \noindent
The problems for the Tasksheet 03 are included below. The deadline for turning
in your work on these problems will be posted on the repository. In addition,
you will turn in your work through the math4610 repository. A directory will be
constructed that will be used as a place to store your work.
\vskip0.1in\hrule\vskip0.1in \noindent
{\bf{\large Tasks}}
\vskip0.1in\hrule\vskip0.1in \noindent
\begin{trivlist}
  \item[\bf Task 1:]
    Create routines that compute the absolute error and relative error in
    approximating a number, \(x\), with another number, \(y\). Write two
    separate routines. The routines should be documented in your software manual
    and also include the production version of the code in your archive file.
\vskip0.1in\hrule\vskip0.1in \noindent
  \item[\bf Task 2:] Create a graphics routine that will create a 2D plot of 
    data that includes the following.
    \begin{trivlist}
      \item[1.] Graph string expressions in a Python module. 
      \item[2.] Include a loop to graph multiple expressions for comparison.
      \item[3.] Use different colors for the graphs of the functions.
      \item[4.] Hardcode axes labels.
      \item[5.] Do note include a title for the graph.
      \item[6.] Include a legend for the curves.
    \end{trivlist}
    If you are working mostly in Python you can choose to include the module in
    your archive. If you are not coding primarily in Python, you should make
    sure you document this in your software manual as an extra code.
\vskip0.1in\hrule\vskip0.1in \noindent
  \item[\bf Task 3:] 
    Write a routine that will approximate the location of a root using fixed
    point iteration or functional iteration using the fixed point algorithm.
    That is, if \(x^*\) is a root of the function, \(f(x)\), then set
    \[
      x = x - f(x)
    \]
    and given an initial approximation, \(x_0\), define the sequence, \(x_k\),
    by
    \[
      x_{k+1} = x_k - f(x_k)
    \]
    for \(k=0,1,2\ldots\). Make sure you put in an appropriate stopping
    criteria. Include documentation for the code in your software manual and add
    the routine to your shared archive.
\vskip0.1in\hrule\vskip0.1in \noindent
  \item[\bf Task 4:] Apply a fixed point iteration to finding the closest root
    to zero of the following function.
    \[
      f(x) = x\ e^{3x^2} - 7\ x
    \]
    Use the fixed point iteration
    \[
      x = x - f(x)
    \]
    to test the idea. Then try
    \[
      x = x - \epsilon\ f(x)
    \]
    where \(\epsilon\) is chosen to get convergence of the sequence of
    approximations. You can to this by applying the convergence criterion for
    fixed point iteration.
\vskip0.1in\hrule\vskip0.1in \noindent
  \item[\bf Task 5:] Write a routine/code to implement the Bisection method for
    approximating the location of a root of a function of one variable. Write
    the code to incorporate the computation of the number of steps to meet a
    given tolerance instead of the while loop version described in class. Test
    your code on the problem defined in Task 4.

\vskip0.1in\hrule\vskip0.1in \noindent
  \item[\bf Task 6:] Search the internet for sites that discuss root finding
        problems in practice. Write a brief summary of what you find including
        the pros and cons of shared libraries. Your write up should be a brief
        paragraph (3 or 4 sentences) that describe your findings. Include links
        to the sites you cite.
\end{trivlist}
\vskip0.1in\hrule\vskip0.1in \noindent
  \href{../../tasksheet_03/html/tasksheet_03.html}{Previous}
  \href{../../toc/md/tasksheet_toc.md}{Table of Contents} |
  \href{../../tasksheet_05/html/tasksheet_05.html}{Next}
\vskip0.1in\hrule\vskip0.1in \noindent
%%%%%%%%%%%%%%%%%%%%%%%%%%%%%%%%%%%%%%%%%%%%%%%%%%%%%%%%%%%%%%%%%%%%%%%%%%%%%%%%
%%%%%%%%%%%%%%%%%%%%%%%%%%%%%%%%%%%%%%%%%%%%%%%%%%%%%%%%%%%%%%%%%%%%%%%%%%%%%%%%
\end{document}
