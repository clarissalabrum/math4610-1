\documentclass[10pt,fleqn]{article}
\usepackage{hyperref}
\usepackage{graphicx}

\setlength{\topmargin}{-.75in}
\addtolength{\textheight}{2.00in}
\setlength{\oddsidemargin}{.00in}
\addtolength{\textwidth}{.75in}

\nofiles

\pagestyle{empty}

\setlength{\parindent}{0in}

\input{/cygdrive/m/tex/commands/commands}

\begin{document}
%%%%%%%%%%%%%%%%%%%%%%%%%%%%%%%%%%%%%%%%%%%%%%%%%%%%%%%%%%%%%%%%%%%%%%%%%%%%%%%%
%%%%%%%%%%%%%%%%%%%%%%%%%%%%%%%%%%%%%%%%%%%%%%%%%%%%%%%%%%%%%%%%%%%%%%%%%%%%%%%%
\vskip0.1in\hrule\vskip0.1in \noindent
{\bf{\Large Math 4610 Fundamentals of Numerical Analysis Tasksheet 4}}
\vskip0.1in\hrule\vskip0.1in \noindent
The problems for the Tasksheet 03 are included below. The deadline for turning
in your work on these problems will be posted on the repository. In addition,
you will turn in your work through the math4610 repository. A directory will be
constructed that will be used as a place to store your work.
\vskip0.1in\hrule\vskip0.1in \noindent
{\bf{\large Tasks}}
\vskip0.1in\hrule\vskip0.1in \noindent
\begin{trivlist}
  \item[\bf Task 1:]
    Create a code that will search for a root of a function, \(f(x)\), using
    Newton's method. Use the problem defined in Tasksheet 4, Task 4 to test the
    code you write. Create a software manual entry and shared library addition
    for your Newton method routine. Include the example problem results in your
    software manual entry.
\vskip0.1in\hrule\vskip0.1in \noindent
  \item[\bf Task 2:] Repeat Task 1 for the secant method. It should be easy to
    modify a Newton method code to implement the secant method.
\vskip0.1in\hrule\vskip0.1in \noindent
  \item[\bf Task 3:] Do a computational convergence analysis on Newton's method
    to verify quadratic convergence. Use the example defined in Tasksheet 4 to
    illustrate the work. 
\vskip0.1in\hrule\vskip0.1in \noindent
  \item[\bf Task 4:] Repeat Task 3 for the secant method.
\vskip0.1in\hrule\vskip0.1in \noindent
  \item[\bf Task 5:] Create a hybrid method that will search for roots by
    combining the Bisection method when the approximations are too far from a
    root and then switches over to Newton's method when the approximations are
    close enough.
\vskip0.1in\hrule\vskip0.1in \noindent
  \item[\bf Task 6:] Search the internet for sites that detail differences
    between the Bisection method, Newton's method, and the Secant method. Write
    a brief summary of what you find including the pros and cons of the methods.
    Your write up should be a brief paragraph (3 or 4 sentences) that describe
    your findings. Include links to the sites you cite.
\end{trivlist}
\vskip0.1in\hrule\vskip0.1in \noindent
  \href{../../tasksheet_04/html/tasksheet_04.html}{Previous}
  \href{../../toc/md/tasksheet_toc.md}{Table of Contents} |
  \href{../../tasksheet_06/html/tasksheet_06.html}{Next}
\vskip0.1in\hrule\vskip0.1in \noindent
%%%%%%%%%%%%%%%%%%%%%%%%%%%%%%%%%%%%%%%%%%%%%%%%%%%%%%%%%%%%%%%%%%%%%%%%%%%%%%%%
%%%%%%%%%%%%%%%%%%%%%%%%%%%%%%%%%%%%%%%%%%%%%%%%%%%%%%%%%%%%%%%%%%%%%%%%%%%%%%%%
\end{document}
