
\usepackage[pdftex]{graphicx}

\begin{document}

\includegraphics[width=3.0in]{./usu_logo.png}



\documentclass{article}
\usepackage{graphicx}
\usepackage{epstopdf}

\epstopdfDeclareGraphicsRule{.gif}{png}{.png}{convert gif:#1 png:\OutputFile}
\AppendGraphicsExtensions{.gif}

\begin{document}

  \includegraphics{demo.eps}\qquad
  \includegraphics{knuth-tex.gif}

\end{document}


Including graphics in a LaTeX document
There are two ways to incorporate images into your LaTeX document, and
both use the graphicx package by means of putting the command
\usepackage{graphicx} near the top of the LaTeX file, just after the
documentclass command. 

The two methods are
1. include only PostScript images (esp. ``Encapsulated PostScript'')
   if your goal is a PostScript document using dvips 
2. include only PDF, PNG, JPEG and GIF images if your goal is a PDF
   document using pdflatex, TeXShop, or other PDF-oriented compiler. 

In all cases, each image must be in an individual 1-image file; no
animation files or multipage documents. 

Only PostScript Images

Most mathematical/scientific graphics software allows you to save
graphics (figures, diagrams, graphs) in PostScript form; this includes
Mathematica, Maple, Matlab, IDL, and xfig (a marvelous X figure-drawing
program). Even bitmap images like JPEG and PNG files can be converted
to PostScript form with programs like ``xv'' or ``convert''. 

Where you want a PostScript image to appear, use the \includegraphics
command, possibly with scaling or rotation options, e.g.,

  \includegraphics{myfig.eps}
  \includegraphics[width=60mm]{myfig.eps}
  \includegraphics[height=60mm]{myfig.eps}
  \includegraphics[scale=0.75]{myfig.eps}
  \includegraphics[angle=45,width=52mm]{myfig.eps}

Of course, any of these can be inserted into a figure environment,
and/or a centering environment or a framebox. Then you must compile
the document with latex followed by dvips -Ppdf, which produces a
PostScript document with embedded PostScript images. You can convert
the PostScript document to PDF using ``ps2pdf'' or ``dvipdf''. 

Two characteristics which the component (Encapsulated) PostScript
image files must have are � no file contains more than one page 
� each has a valid BoundingBox line If the PostScript file does not
include a bounding box line, you can insert it using the text editor
of your choice. 

Only non-PostScript images

The compiler pdflatex (Unix) and TeXShop (Macintosh) convert LaTeX source
directly to PDF, and do not accept PostScript images. Instead, they take
PDF images, as well as bitmap pictures in PNG or JPEG or GIF format. So
to use pdflatex, you must convert any PostScript images to one of these
other forms. For photos, JPEG is best. For other bitmap images, PNG is best.
For non-bitmap images (e.g., graphs, drawings, stuff with text and symbols)
it is best to convert to PDF, using the command epstopdf (in the usual TeX
bin directory, e.g., /usr/local/tex/bin/epstopdf). The

commandunix> epstopdf myfig.eps

produces the file myfig.pdf, which can then be used in the command
\includegraphics{myfig.pdf}. for compilation with pdflatex or TeXShop. 
The possible \includegraphics options with PDF/PNG/JPEG images are much
the same as with PostScript, e.g.,

  \includegraphics{myfig.pdf}
  \includegraphics[width=60mm]{myfig.png}
  \includegraphics[height=60mm]{myfig.jpg}
  \includegraphics[scale=0.75]{myfig.pdf}
  \includegraphics[angle=45,width=52mm]{myfig.jpg}

With PNG or JPEG you should specify an explicit width or height rather
than "scale", since bitmap images have no intrinsic size, nothing
corresponding to Bounding Box information, although graphicx seems to
use 72 pixels per inch as a default size for bitmap images. 

LaTeX slide presentations 


This exposition is a HOWTO for including graphics in LATEX. Many
techniques and tools discussed here have alternatives that may be
preferred by other TEXperts; however, this document outlines my preferences
which arise from years of experience trying many tools and techniques.

Feel free to email me if you would like to offer any suggestions. I
will be glad to (at least) listen to your suggestions, but I make no
promises that you will change my mind without sufficient evidence. 

1  Personal Preference

Even though LATEX supports several graphic formats, I prefer ONLY two of
these formats, namely � Encapsulated PostScript (EPS) when using LATEX 
� Portable Document Format (PDF) when using PDFLATEX

Since LATEX cannot compute the CropBox of a PDF graphic in order to
allocate space in the document, EPS is my preferred format when using
LATEX. Similarly, since PDFLATEX can do (virtually) no PostScript
processing, PDF graphics are my preferred format when using PDFLATEX.
I often have both formats of the same graphic available to my document so
that the same source code can be built using both LATEX and PDFLATEX. 

2  Inclusion

Inclusion of EPS/PDF graphics in LATEX/PDFLATEX is trivial using the
graphicx package. If foo.eps and foo.pdf are the filenames of the EPS
and PDF versions, respectively, then the graphic can be included in the
LATEX document by using the following minimal template:  

\documentclass{minimal} 
  � 
\usepackage{graphicx} 
  � 
\begin{document} 
  � 
\includegraphics{foo} 
  � 
\end{document} 

Notice that the file extension is not specified in the filename argument
of the \includegraphics command. The purpose of this omission is that .eps
will be assumed when the document is built with LATEX and .pdf when it is
built with PDFLATEX. 



3  Raster Graphics

Raster graphics are graphics that are described by painting each individual
element of a grid a particular color. A finer grid will constitute a
�better� looking image because the colors will appear to be more
continuous. Human eyes naturally see objects in this way except that the grid
is radial and extremely dense. Computer screens rasterize all graphics since
the screen is composed of pixels that are each painted a particular color.
Higher resolution screens equate to a finer grid and thus display
�better� looking images. Also, digital cameras capture natural images
and rasterize these images for storage onto the camera�s memory card. 

One disadvantage of raster graphics is its lack of scalability. Raster
graphics cannot be scaled to higher resolutions since such scaling essentially
increases the size of each grid element. This scaling produces a larger image,
but the image looks no better than the original. 

Another disadvantage of raster graphics is its large file sizes. Since each
element of the grid must be painted its own color and since each color is
typically described with 24 bits (or 3 bytes) of storage, a 1 megapixel image
would require 3 megabytes or storage in its uncompressed form. Although lossy
techniques may be used to compress the data, they often degrade the quality
of the image. 

Common raster formats include BMP, GIF, JPG, PNG, et. al. 

4  Vector Graphics

Vector graphics are graphics that are described using pre-defined objects,
transformations, colors, strings, and other mathematical and programmatic
primitives. Unlike raster graphics, grid elements are not painted, and
therefore vector graphics require a less na�ve method for constructing
them. Virtually all mathematical drawings and graphs are (naturally) in vector
format. However, as discussed above, many applications (such as digital
cameras, computer screens, printers, etc.) in their final output rasterize
all graphics. 

The primary advantage of vector graphics is its scalability. An example of
this would be in order to scale a circle centered at the origin to 200% of
its original size, it is enough to simply multiply the radius of the circle
by 2 and the new description is complete. Therefore, there is no loss in quality
in performing such scalings. 

Another advantage of vector graphics is its small filesize. Again, since the
graphics are described by mathematical primitives, the storage required to
denote this description is minimal. 

Common vector formats include EPS, PDF, PS, SVG, et. al. 

5  Vectorizing Raster Graphics

As mentioned above, EPS and PDF are the ONLY formats that I prefer to use with
LATEX and PDFLATEX. Furthermore, EPS and PDF are both vector graphic formats.
Therefore, in order to include raster graphics in my LATEX documents, I always
do one of two things: 

�Trace the raster graphic either manually or automatically 
�Place a vector �wrapper� around the raster graphic

The former of these is performed when the raster image originates from line art
graphics for which its original vector format is not available. There are
several tools that can automatically trace these type of graphics including
AutoTrace, Potrace, and VectorMagic; however, their results may be less than
desirable. 

The most common method of vectorizing a raster image is performed when the
raster image comes from a photograph or a screenshot. In this case, a vector
�wrapper� is placed around the raster graphic to transform the raster
into a vector format. Consequently, the resulting vector graphic will maintain
the same scalability and filesize limitations of the original raster version,
but the work of transforming the graphic into EPS or PDF will already have been
done. This new graphic is not a true vector graphic, but it will possess a
convenient vector wrapper. Because it produces little overhead in filesize and
maintains the image�s original quality, sam2p is my preferred application 
for this scenerio, and my web interface is useful when converting only a few
graphics. 

6  Generating Vector Graphics

Applications such as Adobe Photoshop, Corel Paint Shop Pro, Gimp, etc. are
raster graphic editors. They generally can import vector graphics, but these
graphics are rasterized the moment they are imported. There are a plethora of
graphics programs designed to generate vector graphics. Among these include
Microsoft Visio, CorelDraw, Inkscape, Dia, MetaPost, et. al. Furthermore, most
mathematics software including Matlab, Octave, Maple, Mathematica, Maxima, et.
al., can export graphics in at least an EPS vector format. 

Most modern TEX distributions should contain epstopdf which is used to convert
EPS graphics into PDF so that the BoundingBox of the EPS is preserved as the
CropBox of the resulting PDF. For users that have command-line phobia, GSview
is a nice tool for, among other things, converting EPS into PDF. GSview relies
on Ghostscript to perform much of its backend processing. 

A common problem that Microsoft Windows users often encounter is the inability
of their applications to export EPS or PDF graphics. If these graphics are
not raster in nature, then a simple solution to this problem is to install a
generic PostScript printer driver and print to this printer. Any PostScript
printer driver, e.g., the Apple Color LaserWriter 12/600, may be used, but the
driver should be instructed to print to FILE instead of a physical printer port.
Once this is configured, printing to this �printer� will prompt for a
filename to save the resulting PostScript document. This PostScript document
will generally not be an EPS since it will lack a BoundingBox. This problem
can be solved using GSview by selecting �PS to EPS�. The user will then
have the option of having GSview automatically compute the BoundingBox or
manually specifying it. 

7  MetaPost

MetaPost is a powerful graphics language based on Knuth�s METAFONT, but
with PostScript output and facilities for including typeset text. There are
many MetaPost resources available, but this section will highlight the
fundamentals of including MetaPost graphics in a LaTeX document. I use the
following minimal MetaPost template when labels exist and LATEX is used to
process these labels.  

prologues:=3; 
filenametemplate "%j_%c.mps"; 
verbatimtex 
%&latex 
\documentclass{minimal} 
\begin{document} 
etex 
beginfig(0); 
  � 
endfig; 
end 

As of version 0.970 of MetaPost, the prologues:=3 feature provides embedding
of used fonts into the MetaPost output. I generally set this variable when
desigining my graphics, but when I include them into my LATEX document, I
remove this line since LATEX will include the fonts itself anyway. Also, as of
version 0.920 of MetaPost, the filenametemplate allows the user to set the
naming scheme of MetaPost graphics. The graphicx package identifies a graphic
with a .mps file extension as originating from MetaPost. If the example code
above was saved as foo.mp, then the output would be foo_0.mps and the command 

\includegraphics{foo_0.mps}

can be processed by both LATEX and PDFLATEX (thanks to Hans Hagan�s mptopdf).
Both of the aforementioned new features of MetaPost are available in TEXLive 2007
as well as MiKTEX 2.6. 

8  Conclusion

Vector graphics should always be used whenever the graphic is not raster in
nature. The vector formats that I prefer are EPS and PDF depending on whether
LATEX or PDFLATEX is used to typeset the document. If the graphic is
characteristically raster, then the optimal method for vectorizing it usually
involves placing a EPS or PDF wrapper around it using sam2p. Taking these steps
to place graphics in a LATEX document requires a slightly more than na�ve
solution, but the results should be of higher quality. 

